\section{Hypothesis testing in multilayer models}
In this section, we lay out a framework for hypothesis testing in our proposed joint multilayer structure. Present literature in high-dimensional hypothesis testing either focuses on testing for simularities in the within-layer connections of single-layer networks \citep{CaiLiu16,Liu17}, or coefficients of single response penalized regression \citep{vanDeGeerEtal14,ZhangZhang14,MitraZhang16}. However a method for testing {\it between-layer} connections in a multilayer setup is yet to be proposed.

There are two main challenges in doing the above: firstly the need to mitigate estimation the bias of estimators that are obtained from lasso and group lasso-based procedures and assumptions on the design matrix required for the same, and secondly the dependency among response nodes translating into the need for controlling False Discovery Rate (FDR) while simultaneously testing for such hypotheses. In Section~\ref{sec:testing-subsec-1} we propose a debiased estimator for rows of the coefficient matrix estimates $\bfB^k$ that makes use of already computed (using JSEM) nodewise regression coefficients in the upper layer, and establish asymptotic properties of scaled version of them. Section~\ref{sec:testing-subsec-2} is devoted to pairwise testing, where we asssume $K=2$, and propose asymptotic global tests for detecting differential effects of a variable in the upper layer, as well as pairwise simultaneous tests for detecting elementwise difference in the coefficient matrices across $k$.

%Suppose there are two disease subtypes: $k = 1,2$, and we are interested in testing whether the downstream effect of a predictor in X-data is same across both subtypes, i.e. if $\bfb_{0i}^1 = \bfb_{0i}^2$ for some fixed $i \in \cI_p$. For this we consider the modified optimization problem:
%%
%\begin{align}
%& \min_{\cB, \Theta} \frac{1}{n} \left\{ \sum_{j=1}^q \sum_{k=1}^2 \| \bfY_j^k - \bfY_{-j}^k \bftheta_j^k - \bfX^k \bfB_{j}^k \|^2 + \sum_{j \neq j'} \lambda_{jj'} \| \bftheta_{jj'}^* \| + \sum_{i=1}^p \eta_i \| \bfB_{i*}^* \| \right\} \notag\\
%&= \min \left\{ f ( \cY, \cX, \cB, \Theta) + P (\Theta) + Q (\cB) \right\} 
%\end{align}
%%
%with $n_1 = n_2 = n$ for simplicity; and $\bfB^k = (\bfb_1^k, \ldots, \bfb_q^k), (\bfB_{i*}^*) \in \BR^{ q \times K}$.

\subsection{Debiased estimators and asymptotic normality}
\label{sec:testing-subsec-1}
\cite{ZhangZhang14} proposed a debiasing procedure for lasso estimates and subsequently calculate confidence intervals for individual coefficients $\beta_j$ in high-dimensional linear regression: $\bfy = \bfX \bfbeta + \bfepsilon, \bfy \in \BR^n, \bfX \in \BM(n,p)$ and $\epsilon_r \sim N(0,\sigma^2), r \in \cI_n$ for some $\sigma>0$. Given an initial lasso estimate $\widehat \bfbeta^{(init)} \in \BR^p$ their debiased estimator was defined as:
%
$$
\hat \beta_j^{(\text{deb})} = \hat \beta_j^{(\text{init})} + \frac{\bfz_j^T ( \bfy - \bfX \bfbeta^{(\text{init})})}{\bfz^T \bfx_j}
$$
%
where $\bfz_j$ is the vector of residuals from $\ell_1$-penalized regression of $\bfx_j$ on $\bfX_{-j}$. With centering around the true parameter value, say $\beta_j^0$, and proper scaling this has an asymptotic normal distribution:
%
$$
\frac{\hat \beta_j^{(\text{deb})} - \beta_j^0}{\| \bfz_j \|/| \bfz_j^T \bfx_j |} \sim N(0, \sigma^2)
$$
%
Essentially, they obtain the debiasing factor for the $j^{\Th}$ coefficient by taking residuals from the regularized regression and scale them using the projection of $\bfx_j$ onto a space approximately orthogonal to it. \cite{MitraZhang16} later generalized this idea to group lasso estimates. Further, \cite{vanDeGeerEtal14} and \cite{JavanmardMontanari14} performed debiasing on the entire coefficient vectors.

We start off by defining debiased estimates for individual rows of the coefficient matrices $\bfB^k$ in our two-layer model:
%
\begin{align}\label{eqn:DebiasedBeta}
\widehat \bfc_i^k = \widehat \bfb_i^k + \frac{1}{n t_i^k} \left( \bfX_i^k - \bfX_{-i}^k \widehat \bfzeta_i^k \right)^T (\bfY^k - \bfX^k \widehat \bfB^k )
; \quad i \in \cI_p, k \in \cI_K
\end{align}
%
where $t_i^k = ( \bfX_i^k - \bfX_{-i}^k \widehat \bfzeta_i^k )^T \bfX_{-i}^k/n$, and $\widehat \bfzeta_i^k, \widehat \bfB^k$ are {\it generic estimators} of the neighborhood coefficient matrices in the upper layer and within-layer coefficient matrices, respectively. By structure this is similar to the proposal of \cite{ZhangZhang14}. However, as we see shortly, minimal conditions need to be imposed on the parameter estimates used in \eqref{eqn:DebiasedBeta} for the asymptotic results based on a scaled version of the debiased estimator to go thorugh, and they continue to hold for arbitrary sparsity patterns over $k$ in all the parameters.

Present methods of debiasing coefficients from regularized regression require specific assumptions on the regularization structure of the main regression, as well as on how to calculate the debiasing factor. While \cite{ZhangZhang14}, \cite{JavanmardMontanari14} and \cite{vanDeGeerEtal14} work on coefficients from lasso regressions, \cite{MitraZhang16} debias the coefficients of pre-specified groups in the coefficient vector from a group lasso. The current proposals for the debiasing factor available in the literature include nodewise lasso \citep{ZhangZhang14} and a variance minimization scheme with $\ell_\infty$-constraints \citep{JavanmardMontanari14}. In comparison, we only assume the following generic constraints on the parameter estimates used in our procedure.

\vspace{1em}
\noindent{\bf (T1)} For the upper layer neighborhood coefficients, the following holds for all $k \in \cI_K$:
%
$$
\| \widehat \bfzeta^k - \bfzeta_0^k \|_1 \leq C_\zeta \sqrt { \frac{\log p}{n}}
$$
%
where $C_\zeta = O(1)$ depends only on the true values, i.e. $\{ \zeta^k_0 \}$.

\noindent{\bf (T2)} The lower layer precision matrix estimators satisfy for all $k \in \cI_K$
%
$$
\left\| (\widehat \Omega_y^k)^{1/2} - (\Omega_y^k)^{1/2} \right\|_\infty \leq C_\Omega \sqrt { \frac{\log q}{n}}
$$
%
where $C_\Omega = O(1)$ depends only on $\Omega_{y 0}$.

\noindent{\bf (T3)} For the regression coefficient matrices, the following holds for all $k \in \cI_K$:
%
$$
\| \widehat \bfB^k - \bfB^k_0 \|_1 \leq C_\beta \sqrt { \frac{\log (p q)}{n}}
$$
%
where $C_\beta = O(1)$ depends on $\cB$ only.
\vspace{1em}

Given these conditions, the following theorem provides the asymptotic joint distribution of a scaled version of the debiased coefficients. A similar result for fixed design in the context of single-response linear regression can be found in the preprint by \cite{StuckyVandeGeer17}. However they use nuclear norm as the loss function while obtaining the debiasing factors and use the resulting Karush-Kuhn-Tucker (KKT) conditions to derive their results, whereas we leverage bounds on generic parameter estimates combined with the sub-gaussianity of our design matrices.

\begin{Theorem}\label{Thm:ThmTesting}
Define $\widehat s_i^k = \sqrt{\| \bfX_i^k - \bfX_{-i}^k \widehat \bfzeta_i^k \|^2/n}$, and $m_i^k = \sqrt n t_i^k / \widehat s_i^k$. Consider parameter estimates that satisfy conditions (T1)-(T3). Define the following:
%
\begin{align*}
\widehat \Omega_y &= \diag(\widehat \Omega_y^1, \ldots, \widehat \Omega_y^K)\\
\bfM_i &= \diag(m_i^1, \ldots, m_i^K)\\
\bfC_i &= \diag(\widehat \bfc_i^1, \ldots, \widehat \bfc_i^K)\\
\bfD_i &= \diag(\bfb_{0,i}^1, \ldots, \bfb_{0,i}^K)\\
\end{align*}
%
Then for sample size satisfying $\log p = o(n^{1/2}), \log q = o(n^{1/2})$ we have
%
\begin{align}\label{eqn:ThmTestingEqn}
\widehat \Omega_y^{1/2} \bfM_i (\bfC_i - \bfD_i) \sim
\cN_{Kq} ({\bf 0}, \bfI) + \bfR_n
\end{align}
%
where $\| \bfR_n \|_\infty = o_P(1)$.
\end{Theorem}
%

\subsection{Test formulation}
\label{sec:testing-subsec-2}
Now we simply plug-in estimators from the JMMLE algorithm in Theorem~\ref{Thm:ThmTesting}. This is fairly straightforward. Condition (T1) is ensured by the JSEM penalized neighborhood estimators (immediate from Proposition A.1 in \cite{MaMichailidis15}), and a bound on total sparsity of the true coefficient matrices: $B = o(n/ \log(pq))$, in conjunction with Corollary~\ref{corollary:jmmle-final}, ensures condition (T3), both with probability approaching 1 as $(n,p,q) \rightarrow \infty$. Finally, condition (T2) follows from Corollary~\ref{corollary:jmmle-final}, since {\colrbf tbd}.

An asymptotic joint distribution of debiased versions of the JMMLE regression estimates is now immediate.
%
\begin{Corollary}\label{corollary:CorTesting}
Consider the estimates $\widehat \cB$ and $\widehat \Omega_y $ obtained from Algorithm~\ref{algo:jmmle-algo}, and upper layer neighborhood coefficients from solving the nodewise regression in \eqref{eqn:jsem-model}. Suppose that $\log (pq) /\sqrt n \rightarrow 0$, and the sparsity condition $B = o(n / \log(pq))$ is satisfied. Then, with the same notations in Theorem~\ref{Thm:ThmTesting} we have
%
\begin{align}\label{eqn:CorTestingEqn}
\widehat \Omega_y^{1/2} \bfM_i (\bfC_i - \bfD_i) \sim
\cN_{Kq} ({\bf 0}, \bfI) + \bfR_{1n}
\end{align}
%
where $\| \bfR_{1n} \|_\infty = o_P(1)$.
\end{Corollary}

We are now ready to formulate asymptotic global and simultaneous testing procedures based on Corollary~\ref{corollary:CorTesting}. In this paper, we restrict our attention to testing for pairwise differences only. Specifically, we set $K=2$, and are interested in testing whether there are overall and elementwise differences between the coefficient vectors $\bfb_{0i}^1$ and $\bfb_{0i}^2$.

When $\bfb_{0 i}^1 = \bfb_{0 i}^2$, it is immediate from Corollary~\ref{corollary:CorTesting} that a scaled version of the vector of estimated differences $\widehat \bfc_i^1 - \widehat \bfc_i^2$ follows a $q$-variate multinormal distribution. Consequently we formulate a global test for detecting differential overall downstream effect of the $i^{\Th}$ covariate in the upper layer.

\begin{Algorithm}\label{algo:AlgoGlobalTest}
(Global test for $H_0^i: \bfb_{0 i}^1 = \bfb_{0 i}^2$ at level $\alpha, 0< \alpha< 1$)

\noindent 1. Obtain the debiased estimators $\widehat \bfc_i^1, \widehat \bfc_i^2$ using \eqref{eqn:DebiasedBeta}.

\noindent 2. Calculate the test statistic
%
$$
D_i = (\widehat \bfc_i^1 - \widehat \bfc_i^2)^T
\left( \frac{ (\widehat \Omega_y^1)^{-1}}{(m_i^1)^2} +
\frac{(\widehat \Omega_y^2)^{-1}}{(m_i^2)^2} \right)^{-1} (\widehat \bfc_i^1 - \widehat \bfc_i^2)
$$
%

\noindent 3. Reject $H_0^i$ if $D_i \geq \chi^2_{q, 1-\alpha}$.
\end{Algorithm}

Besides controlling the type-I error at a specified level $\alpha$, the above testing procedure maintains rate optimal power.

\begin{Theorem}
Consider the global test given in Algorithm~\ref{algo:AlgoGlobalTest}, performed using parameter estimates satisfying conditions (T1)-(T3). Say $\bfdelta := \bfb_{0 i}^1 - \bfb_{0 i}^2$. Then the power of the test is given by
%
$$
K_q \left( \chi^2_{q,1-\alpha} + \bfdelta^T 
\left( \frac{ (\Omega_{y 0}^1)^{-1}}{(m_{0 i}^1)^2} + \frac{(\Omega_{y 0}^2)^{-1}}{(m_{0 i}^2)^2}
+ o(1) \right)^{-1} \bfdelta \right)
$$
%
where $K_q$ is the cumulative distribution function of the $\chi^2_q$ distribution, and $m_{0 i}^k$ is the population version of $m_i^k, k=  1,2$. Consequently, for $\| \bfdelta \|_\infty \geq {\colrbf tbd}$, $P( H_0^i \text{ is rejected }) \rightarrow 1$ as $(n,p,q) \rightarrow \infty$.
\end{Theorem}
%

Given the null hypothesis is rejected, we now consider the multiple testing problem of simultaneously testing for all entrywise differences, i.e. testing
%
$$
H_0^{ij}: b_{0 ij}^1 = b_{0ij}^2 \quad \text{vs.} \quad H_1^{ij}: b_{0 ij}^1 \neq b_{0 ij}^2 
$$
%
for $j \in \cI_q$. Here we use the test statistic
%
\begin{align}\label{eqn:PairwiseStatistic}
d_{ij} &= \frac{\widehat c_{ij}^1 - \widehat c_{ij}^2}{\sqrt{\tau_{ij}^1/ (m_i^1)^2 + \tau_{ij}^2/ (m_i^2)^2}}
\end{align}
%
with $\tau_{ij}^k$ being the $(i,j)^{\Th}$ element of $( \widehat \Omega_y^k)^{-1}$, for $k = 1,2$.

Now consider tests where $H_0^{ij}$ is rejected if $| d_{ij} | > \tau$. We denote $\cH_0^i = \{ j: b_{ij}^1 = b_{ij}^2 \}$ and define the false discovery proportion (FDP) and false discovery rate (FDR) for these tests as follows:
%
$$
FDP (\tau) = \frac{\sum_{j \in \cH_0^i} \BI( |d_{ij}| \geq \tau)}{\max\left\{
\sum_{j \in \cI_q} \BI( |d_{ij}| \geq \tau), 1\right\} }\quad
FDR (\tau) = \BE [ FDP (\tau) ]
$$
%
For a pre-specified level $\alpha$, we choose a threshold that ensures both FDP and FDR $\leq \alpha$ using the Benjamini-Hochberg (BH) procedure. % To do this we define the following:
%%
%\begin{align*}
%P_0 = 2 \Phi (1) - 1; \quad
%\hat P_0 = \frac{1}{q} \sum_{j \in \cH_0^i} \BI (|d_{ij}| \leq 1); \quad
%Q_0 = \sqrt 2 \phi (1);\\
%A = \frac{P_0 - \hat P_0}{Q_0}; \quad A(t) = \left[ 1 + |A| \frac{|t| \phi(t)}{\sqrt 2 (1 - \Phi(t))} \right]^{-1}
%\end{align*}
%%
%where $\Phi(\cdot)$ and $\phi(\cdot)$ are the standard normal distribution and density functions, respectively.   
The procedure for FDR control is now given by Algorithm \ref{algo:AlgoFDR}.

\begin{Algorithm}\label{algo:AlgoFDR}
(Simultaneous tests for $H_0^{ij}: b_{0 ij}^1 = b_{0 ij}^2$ at level $\alpha, 0< \alpha< 1$)

\noindent 1. Calculate the pairwise test statistics $d_{ij}$ using \eqref{algo:AlgoFDR} for $j \in \cI_q$.

\noindent 2. Obtain the threshold
%
$$
\hat \tau = \inf \left\{\tau \in \BR: 1 - \Phi(\tau) \leq \frac{\alpha}{2 q}
\max \left( \sum_{j \in \cI_q} \BI( |d_{ij}| \geq \tau), 1 \right) \right\}
$$
%

\noindent 3. For $j \in \cI_q$, reject $H_0^{ij}$ if $|d_{ij}| \geq \hat \tau$.
\end{Algorithm}

%Denote $\hat \cH_0^i = \{ j\ in \cI_q: |d_{ij}| \geq \hat \tau \}$.
This procedure maintains FDR and FDP asymptotically at a pre-specified level $\alpha \in (0,1)$  under weak dependence conditions.

\begin{Theorem}\label{thm:FDRthm}
Suppose $\mu_j = b_{0,ij}^1 - b_{0,ij}^2, \sigma_j^2 = n \BE (\tau_{ij}^1/ (m_i^1)^2 + \tau_{ij}^2/ (m_i^2)^2)$. Assume the following holds as $(n,q) \rightarrow \infty$
%
\begin{align}\label{eqn:FDRthmEqn1}
\left| \left\{ j \in \cI_q: |\mu_j / \sigma_j | \geq
4 \sqrt{ \log q/n} \right\} \right| \rightarrow \infty
\end{align}
%
Now Consider the conditions {\colrbf C1, C1*}


If (C1) is satisfied, then the following holds when $\log q = O(n^{\xi}), 0 < \xi < 3/23$:
%
\begin{align}\label{eqn:FDReqn}
\frac{FDP( \hat \tau)}{(| \cH_0^i|/q) \alpha} \stackrel{P}{\rightarrow} 1; \quad
\lim_{n, q \rightarrow \infty} \frac{FDR( \hat \tau)}{(| \cH_0^i|/q) \alpha} = 1
\end{align}
%
Further, if (C1*) is satisfied, then \eqref{eqn:FDReqn} holds for $\log q = o(n^{1/3})$.
\end{Theorem}
%
%Theorem~\ref{thm:FDRthm} is essentially a restatement of Theorem 4.1 in \cite{LiuShao14}.
The condition \eqref{eqn:FDRthmEqn1} is essential for FDR control in a diverging parameter space \citep{LiuShao14, Liu17}, while the dependence conditions (C1) and (C1*) control the amount of correlation between variables in the Y-layer.

\begin{Remark}
Following \cite{LiuShao14}, a version of Algorithm~\ref{algo:AlgoFDR}, where the null distribution is calibrated using bootstrap instead of normal approximation, gives asymptotic FDR control under (C1*) and $\log q = o(n^{1/2})$. We believe it is possible to obtain \eqref{eqn:FDReqn} under the weaker condition (C1) for $ \log q = o(n^{1/2})$ by extending the framework of \cite{Liu17} that performs multiple testing in multiple (single layer) GGMs, with the added advantage of being generalizable to the case of $K > 2$. However, this requires a significant amount of theoretical analysis, and we leave it for future research.
\end{Remark}

\begin{Remark}
{\colrbf does within-group thresholding with FDR control for $K=1$}
\end{Remark}