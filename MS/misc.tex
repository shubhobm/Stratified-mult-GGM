\section{Proofs of main results}
%{\colrbf need modification}

\begin{proof}[Proof of Theorem~\ref{thm:algo-convergence}]
The theorem is a generalization of Theorem 1 in \cite{LinEtal16}. The proof follows directly from the proof of that theorem, substituting $( \widehat B^{(k)}, \widehat \Theta_\epsilon^{(k)}), (B^*, \Theta_\epsilon^*)$ and $ (B^\infty, \Theta_\epsilon^\infty)$ therein with $(\widehat \cB^{(t)}, \widehat \Theta_y^{(t)}), ( \cB_0, \Theta_{y 0})$ and $ (\cB^\infty, \Theta_y^\infty)$, and corresponding variations as required.

\end{proof}

We use the following condition extensively in proving the results that follow.

\vspace{1em}
\noindent {\bf Condition 3} (Restricted eigenvalues). A symmetric matrix $\bfM \in \BM(b,b)$ is said to satisfy the restricted eigenvalue or RE condition with parameters $\psi, \phi >0$, denoted as curvature and tolerance, respectively, if
%
$$
\bftheta^T \bfM \bftheta \geq \psi \| \bftheta \|^2 - \phi \| \bftheta \|_1^2
$$
%
for all $\bftheta \in \BR^b$. In short, this is denoted by $\bfM \sim RE(\psi, \phi)$.

Starting from \cite{BickelRitovTsybakov09}, different versions of the RE conditions have been proposed and used in high-dimensional analysis \citep{LohWainwright12,BasuMichailidis15,MaMichailidis15,vandeGeerBuhlmann09} to ensure that a covariance matrix satisfies a somewhat relaxed positive-definitess condition.

\begin{proof}[Proof of Theorem~\ref{thm:thm-Theta}]
The proof strategy is as follows. We first show that given fixed $(\cX,\cE)$, and some conditions on $\widehat \bfE^k := \bfY^k - \bfX^k \widehat \bfB^k, k \in \cI_K$, the bounds in in Theorem~\ref{thm:thm-Theta} hold. We then show that for random $(\cX,\cE)$, those conditions hold with probability approaching 1.

\begin{Lemma}\label{thm:ThetaThm}
Assume fixed $\cX, \cE$ and deterministic $\widehat \cB = \{ \widehat \bfB^k \}$, and the following conditions.

\noindent{\bf(A1)} For $k \in \cI_K$,
%
$$
\| \widehat \bfB^k - \bfB^k_0 \|_1 \leq C_\beta \sqrt{\frac{\log (pq)}{n}}
$$
%
with $C_\beta = O(1)$ is non-negative and depends on $\cB$ only.

%\noindent{\bf(T2)} $\| \bfX^k (\widehat \bfB^k - \bfB^k_0 ) \|_\infty \leq c(v_\beta)$, where $c(v_\beta)$ is $O(1)$ and depends on $v_\beta$.

\noindent{\bf(A2)} For all $j \in \cI_q$,
%
$$
\frac{1}{n} \left\| (\widehat \bfE_{-j}^k)^T \widehat \bfE^k \bfT_{0,j}^k \right\|_\infty \leq
\BQ \left(C_\beta, \Sigma_x^k, \Sigma_y^k \right) \sqrt {\frac{ \log (p q)}{n}},
$$
%
where $\BQ \left(C_\beta, \Sigma_x^k, \Sigma_y^k \right) = O(1)$ is non-negative and depends on $\cB, \Sigma_x^k$ and $\Sigma_y^k$ only.

\noindent{\bf(A3)} Denote $\widehat \bfS^k = (\widehat \bfE^k)^T \widehat \bfE^k/n$. Then $\widehat \bfS^k \sim RE(\psi^k, \phi^k)$ with $Kq \phi \leq \psi/2$ where $ \psi = \min_k \psi^k, \phi = \max_k \phi^k $.

%\noindent{\bf(T4)} Assumption (A2) holds for $\Sigma_y^k$.

Then the following hold

\noindent (I) Given the choice of tuning parameter
%
$$
\gamma_n \geq 4 \sqrt{| g_{\max}|} \BQ_0 \sqrt {\frac{ \log (p q)}{n}}; \quad
\BQ_0 := \max_{k \in \cI_K} \BQ \left(C_\beta, \Sigma_x^k, \Sigma_y^k \right)
$$
%
\begin{align}
\| \widehat \Theta_j - \Theta_{0,j} \|_F & \leq 12 \sqrt{s_j} \gamma / \psi, \label{eqn:theta-norm-bound-1a}\\
\sum_{j \neq j', g \in \cG_y^{jj'}} \| \hat \bftheta_{jj'}^{[g]} - \bftheta_{0,jj'}^{[g]} \| & \leq 48 s_j \gamma / \psi. \label{eqn:theta-norm-bound-2a}
\end{align}
%

\noindent (II) For the choice of tuning parameter $\gamma = 4 \sqrt{| g_{\max}|} \BQ_0 \sqrt{\log (p q)/n}$,
%
\begin{align}\label{eqn:OmegaBounds0a}
\frac{1}{K} \sum_{k=1}^K \| \widehat \Omega_y^k - \Omega_y^k \|_F \leq
O \left( \BQ_0 \sqrt{\frac{| g_{\max}| S}{K}} \right)
\end{align}
%
%Further if A1 holds with $s = s_0$, and A3 is satisfied then
%
%(II) Direction consistency.
\end{Lemma}

Condition (A1) holds by assumption. When $\cX$ and $\cE$ are random, the following proposition ensures that (A2) and (A3) hold with probabilities approaching to 1.

\begin{Lemma}\label{prop:ErrorRE}
Consider deterministic $\widehat \cB$ satisfying assumption (A1). Then for sample size $n \succsim \log (pq)$ and $k \in \cI_K$,

\begin{enumerate}
\item $\widehat \bfS^k$ satisfies the RE condition: $ \widehat \bfS^k \sim RE (\psi^k, \phi^k)$, where 
%
$$
\psi^k = \frac{ \Lambda_{\min} (\Sigma_x^k)}{2}; \quad \phi^k = \frac{ \psi^k \log p}{n} + 2 C_\beta c_2 [ \Lambda_{\max} (\Sigma_x^k) \Lambda_{\max} (\Sigma_y^k) ]^{1/2} \frac{ \log(pq)}{n}
$$
%
with probability $\geq 1 - 6c_1 \exp [-(c_2^2-1) \log(pq)] - 2 \exp (- c_3 n), c_1, c_3 > 0, c_2 > 1$.
%
\item The following deviation bound is satisfied for any $j \in \cI_q$
%
$$
\left\|\frac{1}{n} (\widehat \bfE_{-j}^k)^T \widehat \bfE^k \bfT_{0,j}^k \right\|_\infty \leq \BQ \left(C_\beta, \Sigma_x^k, \Sigma_y^k \right) \sqrt {\frac{ \log (p q)}{n}}
$$
%

with probability $\geq 1 - 1/p^{\tau_1-2} - 6c_1 \exp [-(c_2^2-1) \log(pq)] - 6c_4 \exp [-(c_5^2-1) \log(pq)], c_4 > 0, c_5 > 1, \tau_1 > 2$, where
%
\begin{align*}
\BQ \left(C_\beta, \Sigma_x^k, \Sigma_y^k \right) &=
\left[ 2 C_\beta^2 V_x^k + 4 C_\beta c_2 [ \Lambda_{\max} (\Sigma_x^k) \Lambda_{\max} (\Sigma_y^k) ]^{1/2} \right] \sqrt{\frac{ \log(pq)}{n}} +\\
& c_5 \left[ \Lambda_{\max} ( \Sigma_{y,-j}^k) \sigma_{y,j,-j}^k \right]^{1/2} \sqrt{\frac{\log q}{\log (pq)}}
\end{align*}
%
with $\sigma_{y,j,-j}^k = \BV( E_j - \BE_{-j} \bftheta_{0,j})$, and 
%
$$
V_x^k = \sqrt{ \frac{ \log 4 + \tau_1 \log p}{c_x^k n}} + \max_i \sigma_{x,ii}^k; \quad
c_x^k = \left[ 128 ( 1 + 4 \Lambda_{\max} (\Sigma_x)  )^2 \max_i (\sigma_{x,ii})^2 \right]^{-1}
$$
\end{enumerate} 
\end{Lemma}

\noindent We prove the main theorem by putting together Lemma~\ref{thm:ThetaThm} and Lemma~\ref{prop:ErrorRE}.
\end{proof}

\begin{proof}[Proof of Theorem~\ref{thm:thm-B}]
%The proof strategy is as follows. We first show that given fixed $(\cX,\cE)$, and some conditions on the quantities $\widehat \bfgamma, \widehat \bfGamma$, the bounds in in \ref{thm:thm-B} hold. We then show that for random $(\cX,\cE)$, these conditions hold with probability approaching 1.

\begin{Lemma}\label{thm:BetaThm}
Assume fixed $(\cX, \cE)$, and deterministic $\widehat \Theta = \{ \widehat \Theta_j \}$, so that

\noindent{\bf(A1)} For $j \in \cI_q$,
%
$$
\| \widehat \Theta_j - \Theta_{0,j} \|_F \leq C_\Theta \sqrt{\frac{\log q}{n}},
$$
%
for some $C_\Theta = O(1)$ dependent on $\Theta$ only.

\noindent{\bf(A2)} Denote $\widehat \bfGamma^k = (\widehat \bfT^k)^2 \otimes (\bfX^k)^T \bfX^k/n, \widehat \bfgamma^k = (\widehat \bfT^k)^2 \otimes (\bfX^k)^T \bfY^k/n$. Then the deviation bound holds:
%
$$
\left\| \widehat \bfgamma^k - \widehat \bfGamma^k \bfbeta_0 \right\|_\infty \leq \BR( C_\Theta, \Sigma_x^k, \Sigma_y^k) \sqrt{ \frac{ \log(pq)}{n}}.
$$
%
where $\BR (C_\Theta, \Sigma_x^k, \Sigma_y^k ) = O(1)$ depends on $\Theta, \Sigma_x^k$ and $\Sigma_y^k$ only.

\noindent{\bf(A3)} $\widehat \bfGamma \sim RE(\psi_*, \phi_*)$ with $Kpq \phi_* \leq \psi_*/2$.

Then, given the choice of tuning parameter
%
$$
\lambda_n \geq 4 \sqrt{| h_{\max} |} \BR_0 \sqrt{ \frac{ \log(pq)}{n}}; \quad 
\BR_0 := \max_{k \in \cI_K} \BR \left(C_\Theta, \Sigma_x^k, \Sigma_y^k \right)
$$
%
the following holds
%
\begin{align}
\| \widehat \bfbeta - \bfbeta_0 \|_1 & \leq 48 \sqrt{ | h_{\max} |} s_\beta \lambda_n / \psi^* \label{eqn:BetaThmEqn1}\\
\| \widehat \bfbeta - \bfbeta_0 \| & \leq 12 \sqrt s_\beta \lambda_n / \psi^* \label{eqn:BetaThmEqn2}\\
\sum_{h \in \cH} \| \bfbeta^{[h]} - \bfbeta_0^{[h]} \| & \leq 48 s_\beta \lambda_n / \psi^* \label{eqn:BetaThmEqn3}\\
(\widehat \bfbeta - \bfbeta_0 )^T \widehat \bfGamma (\widehat \bfbeta - \bfbeta_0 ) & \leq
72 s_\beta \lambda_n^2 / \psi^* \label{eqn:BetaThmEqn4}
\end{align}
\end{Lemma}

Condition (A1) holds by assumption. Next we verify that conditions (A2) and (A3) hold with high probability given fixed $\widehat \Theta$.

\begin{Lemma}\label{prop:ThmBetaRE}
Consider deterministic $\widehat \Theta$ satisfying assumption (A1). Also assume conditions (E1), (E2) from the main paper. Then for sample size $n \succsim \log (pq)$,

\begin{enumerate}
\item $\widehat \bfGamma$ satisfies the RE condition: $ \widehat \bfGamma \sim RE (\psi_*, \phi_*)$, where 
%
$$
\psi_* = \min_k \psi^k \left( \min_i \psi_k^j - d_k C_\Theta \right), 
\phi_* = \max_k \phi^k \left( \min_i \phi_k^j + d_k C_\Theta \right)
$$
%
with probability $\geq 1 - 2 \exp(c_3 n), c_3>0$.
%
\item The deviation bound in (A2) is satisfied with probability $ \geq 1 - 12 c_1 \exp[ (c_2^2-1) \log (pq)], c_1>0, c_2>1$, where
$$
\BR \left(C_\Theta, \Sigma_x^k, \Sigma_y^k \right) = c_2 \sqrt{\Lambda_{\max} (\Sigma_x^k)} \left( d_k C_\Theta \Lambda_{\min} (\Sigma_y^k) +
\frac{1}{\Lambda_{\min} (\Sigma_y^k) } \right)
$$
\end{enumerate}
\end{Lemma}

The theorem is immediate by putting together Lemma~\ref{thm:BetaThm} and Lemma~\ref{prop:ThmBetaRE}.
\end{proof}

{\colrbf Discuss tighter bound compared to vanilla JSEM}

After providing the error bounds for solutions to the subproblem \eqref{eqn:EstEqn2}, we concentrate on the subproblem \eqref{eqn:EstEqn1}. Following a similar strategy, we first get error bounds for $\widehat \bfbeta$ assuming everything else fixed.

%with
%$$
%\bfbeta = \begin{bmatrix}
%\ve (\bfB^1)\\
%\vdots\\
%\ve (\bfB^K)\\
%\end{bmatrix}; \quad
%\bfGamma = \begin{bmatrix}
%I_q \otimes (\bfX^1) TX^1 / n) & &\\
%& \ddots &\\
%& & I_q \otimes (\bfX^K)^T X^K / n)
%\end{bmatrix} 
%$$
\begin{proof}[Proof of Theorem~\ref{thm:ThetaThm}]
The proof has three parts, where we prove the consistency of the neighborhood regression coefficients, selection of edge sets, and finally the refitting step, respectively. This is the same structure as the proof of Theorem 1 in \cite{MaMichailidis15}, where they prove consistency of the (single layer) JSEM estimates. The derivation of the first part is different from that in the JSEM proof, which we shall show in detail (in the proof of Proposition~\ref{prop:Thm1ProofProp2}). The second and third parts follow similar lines, incorporating the updated quantities from the first part. For these we provide outlines and leave the details to the reader.

%Define $\bfT_{0,j}^k$ the same way as $\bfT_j^k$. For any $g \in \cG^{jj'}, k \in g$, and $j \neq j'$, let
%%
%$$
%\widehat \bfepsilon_j^k = (\bfY^k - \bfX^k \widehat \bfB^k ) \bfT_{0,j}^k; \quad
%\widehat \zeta_{jj'}^k = \frac{(\widehat \bfepsilon_j^k)^T \bfY_{j'}^k}{n}; \quad
%\widehat \bfzeta_{jj'}^{[g]} = (\widehat \zeta_{jj'}^k)_{k \in g}
%$$
%%
%Consider the random event $\cA = \bigcap_{j, j'\neq j, g} \cA_{jj'}^g$ with $\cA_{jj'}^g = \{ 2 \| \widehat \bfzeta_{jj'}^{[g]} \| \leq \lambda_{jj'}^g \}$.
%
%\begin{Lemma}\label{lemma:LemmaE2}
%Given that $\lambda_{jj'}^g$ are chosen as
%%
%$$
%\lambda_{jj'}^g \geq \max_{k \in g} \frac{2}{\sqrt{n \omega_{jj}^k}} \left( \sqrt{|g| (1 + 2 c(v_\beta)) } + \frac{\pi}{\sqrt 2} \sqrt {r \log G_0} \right)
%$$
%%
%we shall have $ \BP (\cA) \geq 1 - 2p G_0^{1-q} $ for some $r>1$.
%\end{Lemma}
%
%\begin{proof}[Proof of Lemma~\ref{lemma:LemmaE2}]
%We follow the proof of Lemma E.2 in 15-656, with $\bfY_j^k, \widehat \bfepsilon_j^k, \widehat \zeta_{jj'}^k, \widehat \bfzeta_{jj'}^{[g]}$ in place of $\bfX_j^k, \bfepsilon_i^k, \zeta_{ij}^k, \bfzeta_{ij}^{[g]}$ respectively. Proceeding in a similar fashion we get
%%
%$$
%\| \widehat \bfzeta_{jj'}^{[g]} \|^2 = \frac{1}{n} \left[ \| \bfZ^{[g]} \|^2 + \sum_{k \in g} \left\{ 2 Z^k (\bfQ_{j'}^k)^T \bfdelta_j^k + | (\bfQ_{j'}^k)^T \bfdelta_j^k |^2 \right\} \right]
%$$
%%
%where $\bfZ^{[g]} = (Z^k)_{k \in g}; Z^k = (\bfQ_{j'}^k)^T \bfepsilon_j^k$ with $\bfepsilon_j^k :=(\bfY^k - \bfX^k \bfB_0^k ) \bfT_{0,j}^k$, $\bfQ_j^k$ is the first eigenvector of $\bfY_j^k (\bfY_j^k)^T/n$, and $\bfdelta_j^k := \bfX^k (\bfB_0^k - \widehat \bfB^k) \bfT_{0,j}^k$.
%
%By cauchy-schwarz inequality, $| (\bfQ_{j'}^k)^T \bfdelta_j^k | \leq \| \bfdelta_j^k \| \leq \| \bfX^k (\bfB_0^k - \widehat \bfB^k) \|_\infty \| \bfT_{0,j} \|_1$. Now since $\Sigma_y^k$ is diagonally dominant,
%%
%$$
%\sum_{j \neq j'} |T_{0,jj'}^k| =  \sum_{j \neq j'} |\theta_{0,jj'}^k| = \sum_{j \neq j'} \frac{|\sigma_{y,jj'}^k|}{ \sigma_{y,jj}^k} \leq 1
%$$
%%
%Also $T_{0,jj}^k = 1$, so that $\| \bfT_{0,j}^k \|_1 \leq 2$. Thus $\| \bfdelta_j^k \| \leq 2 \| \bfX^k (\bfB_0^k - \widehat \bfB^k) \|_\infty$. Hence by assumption (T2),
%%
%$$
%\| \widehat \bfzeta_{jj'}^{[g]} \| \leq \frac{1}{\sqrt n} ( \| \bfZ^{[g]} \| + 2 |g| c(v_\beta))
%$$
%%
%so that
%$$
%\BP ( \{ \cA_{jj'}^g \}^c ) = \BP \left( \| \widehat \bfzeta_{jj'}^{[g]} \| > \frac{\lambda_{jj'}^g}{2} \right) \leq \BP \left( \| \bfZ^{[g]} \|  > \frac{\sqrt n \lambda_{jj'}^g}{2} - 2 |g| c(v_\beta)  \right)
%$$
%We now proceed through the proof of Lemma E.2 in 15-656 to end up with the choice of $\lambda_{ij}^g$.
%\end{proof}
%

%All subsequent derivations in the theorem go through with the new choice of $\lambda_{ij}^g$.

{\it Step 1: consistency of neighborhood regression.} The following proposition establishes error bounds for estimated neighborhood coefficients in the Y-network.

\begin{Proposition}\label{prop:Thm1ProofProp2}
Consider the estimation problem in (\ref{eqn:EstEqn2}) and take $\gamma_n \geq 4 \sqrt{| g_{\max}|} \BQ_0$. Given the conditions (T2) and (T3) hold, for any solution of (\ref{eqn:EstEqn2}) we shall have
%
\begin{align}
\| \widehat \Theta_j - \Theta_{0,j} \|_F & \leq 12 \sqrt{s_j} \gamma_n / \psi \label{eqn:Thm1ProofProp2Bd1}\\
\sum_{j \neq j', g \in \cG_y^{jj'}} \| \hat \bftheta_{jj'}^{[g]} - \bftheta_{0,jj'}^{[g]} \| & \leq 48 s_j \gamma_n / \psi \label{eqn:Thm1ProofProp2Bd2}
\end{align}
%
Also denote the non-zero support of $\widehat \Theta_j$ by $\widehat \cS_j$, i.e. $\widehat \cS_j = \{ (j',g): \hat \bftheta_{jj'}^{[g]} \neq {\bf 0} \}$. Then
%
\begin{align}
| \widehat \cS_j| \leq 128 s_j / \psi \label{eqn:Thm1ProofProp2Bd3}
\end{align}
\end{Proposition}

{\it Step 2: Edge set selection.} We denote the selected edge set for the $k^\text{th}$ Y-network by $\hat E^k$. Denote its population version by $E_0^k$. Further, let
%
$$
\tilde \Omega_y^k = \diag (\Omega_y^k) + \Omega_{y, E_0^k \cap \hat E^k}^k
$$
%
With similar derivations to the proof of Corollary A.1 in \cite{MaMichailidis15}, The following two upper bounds can be established:
%
\begin{align}
| \hat E^k | \leq \frac{ 128 S }{\psi} \label{eqn:Thm1ProofProp2Bd4}\\
\frac{1}{K} \sum_{k=1}^K \| \tilde \Omega_y^k - \Omega_y^k \|_F \leq
\frac{12 c_0 \sqrt{S} \gamma_n} {\sqrt K \psi} \label{eqn:Thm1ProofProp2Bd5}
\end{align}
%
following which, taking $\gamma_n = 4 \sqrt{| g_{\max}|} \BQ_0$,
%
\begin{align}
\Lambda_{\min} ( \tilde \Omega_y^k) \geq d_0 - \frac{48 c_0 \BQ_0 \sqrt{| g_{\max}| S}}{ \psi} \geq (1 - t_1) d_0 > 0 \label{eqn:Thm1ProofProp2Bd6}\\
\Lambda_{\max} ( \tilde \Omega_y^k) \leq c_0 + \frac{48 c_0 \BQ_0 \sqrt{| g_{\max}| S}}{ \psi} \leq c_0 + t_1 d_0 < \infty \label{eqn:Thm1ProofProp2Bd7}
\end{align}
%
with $0 < t_1 < 1$, and the sample size $n$ satisfying
%
$$
n \geq | g_{\max}| S \left[ \frac{48 c_0 Q_0}{\psi t_1 d_0} \right]^2; \quad Q_0 := \sqrt n \BQ_0
$$

{\it Step 3: Refitting.}
Following the same steps as part A.3 in the proof of Theorem 4.1 in \cite{MaMichailidis15}, it can be proven using \eqref{eqn:Thm1ProofProp2Bd4}--\eqref{eqn:Thm1ProofProp2Bd7} that
%
$$
\sum_{k=1}^K \| \widehat \Omega_y^k - \tilde \Omega_y^k \|^2 \leq O \left( \BQ_0^2 | g_{\max}| S \right)
$$
%
The proof is now complete by combining this with \eqref{eqn:Thm1ProofProp2Bd5} then applying Cauchy-Schwarz inequality and triangle inequality.

%We now prove the norm consistency of $\widehat \Theta_i - \Theta_{0,i}$.
%
%\begin{Proposition}\label{prop:PropA1}
%Copy from Proposition A.1 in 15-656
%\begin{align}
%\sum_{j \neq i, g \in \cG^{ij}} \| \widehat \bftheta_{ij}^{[g]} - \bftheta_{0,ij}^{[g]} \| & \leq \\
%\cM ( \widehat \Theta_i ) & \leq \\
%\| \widehat \Theta_i - \widehat \Theta_{0,i} \|_F & \leq
%\end{align}
%\end{Proposition}
%
%\begin{proof}[Proof of Proposition~\ref{prop:PropA1}]
%%We first proceed in a similar fashion as the proof of Lemma 3.1 in \cite{LouniciEtal11}. For any $\bftheta_i^k \in \BR^q$, we have
%%%
%%$$
%%\sum_{k=1}^K \frac{1}{n} \| \bfY_i^k - \bfY_{-i}^k \widehat \bftheta_i^k - \bfX^k \widehat \bfB_i^k \|^2 + \sum_{j \neq i, g \in \cG^{ij}} \| \widehat \bftheta_{ij}^{[g]} \| \leq 
%%\sum_{k=1}^K \frac{1}{n} \| \bfY_i^k - \bfY_{-i}^k \bftheta_i^k - \bfX^k \widehat \bfB_i^k \|^2 + \sum_{j \neq i, g \in \cG^{ij}} \| \bftheta_{ij}^{[g]} \|
%%$$
%%%
%%subtracting and adding $\bfY_{-i}^k \bftheta_{0,i}^k$ inside the squared norms on both sides, then simolifying and writing $\widehat \bfepsilon_i^k = \bfY_i^k - \bfY_{-i}^k \bftheta_{0,i}^k - \bfX_i^k \widehat \bfB_i^k$ we get
%%%
%%\begin{align*}
%%\frac{1}{n} \| \bfY_{-i}^k ( \widehat \bftheta_i^k - \bftheta_{0,i}^k ) \|^2 & \leq
%%\| \bfY_{-i}^k ( \bftheta_i^k - \bftheta_{0,i}^k ) \|^2 + 
%%\frac{2}{n} ( \widehat \bfepsilon_i^k )^T \bfY_{-i}^k ( \widehat \bftheta_i^k - \bftheta_i^k )\\
%%& + 2 \sum_{j \neq i, g \in \cG^{ij}} \lambda_{ij}^g ( \| \bftheta_{ij}^{[g]} - \widehat \bftheta_{ij}^{[g]} \|)
%%\end{align*}
%The statement of this proposition is same as that of Proposition A.1 in 15-656, and can proved in a similar fashion. The only difference is a modified choice of $\lambda_{ij}^g$, which we obtain from Proposition~\ref{lemma:LemmaE2}.
%
%\end{proof}

%\textit{Part II.} Proof of Thm 2 in 15-656 follows. We only need a new bound for $Var (\bfY_i^k | \bfY_{-i}^k, \bfX^k, \widehat \bfB_i^k)$. For this we have
%%
%$$ Var (\bfY_i^k | \bfY_{-i}^k, \bfX^k, \widehat \bfB_i^k) = \BE (\widehat \bfepsilon_i^k)^2
%= \BE ( \bfepsilon_i^k + \bfdelta_i^k)^2
%\leq \left( \frac{1}{d_0} + \frac{c(v_\beta)}{n} \right)^2
%$$
%%
%applying Cauchy-Schwarz inequality followed by assumption (A2). Now Replace $1/\sqrt{n d_0}$ in choice of $\lambda, \alpha_n$ in Thm 2 statement with $1/\sqrt{n} (\sqrt{1/d_0} + \sqrt{c(v_\beta)/ n})$.

\end{proof}

%\begin{Proposition}
%Given fixed $\widehat \cB$, prediction errors follow bound in T2 with high enough probability.
%\end{Proposition}

\begin{proof}[Proof of Proposition~\ref{prop:ErrorRE}]

%For any sub-gaussian zero-mean design matrix $\bfX \in \BM(n,p)$ with parameters $(\Sigma_x, \sigma_x^2)$, and any $\hat \bfB, \bfB_0 \in \BM(p,q)$ such that $\| \hat \bfB - \bfB_0 \|_F \leq v_\beta$, we follow the proof of Proposition 3 in \cite{LinEtal16} to obtain that for sample size
%%
%\begin{align}\label{eqn:ErrorREeqn1}
%n & \geq 512 ( 1 + 4 \sigma_x^2)^4 \max_j (\Sigma_{x,jj})^4 \log (4p^{\tau_1})
%\end{align}
%the following holds
%%
%\begin{align}\label{eqn:ErrorREeqn2}
%\left\| (\widehat \bfB - \bfB_0)^T \left( \frac{\bfX^T \bfX}{n} \right) (\widehat \bfB - \bfB_0) \right\|_\infty & \leq
%v_\beta^2 \left[ \sqrt{ \frac{ \log 4 + \tau_1 \log p}{c_x n}} + \max_j \Sigma_{x,jj} \right]
%\end{align}
%%
%with probability $\geq 1 - 1/p^{\tau_1-2}$ for some $\tau_1>2$, where
%%
%$$
%c_x = \left[ 128 ( 1 + 4 \sigma_x^2)^2 \max_j (\Sigma_{x,jj})^2 \right]^{-1}
%$$
%%
%Here we substitute $\bfX, \hat \bfB, \bfB_0$ with $\bfX^k, \hat \bfB^k, \hat \bfB_0^k$ respectively. Since rows of $\bfX^k$ come independently from $\cN( {\bf 0}, \Sigma_x^k)$, $\sigma_x^2$ in our case is the spectral norm of $\Sigma_x^k$ \citep{LohWainwright12}, which is $\Lambda_{\max} (\Sigma_x^k)$. Finally
%%
%$$
%\| \bfX^k ( \hat \bfB^k - \bfB_0^k ) \|_\infty \leq 
%\sqrt{\left\| (\widehat \bfB^k - \bfB_0^k)^T (\bfX^k)^T \bfX^k (\widehat \bfB^k - \bfB_0^k) \right\|_\infty}
%$$
%%
%The expression of part 1 is immediate now, and (\ref{eqn:ErrorREeqn1}) ensures that part 1 holds when the leading term of the sample size requirement is $n \succsim \log (pq)$.

We drop the superscript $k$ since there is no scope of ambiguity. For part 1, we start with an auxiliary lemma:
%
\begin{Lemma}\label{lemma:ErrorRElemma1}
For a sub-gaussian design matrix $\bfX \in \BM(n,p)$ with columns having mean ${\bf 0}_p$ and covariance matrix $\Sigma_x$, the sample covariance matrix $\widehat \Sigma_x = \bfX^T \bfX/n$ satisfies the RE condition
%
$$
\widehat \Sigma_x \sim RE \left( \frac{\Lambda_{\min} ( \Sigma_x) }{2}, \frac{\Lambda_{\min} ( \Sigma_x) \log p }{2 n} \right)
$$
%
with probability $\geq 1 - 2 \exp(-c_3 n)$ for some $c_3 > 0$.
\end{Lemma}
%
Now denote $\widehat \bfE = \bfY - \bfX \widehat \bfB$. For $\bfv \in \BR^q$, we have
%
\begin{align}\label{eqn:ErrorREeqn3}
\bfv^T \widehat \bfS \bfv &= \frac{1}{n} \| \widehat \bfE \bfv \|^2 \notag\\
&= \frac{1}{n} \| (\bfE + \bfX ( \bfB_0 - \widehat \bfB ))\bfv \|^2 \notag\\
&= \bfv^T \bfS \bfv + \frac{1}{n} \| \bfX (\bfB_0 - \widehat \bfB) \bfv \|^2 + 2 \bfv^T (\bfB_0 - \widehat \bfB)^T \left( \frac {(\bfX)^T \bfE}{n} \right) \bfv
\end{align}
%
For the first summand, $ \bfv^T \bfS^k \bfv \geq \psi_y \| \bfv \|^2 - \phi_y \| \bfv \|_1^2$ with $\psi_y = \Lambda_{\min} (\Sigma_y)/2, \phi_y = \psi_y \log p/n$ by applying Lemma \ref{lemma:ErrorRElemma1} on $\bfS$. The second summand is greater than or equal to 0. For the third summand,
%
$$
2 \bfv^T (\bfB_0 - \widehat \bfB)^T \left( \frac {(\bfX)^T \bfE}{n} \right) \bfv \geq
-2 v_\beta \left\| \frac {(\bfX)^T \bfE}{n} \right\|_\infty \| \bfv \|_1^2
$$
%
by assumption (T1). Now we use another lemma:
%
\begin{Lemma}\label{lemma:ErrorRElemma2}
For zero-mean independent sub-gaussian matrices $\bfX \in \BM(n,p), \bfE \in \BM(n,q)$ with parameters $(\Sigma_x, \sigma_x^2)$ and $(\Sigma_e, \sigma_e^2)$ respectively, given that $n \succsim \log(pq)$ the following holds with probability $\geq 1 - 6c_1 \exp [-(c_2^2-1) \log(pq)]$ for some $c_1 >0, c_2 > 1$:
%
$$
\frac{1}{n} \| \bfX^T \bfE \|_\infty \leq c_2 [ \Lambda_{\max} (\Sigma_x) \Lambda_{\max} (\Sigma_e) ]^{1/2} \sqrt{\frac{ \log(pq)}{n}}
$$
%
\end{Lemma}
%
Subsequently we collect all summands in (\ref{eqn:ErrorREeqn3}) and get
%
$$
\bfv^T \widehat{ \bfS} \bfv \geq \psi_y \| \bfv \|^2 - \left( \phi_y + 2 v_\beta c_2 [ \Lambda_{\max} (\Sigma_x) \Lambda_{\max} (\Sigma_y) ]^{1/2} \sqrt{\frac{ \log(pq)}{n}} \right) \| \bfv \|_1^2
$$
with probability $\geq 1 - 2\exp(- c_3 n) - 6c_1 \exp [-(c_2^2-1) \log(pq)]$. This concludes the proof of part 1.

To prove part 2, we decompose the quantity in question:
%
\begin{align}\label{eqn:ErrorRElemma2maineqn}
\left\| \frac{1}{n} \widehat \bfE_{-j}^T \widehat \bfE \bfT_{0,j} \right\|_\infty &=
\left\| \frac{1}{n} \left[ \bfE_{-j} + \bfX (\bfB_{0,j} - \widehat \bfB_j) \right]^T \left[ \bfE + \bfX (\bfB_0 - \widehat \bfB) \right] \bfT_{0,j} \right\|_\infty \notag\\
& \leq \left\| \frac{1}{n} \bfE_{-j}^T \bfE \bfT_{0,j} \right\|_\infty +
\left\| \frac{1}{n} \bfE_{-j}^T \bfX (\bfB_0 - \widehat \bfB) \bfT_{0,j} \right\|_\infty \notag\\
& + \left\| \frac{1}{n} (\bfB_{0,j} - \widehat \bfB_j)^T \bfX^T \bfX (\bfB_0 - \widehat \bfB) \bfT_{0,j} \right\|_\infty +
\left\| \frac{1}{n} (\bfB_{0,j} - \widehat \bfB_j)^T \bfX^T \bfE \bfT_{0,j} \right\|_\infty \notag\\
&= \| \bfW_1 \|_\infty + \| \bfW_2 \|_\infty + \| \bfW_3 \|_\infty + \| \bfW_4 \|_\infty
\end{align}
%
Now
%
$$
\bfW_1 = \frac{1}{n} \bfE_{-j}^T ( \bfE_j - \bfE_{-j} \bftheta_{0,j})
$$
%
For node $j$ in the $y$-network, $\BE_{-j}$ and $E_j - \BE_{-j} \bftheta_{0,j}$ are the neighborhood regression coefficients and residuals, respectively. Thus they are orthogonal, so we can apply Lemma \ref{lemma:ErrorRElemma2} on $\bfE_{-j}$ and $\bfE_j - \bfE_{-j} \bftheta_{0,j}$ to obtain that for $n \succsim \log (q-1)$,
%
\begin{align}\label{eqn:ErrorRElemma2eqn5}
\| \bfW_1 \|_\infty & \leq c_5 \left[ \Lambda_{\max} ( \Sigma_{y,-j}) \sigma_{y,j,-j} \right]^{1/2} \sqrt{\frac{\log(q-1)}{n}}
\end{align}
%
holds with probability $\geq 1 - 6c_4 \exp [-(c_5^2-1) \log(pq)]$ for some $c_4 > 0, c_5 > 1$.
%

The same bounds hold for $\bfW_2$ and $\bfW_4$:
%
\begin{align*}
\| \bfW_2 \|_\infty & \leq \left\| \frac{1}{n} \bfE_{-j}^T \bfX (\bfB_0 - \widehat \bfB) \right\|_\infty \| \bfT_{0,j} \|_1 \leq
\left\| \frac{1}{n} \bfE^T \bfX \right\|_\infty \| \bfB_0 - \widehat \bfB \|_1 \| \bfT_{0,j} \|_1\\
\| \bfW_4 \|_\infty & \leq \left\| \frac{1}{n} (\bfB_{0,j} - \widehat \bfB_j)^T \bfX^T \bfE \right\|_\infty \| \bfT_{0,j} \|_1 \leq
\left\| \frac{1}{n} \bfE^T \bfX \right\|_\infty \| \bfB_0 - \widehat \bfB \|_1 \| \bfT_{0,j} \|_1\\
\end{align*}
%
Now since $\Omega_y$ is diagonally dominant, $|\omega_{y,jj}| \geq \sum_{j \neq j'} |\omega_{y,jj'}|$ for any $j \in \cI_q$. Hence
%
$$
\| \bfT_{0,j} \|_1 = \sum_{j'=1}^q | T_{jj'} | = 1 + \sum_{j \neq j'} | \theta_{jj'} | = 1 + \frac{1}{\omega_{y,jj}} \sum_{j \neq j'} | \omega_{y,jj'} | \leq 2
$$
%
so that for $n \succsim \log (pq)$,
%
\begin{align}\label{eqn:ErrorRElemma2eqn6}
\| \bfW_2 \|_\infty + \| \bfW_4 \|_\infty  & \leq
4 v_\beta c_2 [ \Lambda_{\max} (\Sigma_x) \Lambda_{\max} (\Sigma_y) ]^{1/2} \sqrt{\frac{ \log(pq)}{n}}
\end{align}
%
with probability $\geq 1 - 6c_1 \exp [-(c_2^2-1) \log(pq)]$ by applying Lemma~\ref{lemma:ErrorRElemma2} and assumption (T1).

Finally for $\bfW_3$, we apply Lemma 8 of \cite{RavikumarEtal11} on the (sub-gaussian) design matrix $\bfX$ to obtain that for sample size
%
\begin{align}\label{eqn:ErrorRElemma2eqn7}
n \geq 512 ( 1 + 4 \Lambda_{\max} (\Sigma_x^k))^4 \max_i (\sigma_{x,ii}^k )^4 \log (4p^{\tau_1})
\end{align}
%
we get that with probability $ \geq 1 - 1/p^{\tau_1-2}, \tau_1 > 2$,
%
$$
\left\| \frac{\bfX^T \bfX}{n} \right\|_\infty \leq \sqrt{ \frac{ \log 4 + \tau_1 \log p}{c_x n}} + \max_i \sigma_{x,ii} = V_x; \quad
c_x = \left[ 128 ( 1 + 4 \Lambda_{\max} (\Sigma_x)  )^2 \max_i (\sigma_{x,ii})^2 \right]^{-1}
$$
%
Thus with the same probability,
%
\begin{align}\label{eqn:ErrorRElemma2eqn4}
\| \bfW_4 \|_\infty \leq \left\| \frac{\bfX^T \bfX}{n} \right\|_\infty \| \widehat \bfB - \bfB_0 \|_1^2 \| \bfT_{0,j} \|_1 
\leq 2 v_\beta^2 V_x
\end{align}
%
We now bound the right hand side of (\ref{eqn:ErrorRElemma2maineqn}) using (\ref{eqn:ErrorRElemma2eqn5}), (\ref{eqn:ErrorRElemma2eqn6}) and (\ref{eqn:ErrorRElemma2eqn4}) to complete the proof, with the leading term of the sample size requirement being $n \succsim \log(pq)$.

%%
%\begin{align}
%\| \widehat \bfS^k \bfT_{0,j}^k \|_\infty & \leq
%\| \widehat \bfS^k \|_\infty \| \bfT_{0,j}^k  \|_1 \notag\\
%& \leq \left[ \| \widehat \bfS^k - \bfS^k \|_\infty + \| \bfS^k \|_\infty \right] \| \bfT_{0,j}^k  \|_1 \label{eqn:ErrorRElemma2eqn1}
%\end{align}
%%
%Since $\Sigma_y^k$ is diagonally dominant, (see proof of Proposition~\ref{lemma:LemmaE2}) we have
%%
%\begin{align}\label{eqn:ErrorRElemma2eqn2}
% \| \bfT_{0,j}^k \|_1 \leq 2
%\end{align}
%%
%Moving on to $\| \widehat \bfS^k - \bfS^k \|_\infty$:
%%
%\begin{align}
%\| \widehat \bfS^k - \bfS^k \|_\infty & \leq \left\| \frac{2}{n} (\bfE^k)^T \bfX^k (\bfB^k_0 - \widehat \bfB^k) \right\|_\infty + 
%\left\| (\widehat \bfB^k - \bfB_0^k)^T \left( \frac{(\bfX^k)^T (\bfX^k)}{n} \right) (\widehat \bfB^k) - \bfB_0^k)) \right\|_\infty \notag\\
%& \leq 2 \left\| \frac{(\bfX^k)^T \bfE^k}{n} \right\|_\infty \left\| \bfB^k_0 - \widehat \bfB^k) \right\|_1 + 
%\left\| (\widehat \bfB^k - \bfB_0^k)^T \left( \frac{(\bfX^k)^T (\bfX^k)}{n} \right) (\widehat \bfB^k) - \bfB_0^k)) \right\|_\infty \label{eqn:ErrorRElemma2eqn3}
%\end{align}
%%
%By applying Lemma~\ref{lemma:ErrorRElemma2} on $\bfX^k, \bfE^k$, and assumption (T1) we have for $n \succsim \log (pq)$ and with probability $\geq 1 - 6c_1 \exp [-(c_2^2-1) \log(pq)]$,
%%
%$$
%2 \left\| \frac{(\bfX^k)^T \bfE^k}{n} \right\|_\infty \left\| \bfB^k_0 - \widehat \bfB^k) \right\|_1 \leq
%2 v_\beta c_2 [ \Lambda_{\max} (\Sigma_x^k) \Lambda_{\max} (\Sigma_y^k) ]^{1/2} \sqrt{\frac{ \log(pq)}{n}}
%$$
%%
%The second term is bounded by substituting $\bfX, \hat \bfB, \bfB_0$ with $\bfX^k, \hat \bfB^k, \hat \bfB_0^k$ respectively in (\ref{eqn:ErrorREeqn2}).
%
%For a bound on the $\ell_\infty$-norm of $\bfS^k = (\bfX^k)^T \bfX^k/n$, we apply Lemma 8 of \cite{RavikumarEtal11} on the (sub-gaussian) design matrix $\bfX^k$ to obtain that for sample size
%$$
%n \geq 512 ( 1 + 4 \Lambda_{\max} (\Sigma_x^k))^4 \max_j (\sigma_{x,jj}^k )^4 \log (4p^{\tau_1})
%$$
%%
%we get that with probability $ \geq 1 - 1/p^{\tau_1-2}$,
%%
%\begin{align}\label{eqn:ErrorRElemma2eqn4}
%\| \bfS^k \|_\infty & \leq \sqrt{ \frac{ \log 4 + \tau_1 \log p}{c_x^k n}} + \max_j \sigma_{x,jj}^k; \quad
%c_x^k = \left[ 128 ( 1 + 4 \Lambda_{\max} (\Sigma_x^k)  )^2 \max_j (\sigma_{x,jj}^k)^2 \right]^{-1}
%\end{align}
%%
%We now bound the right hand side of (\ref{eqn:ErrorRElemma2eqn1}) using (\ref{eqn:ErrorRElemma2eqn2}), (\ref{eqn:ErrorRElemma2eqn3}) and (\ref{eqn:ErrorRElemma2eqn4}), which proves part 3.
\end{proof}

\begin{proof}[Proof of Theorem~\ref{thm:BetaThm}]
The proof follows that of Proposition~\ref{prop:Thm1ProofProp2}, with a different group norm structure. We only point out the differences.

Putting $\bfbeta = \bfbeta_0$ in (\ref{eqn:EstEqn1}) we get
%
$$
-2 \widehat \bfbeta^T \widehat \bfgamma + \bfbeta^T \widehat \bfGamma \widehat \bfbeta + \lambda_n \sum_{h \in \cH} \| \widehat \bfbeta^{[h]}  \| \leq
-2 \bfbeta_0^T \widehat \bfgamma + \bfbeta_0^T \widehat \bfGamma \bfbeta_0 + \lambda_n \sum_{h \in \cH} \| \bfbeta_0^{[h]}  \|
$$
%
Denote $\bfb = \widehat \bfbeta - \bfbeta_0$. Then we have
%
$$
\bfb^T \widehat \bfGamma \bfb \leq 2 \bfb^T ( \widehat \bfgamma - \widehat \bfGamma \bfbeta_0 ) + \lambda_n
\sum_{h \in \cH} ( \| \bfbeta_0^{[h]} \| - \| \bfbeta_0^{[h]} + \bfb^{[h]} \|)
$$
%
Proceeding similarly as the proof of Proposition~\ref{prop:Thm1ProofProp2}, with a different deviation bound and choice of $\lambda_n$, we get expressions equivalent to (\ref{eqn:Thm1ProofProp2Eqn3}) and (\ref{eqn:Thm1ProofProp2Eqn2}) respectively:
%
\begin{align}
\bfb^T \widehat \bfGamma \bfb & \leq \frac{3}{2} \sum_{h \in \cH} \| \bfb^{[h]} \| \\
\frac{\psi^*}{3} \| \bfb \|^2 & \leq \lambda_n \sum_{h \in \cH} \| \bfb^{[h]} \| \leq 4 \lambda_n \sqrt{s_\beta} \| \bfb \|
\end{align}
%
Furthermore, $\| \bfb \|_1 \leq \sqrt{ | h_{\max} |} \sum_{h \in \cH} \| \bfb^{[h]} \| $. The bounds in (\ref{eqn:BetaThmEqn1}), (\ref{eqn:BetaThmEqn2}), (\ref{eqn:BetaThmEqn3}) and (\ref{eqn:BetaThmEqn4}) now
follow.

\end{proof}

\begin{proof}[Proof of Lemma~\ref{prop:ThmBetaRE}]
For part 1 it is enough to prove that with $ \widehat \Sigma_x^k := (\bfX^k)^T \bfX^k/n$,
%
\begin{align}\label{eqn:ThmBetaREProofEqn1}
\widehat \bfT_k^2 \otimes \widehat \Sigma_x^k & \sim RE (\psi_*^k, \phi_*^k)
\end{align}
%
with high enough probability. because then we can take $\psi_* = \min_k \psi_*^k, \phi_* = \max_k \phi_*^k$. The proof of (\ref{eqn:ThmBetaREProofEqn1}) follows similar lines of the proof of Proposition 1 in \cite{LinEtal16}, only replacing $\Theta_\epsilon, \widehat \Theta_\epsilon, \bfX$ therein with $(\bfT^k)^2, (\widehat \bfT^k)^2, \bfX^k$, respectively. We omit the details.

Part 2 follows the proof of Proposition 2 in \cite{LinEtal16}.
%, with an extra step to prove that $(\bfT^k)^2$ is diagonally dominant. For this, denote the elements of $(\bfT^k)^2$ by $\tau_{jj'}, j; j' \in \cI_q$. Then
%%
%\begin{align*}
%\tau_{jj} - \sum_{j \neq j'} | \tau_{jj'}| & = \sum_{l=1}^q \left[
%\frac{\omega_{jl}^2}{\omega_{jj}^2 } - \sum_{j \neq j'} \frac{| \omega_{jl} \omega_{j'l} |}{\omega_{jj} \omega_{j'j'} } \right]
%\end{align*}
%%
%For each summand,
%%
%\begin{align*}
%\frac{\omega_{jl}^2}{\omega_{jj}^2 } - \sum_{j \neq j'} \frac{| \omega_{jl} \omega_{j'l} |}{\omega_{jj} \omega_{j'j'}} & = \frac{| \omega_{jl} |}{\omega_{jj} } \left[ \frac{| \omega_{jl}| }{\omega_{jj} } - \sum_{j' \neq j} \frac{| \omega_{j'l} |}{\omega_{j'j'}} \right] \\
%\end{align*}
%

\end{proof}

We also have the following auxiliary result that helps establish results in the testing section.

\begin{Lemma}\label{lemma:Omega-lemma}
{\colrbf lemma and proof tbd}
\end{Lemma}

\begin{proof}[Proof of Theorem~\ref{Thm:ThmTesting}]
Let us define the following:
%
\begin{align*}
\widehat \Omega_y &= \diag(\widehat \Omega_y^1, \widehat \Omega_y^2)\\
\bfM_i &= \diag(m^1_i, m^2_i)\\
\widehat \bfC_i &= \diag(\widehat \bfc^1_i, \widehat \bfc^2_i)\\
\widehat \bfD_i &= \diag(\widehat \bfb^1_i, \widehat \bfb^2_i)\\
\bfD_i &= \diag(\bfb^1_{0,i}, \bfb^2_{0,i})\\
\bfR_i^k &= \bfX_i^k - \bfX_{-i}^k \widehat \bfzeta_i^k; k = 1,2
\end{align*}
%
Then from \eqref{eqn:DebiasedBeta} we have
%
\begin{align}\label{eqn:ThmTestingProofeq1}
\bfM_i ( \widehat \bfC_i - \widehat \bfD_i )^T &= \frac{1}{\sqrt n}
\begin{bmatrix}
\frac{1}{\widehat s^1_i} (\bfR^1_i)^T \widehat \bfE^1 \\
\frac{1}{\widehat s^2_i} (\bfR^2_i)^T \widehat \bfE^2
\end{bmatrix}
\end{align}
%
We now decompose $\widehat \bfE^k:$
%
\begin{align*}
\widehat \bfE^k &= \bfY^k - \bfX^k \widehat \bfB^k \\
&= \bfE^k + \bfX^k (\bfB_0^k - \widehat \bfB^k)\\
&= \bfE^k + \bfX_i^k (\bfb_{0,i}^k - \widehat \bfb_i^k) + \bfX_{-i}^k (\bfB_{0,-i}^k - \widehat \bfB_{-i}^k)
\end{align*}
%
Putting them back in \eqref{eqn:ThmTestingProofeq1} and using $t^k = (\bfR^k)^T \bfX^k/n$,
%
\begin{align}
\bfM_i ( \widehat \bfC_i - \widehat \bfD_i)^T &= \frac{1}{\sqrt n}
\begin{bmatrix}
\frac{1}{\widehat s^1_i} (\bfR^1_i)^T \bfE^1 \\
\frac{1}{\widehat s^2_i} (\bfR^2_i)^T \bfE^2
\end{bmatrix} +
\bfM_i (\bfD_i - \widehat \bfD_i )^T +
\frac{1}{\sqrt n}
\begin{bmatrix}
\frac{1}{\widehat s^1_i} (\bfR^1_i)^T \bfX_{-i}^1 (\bfB_{0,-i}^1 - \widehat \bfB_{-i}^1) \\
\frac{1}{\widehat s^2_i} (\bfR^2_i)^T \bfX_{-i}^2 (\bfB_{0,-i}^2 - \widehat \bfB_{-i}^2)
\end{bmatrix} \notag\\
\Rightarrow
\widehat \Omega_y^{1/2} \bfM_i ( \widehat \bfC_i - \bfD_i)^T &=
\frac{\widehat \Omega_y^{1/2}}{\sqrt n}
\begin{bmatrix}
\frac{1}{\widehat s^1} (\bfR^1_i)^T \bfE^1 \\
\frac{1}{\widehat s^2} (\bfR^2_i)^T \bfE^2
\end{bmatrix} +
\frac{\widehat \Omega_y^{1/2}}{\sqrt n}
\begin{bmatrix}
\frac{1}{\widehat s^1_i} (\bfR^1_i)^T \bfX_{-i}^1 (\bfB_{0,-i}^1 - \widehat \bfB_{-i}^1) \\
\frac{1}{\widehat s^2_i} (\bfR^2_i)^T \bfX_{-i}^2 (\bfB_{0,-i}^2 - \widehat \bfB_{-i}^2)
\end{bmatrix}\label{eqn:ThmTestingProofeq2}
\end{align}

At this point, we drop $k$ in the subscripts, and prove the following:

\begin{Lemma}\label{Lemma:ThmTestingLemma}
Given conditions (C1) and (C2), the following holds for sample size $n$ such that $n \succsim \log (pq)$ and $\sigma_{x,i,-i} - n^{-1/4} - v_\zeta \sqrt{V_x} > 0$:
%
$$
\frac{1}{\sqrt n \widehat s_i}  \widehat \Omega_y^{1/2} \bfE^T \bfR_i \sim
\cN_q ({\bf 0}, \bfI) + \bfS_{1n};
$$
%
\begin{align}\label{eqn:ThmTestingProofeq3}
\| \bfS_{1n} \|_\infty & \leq 
\frac{v_\Omega (2 + v_\zeta) c_2 [ \Lambda_{\max} (\Sigma_x) \Lambda_{\max} (\Sigma_e) ]^{1/2} \sqrt{ \log(pq)}}{\sigma_{x,i,-i} - n^{-1/4} - v_\zeta \sqrt{V_x}} =
O \left( \frac{ \log(pq)}{\sqrt n} \right)
\end{align}
%
with probability larger than or equal to
%
\begin{align}\label{eqn:ThmTestingProofeq30}
1 - 6c_1 e^{-(c_2^2-1) \log p q} - \frac{1}{p^{\tau_1-2}} - \frac{\kappa_i}{\sqrt n}
\end{align}
%
for some $c_1, c_4 >0, c_2, c_5 > 1$, and $\kappa_i := \BV [(X_i - \BX_{-i} \bfzeta_{0,-i})^2]$.
%
Additionally, given condition (C3)
%
\begin{align}\label{eqn:ThmTestingProofeq4}
& \left\| \frac{1}{\sqrt n \widehat s_i} \bfR_i^T \bfX_{-i} (\bfB_{0,-i} - \widehat \bfB_{-i})
\widehat \Omega_y^{1/2} \right\|_\infty \notag\\
& \leq
\frac{v_\beta ( \Lambda_{\min} (\Sigma_y)^{1/2} + v_\Omega)}{\sigma_{x,i,-i} - n^{-1/2} - v_\zeta \sqrt{V_x}} 
\left[ c_{7} \sqrt{ (\sigma_{x,i,-i} \Lambda_{\max} (\Sigma_{x, -i}) ) \log p} + \sqrt n v_\zeta V_x \right] =
O \left( \frac{ \log(pq)}{\sqrt n} \right)
\end{align}
%
with probability condition \eqref{eqn:ThmTestingProofeq30}.
\end{Lemma}

Given Lemma~\ref{Lemma:ThmTestingLemma}, the first and second summands on the right hand side of \eqref{eqn:ThmTestingProofeq2} are bounded above by applying each of \eqref{eqn:ThmTestingProofeq3} and \eqref{eqn:ThmTestingProofeq4} twice, respectively. This completes our proof.
\end{proof}

\begin{proof}[Proof of Theorem~\ref{thm:FDRthm}]

\end{proof}


