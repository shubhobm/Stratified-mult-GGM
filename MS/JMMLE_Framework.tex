\section{The Joint Multiple Multilevel Estimation Framework}
\subsection{Formulation}
Suppose there are $K$ independent datasets, each pertaining to an $M$-layered Gaussian Graphical Model (GGM). The $k^{\Th}$ model has the following structure:

\vspace{1em}
\begin{tabular}{ll}
{\it Layer 1}- &
%
$\BD_1^k = (D_{1 1}^k, \ldots, D^k_{1 p_1}) \sim
\cN (0, \Sigma_1^k); \quad k \in \cI_K,$\\
{\it Layer $m$} $(1< m \leq M)$-  &
%
$ \BD_m^k = \BD_{m-1}^k \bfB_m^k + \BE_m^k$, with $\bfB_m^k \in \BM(p_{m-1}, p_m) $\\
& and $\BE_m^k = (E_{m 1}^k, \ldots, E^k_{m p_m}) \sim
\cN (0, \Sigma_m^k); \quad k \in \cI_K $.\\
\end{tabular}
\vspace{1em}

We assume known structured sparsity patterns, denoted by $\cG_m$ and $\cH_m$, for the parameters of interest in the above model, i.e. the precision matrices $\Omega_m^k := (\Sigma_m^k)^{-1}$ and the regression coefficient matrices $\bfB_m^k$, respectively. These patterns provide information on horizontal dependencies across $k$ for the corresponding parameters, and our goal here is to leverage them to estimate the full hierarchical structure of the network- specifically to obtain the undirected edges inside nodes of a single layer, and the directed edges between two successive layers through jointly estimating $\{ \Omega_m^k \}$ and $\{ \bfB_m^k \}$.

Consider a two-layer model, which is a special case of the above model with $M=2$:
%
\begin{eqnarray}
\BX^k = (X^k_1, \ldots, X^k_p)^T \sim \cN (0, \Sigma^k_x);\\
\BY^k = \BX^k \bfB^k + \BE^k; \quad \BE^k = (E^k_1, \ldots, E^k_p)^T \sim \cN (0, \Sigma^k_y);\\
\bfB^k \in \BM(p,q), \quad \Omega^k_x = (\Sigma^k_x)^{-1}; \quad \Omega^k_y = (\Sigma^k_y)^{-1};
\end{eqnarray}
%
where we want to estimate $\{ (\Omega^k_x, \Omega^k_y, \bfB^k); k \in \cI_K$ from data $\cZ^k = \{ (\bfY^k, \bfX^k); \bfY^k \in \BM(n,q), \bfX^k \in \BM(n,p), k \in \cI_K\}$ in presence of known grouping structures $\cG_x, \cG_y, \cH$ respectively and assuming $n_k = n$ for all $k \in \cI_K$ for simplicity. We focus most of the theoretical discussion in the rest of the paper on jointly estimating $\Omega_y:= \{ \Omega_y^k \}$ and $\cB := \{ \bfB^k \}$. This is because for $M>2$, within-layer undirected edges of any $m{\Th}$ layer $(m>1)$ and between-layer directed edges from the $(m-1)^{\Th}$ layer to the $m^{\Th}$ layer can be estimated from the corresponding data matrices in a similar fashion. On the other hand, parameters in the first layer are analogous to $\Omega_x := \{ \Omega_x^k \}$ that are dependent only on $\{ \bfX^k\}$, so any method for joint estimation of multiple graphical models can be used to estimate them (e.g. \cite{GuoEtal11, MaMichailidis15}). This provides all building blocks for estimating the full hierarchical structure of our $M$-layered multiple GGMs.

%Setting $M=1$ reduces the above model to joint estimation of GGMs with structured sparsity \citep{MaMichailidis15}, while setting $K=1$ reduces the model to a multi-layer GGM, which can be estimated by breaking it down to successive two-layer models and then minimizing a penalized conditional log-likelihood function \citep{LinEtal16}.

\subsection{Algorithm}
We assume an element-wise group sparsity pattern over $k$ for the precision matrices $\Omega_x^k$:
%
\[
\cG_x = \{ \cG_x^{ii'}: i \neq i'; i, i' \in \cI_p \},
\]
%
where each $\cG_x^{ii'}$ is a partition of $\cI_K$. Subsequently we use the Joint Structural Estimation Method (JSEM) \citep{MaMichailidis15} to estimate $\Omega_x$, which first uses the group structure given by $\cG_x$ in penalized nodewise regressions \citep{MeisenBuhlmann06} to obtain neighborhood coefficients of each variable $X_i, i \in \cI_p$, then fits a graphical lasso model over the combined support sets to obtain sparse estimates of the precision matrices:
%
\begin{align}\label{eqn:jsem-model}
\widehat \zeta_i &= \argmin_{\zeta_i} \left\{
\frac{1}{n} \sum_{k=1}^K \| \bfX_i^k - \bfX_{-i}^k \bfzeta_i^k \|^2 +
\sum_{i' \leq i} \sum_{g \in \cG_x^{ii'}} \eta_n \| \bfzeta_{ii'}^{[g]} \| \right\} \notag\\
\widehat E_x^k &= \{(i,i'): 1 \leq i < i' \leq p, \hat \zeta_{ii'}^k \neq 0 \text{ OR } \hat \zeta_{i'i}^k \neq 0 \}\notag\\
\widehat \Omega_x^k &= \argmin_{\Omega_x^k \in \BS_+ (\hat E_x^k)}
\left\{ \Tr (\widehat \bfS_x^k \Omega_x^k ) - \log \det (\Omega_x^k) \right\}
\end{align}
%
where $\widehat \bfS_x^k := (\bfX^k)^T \bfX^k/n_k$.

For the precision matrices $\Omega_y^k$ we assume an element-wise sparsity pattern $\cG_y$ defined in a similar manner as $\cG_x$, while the sparsity pattern $\cH$ for $\cB$ is more general, each group $h \in \cH$ being defined as:
%
$$
h = \{ (\cS_p, \cS_q, \cS_K): \cS_p \subseteq \cI_p, \cS_q \subseteq \cI_q, \cS_K \subseteq \cI_K \}
; \quad \bigcup_{h \in \cH} h = \cI_p \times \cI_q \times \cI_K
$$
%
We obtain sparse estimates of $\Omega_y$ and $\cB$ by solving the following group-penalized least square minimization problem:
%
\begin{align}
\{ \widehat \cB, \widehat \Theta \} &= 
\argmin_{\cB, \Theta} \left\{ \frac{1}{n} \sum_{j=1}^q \sum_{k=1}^K \| \bfY^k_j - (\bfY_{-j}^k - \bfX^k \bfB_{-j}^k) \bftheta_j^k - \bfX^k \bfB_j^k \|^2 \right. \notag\\
& \left. + \sum_{j \neq j'} \sum_{g \in \cG_y^{jj'}} \gamma_n \| \bftheta_{jj'}^{[g]} \| + \sum_{h \in \cH} \lambda_n \| \bfB^{[h]} \| \right\} \label{eqn:jmmle-objfun}\\
\widehat E_y^k &= \{(j,j'): 1 \leq j < j' \leq q, \hat \theta_{jj'}^k \neq 0 \text{ OR } \hat \theta_{j'j}^k \neq 0 \}\notag\\
\widehat \Omega_y^k &= \argmin_{\Omega_y^k \in \BS_+ (\hat E_y^k)}
\left\{ \Tr (\widehat \bfS_y^k \Omega_y^k ) - \log \det (\Omega_y^k) \right\} \label{eqn:omega-y-calc}
\end{align}
%&= \min \left\{ f ( \cY, \cX, \cB, \Theta) + P (\Theta) + Q (\cB) \right\} 
%
The outcome of a node in the lower layer is thus modeled using all other nodes in that layer {\it and} nodes in the immediate upper layer, with their effects quantified using $\widehat \bftheta_j^k$ and $\widehat \bfB_j^k$, respectively.

\subsubsection{Alternating algorithm}
The objective function in \eqref{eqn:jmmle-objfun} is bi-convex, i.e. convex in $\cB$ for fixed $\Theta$, and vice-versa, but not jointly convex in $\{ \cB, \Theta \}$. Consequently, we use an alternating iterative algorithm to solve for $\{ \cB, \Theta \}$ that minimizes \eqref{eqn:jmmle-objfun} by iteratively cycling between $\cB$ and $\Theta$, i.e. holding one set of parameters fixed and solving for the other, then alternating until convergence.

Choice of initial values plays a crucial role in the performance of this alternating algorithm.% Following the analysis of \cite{LinEtal16}, who proposed an estimation framework for the special case of our multi-level structure for $K=1$ based on a similar algorithm, we choose the initial values $\{ \bfB^{k (0)} \}$ by first selecting a support set for the $j^{\Th}$ column, say $\tilde \cS_j^k$, as the support of the debiased lasso estimate of \cite{JavanmardMontanari14}, then fitting a Lasso regression model with small penalty value $\lambda_n^0$:
%
%\begin{align}\label{eqn:init-B}
%\widehat \bfB_j^{k (0)} = \argmin_{\supp(\bfB_j^k) \subseteq \tilde \cS_j^k} \|\bfY_j^k - \bfX^k \bfB_j^k \|^2 + \lambda_n^0 \| \bfB_j^k \|_1
%\end{align}
%
We choose the initial values $\{ \bfB^{k (0)} \}$ by fitting separate lasso regression models for each column of the coefficient matrices:
%
\begin{align}\label{eqn:init-B}
\widehat \bfB_j^{k (0)} = \argmin_{\bfB_j^k \in \BR^p} \|\bfY_j^k - \bfX^k \bfB_j^k \|^2 + \lambda_n \| \bfB_j^k \|_1; \quad
j \in \cI_q, k \in \cI_K.
\end{align}
%

We obtain initial estimates of $ \Theta_j, j \in \cI_q$ by performing group-penalized nodewise regression on the residuals $\widehat \bfE^{k (0)} := \bfY^k - \bfX^k \bfB_j^{k (0)}$:
%
\begin{align}\label{eqn:init-Theta}
\widehat \Theta_j^{(0)} = \argmin_{\Theta_j} \frac{1}{n} \sum_{k=1}^K \|
\widehat \bfE_j^{k (0)} - \widehat \bfE_{-j}^{k (0)} \bftheta_j^k \|^2
+ \gamma_n \sum_{j \neq j'} \sum_{g \in \cG_y^{jj'}} \| \bftheta_{jj'}^{[g]} \|.
\end{align}

The steps of our full estimation procedure, which we call the Joint Multiple Multilayer Estimation (JMMLE) method, can thus be summarized in Algorithm \ref{algo:jmmle-algo}.

\begin{Algorithm}
(The JMMLE Algorithm)
\label{algo:jmmle-algo}

\noindent 1. Initialize $\widehat \cB$ using \eqref{eqn:init-B}.

\noindent 2. Initialize $\widehat \Theta$ using \eqref{eqn:init-Theta}.

\noindent 3. Update $\widehat \cB$ as:
%
\begin{align}\label{eqn:update-B}
\widehat \cB^{(t+1)} &= \argmin_{\substack{\bfB^k \in \BM(p,q)\\k \in \cI_K}} \left\{ \frac{1}{n} \sum_{j=1}^q \sum_{k=1}^K \| \bfY^k_j - (\bfY_{-j}^k - \bfX^k \bfB_{-j}^k) \widehat \bftheta_j^{k (t)} - \bfX^k \bfB_j^{k } \|^2
+ \lambda_n \sum_{h \in \cH} \| \bfB^{[h]} \| \right\}
\end{align}

\noindent 4. Obtain $\widehat \bfE^{k (t+1)} := \bfY^k - \bfX^k \bfB_j^{k (t)}, k \in \cI_K$. Update $\widehat \Theta$ as:
%
\begin{align}\label{eqn:update-Theta}
\widehat \Theta_j^{(t+1)} = \argmin_{\Theta_j \in \BM(q-1, K)}
\left\{ \frac{1}{n} \sum_{k=1}^K
\| \widehat \bfE_j^{k (t+1)} - \widehat \bfE_{-j}^{k (t+1)} \bftheta_j^k \|^2
+ \gamma_n \sum_{j \neq j'} \sum_{g \in \cG_y^{jj'}} \| \bftheta_{jj'}^{[g]} \| \right\}
\end{align}

\noindent 5. Continue till convergence.

\noindent 6. Calculate $\widehat \Omega_y^k, k \in \cI_K$ using \eqref{eqn:omega-y-calc}.
\end{Algorithm}

\subsubsection{Tuning parameter selection}
The nodewise regression step in the JSEM model \eqref{eqn:jsem-model} uses Bayesian Information Criterion (BIC) for tuning parameter selection. The step for updating $\{ \Theta \}$, i.e. \eqref{eqn:update-Theta}, in our JMMLE algorithm is analogous to this procedure, hence we use BIC to select the penalty parameter $\gamma_n$. In our setting the BIC for a given $\gamma$ and fixed $\cB$ is given by:
%
\begin{align*}
\text{BIC} (\gamma; \cB) &=
\Tr \left( \bfS_y^k \widehat \Omega_{y,\gamma}^k \right) - \log \det \left( \widehat \Omega_{y,\gamma}^k \right) +
\frac{\log n}{n} \sum_{k=1}^K | \widehat E_{y,\gamma}^k |
\end{align*}
%
where $\gamma$ in subscript indicates the corresponding quantity is calculated taking $\gamma$ as the tuning parameter, and $\bfS_y^k := (\bfY^k - \bfX^k \bfB^k)^T (\bfY^k - \bfX^k \bfB^k)/n$. Every time $\widehat \Theta$ is updated in the JMMLE algorithm, we choose the optimal $\gamma$ as the one with the smallest BIC over a fixed set of values $\cC_n$. Thus for a fixed $\lambda$, our final choice of $\gamma$ will be 
$
\gamma^* (\lambda) = \argmin_{\gamma \in \cC_n} \text{BIC} (\gamma; \widehat \cB_\lambda)
$.

We use the High-dimensional BIC (HBIC) to select the other tuning parameter, $\lambda$:
%
\begin{align*}
\text{HBIC} (\lambda; \Theta) &=
\frac{1}{n} \sum_{j=1}^q \sum_{k=1}^K \| \bfY^k_j - (\bfY_{-j}^k - \bfX^k \widehat \bfB_{-j,\lambda}^k ) \bftheta_j^{k } - \bfX^k \widehat \bfB_{j,\lambda}^k \|^2 +\\
& \log (\log n) \frac{\log (pq)}{n} \sum_{k=1}^K
\left( \| \bfB^k \|_0 + | \widehat E_{y, \gamma^* (\lambda)}^k| \right)
\end{align*}
%
We choose an optimal $\lambda$ as the minimizer of HBIC by training multiple JMMLE models using Algorithm \ref{algo:jmmle-algo} over a finite set of values $\lambda \in \cD_n$: 
$
\lambda^* = \argmin_{\lambda \in \cD_n} \text{HBIC} (\lambda, \widehat \Theta_{\gamma^*(\lambda)})
$.

%\begin{enumerate}
%\item Run neighborhood selection on $y$-network incorporating effects of $x$-data and an additional blockwise group penalty:
%%
%
%%
%where $\Theta = \{ \Theta_i \}, \cB = \{ \bfB^k \}, \cY = \{ \bfY^k \}, \cX = \{ \bfX^k \}, \cE = \{ \bfE^k \}$.
%
%This estimates $\cB$ { \colrbf (possibly refit and/or within-group threshold) }.
%
%\item Step I part 2 and step II of JSEM (see 15-656 pg 6) follows to estimate $\{ \Omega_y^k \}$.
%\end{enumerate}

%The objective function is bi-convex, so we are going to do the following in step 1-
%
%\begin{itemize}
%\item Start with initial estimates of $\cB$ and $\Theta$, say $\cB^{(0)}, \Theta^{(0)}$.
%\item Iterate:
%%
%\begin{align}
%\Theta^{(t+1)} &= \argmin \left\{ f ( \cY, \cX, \cB^{(t)}, \Theta^{(t)}) + P (\Theta^{(t)}) \right\}\\
%\cB^{(t+1)} &= \argmin \left\{ f ( \cY, \cX, \cB^{(t)}, \Theta^{(t+1)}) + Q (\cB^{(t)}) \right\}
%\end{align}
%\item Continue till convergence.
%\end{itemize}
%%

\subsection{Properties of JMMLE estimators}

We now provide theoretical results ensuring the convergence of our alternating algorithm, as well as the consistency of estimators obtained from the algorithm. We present statements of theorems in the main paper, giving detailed proofs and auxiliary results in the Appendix.

We introduce some notations that help establish the theorems that follow. Denote the true values of the parameters as $\Omega_{x 0} = \{ \Omega_{x 0}^k \}, \Omega_{y 0} = \{ \Omega_{y 0}^k \}, \Theta_0 = \{ \Theta_{0 j} \}, \cB_0 = \{ \cB_0^k \}$. The notation $\supp(\bfA)$ indicates the non-zero edge set in a matrix (or vector)valued parameter $\bfA \in \BM(A,B)$, i.e. $\supp(\bfA) = \{(a,b): (\bfA)_{ab} \neq 0\}$. Sparsity levels of individual true parameters are indicated by $s_j := | \supp (\Theta_j)|, b_k := | \supp(\bfB^k) |$. Also define $S := \sum_{j=1}^q s_j, B:= \sum_{k=1}^K b_k, s:= \max_{j \in \cI_q } s_j$. For positive real numbers $A, B$ we write $A \succsim B$ if there exists $c>0$ independent of $A,B$ such that $A \geq cB$. 

Our first result establishes the convergence of Algorithm~\ref{algo:jmmle-algo} for any fixed realization of $\cX$ and $\cE$.

\begin{Theorem}
Suppose for any fixed $(\cX, \cE)$, estimates in each iterate of Algorithm~\ref{algo:jmmle-algo} are uniformly bounded by some quantity dependent on only $p, q$ and $n$:
%
\begin{align}
\left\| (\widehat \cB^{(t)}, \widehat \Theta_y^{(t)}) - ( \cB_0, \Theta_{y 0}) \right\|_F
\leq R(p,q,n);
\quad t \geq 1
\end{align}
%
Then any limit point $(\cB^\infty, \Theta_y^\infty)$ of the algorithm is a stationary point of the objective function, i.e. a point where partial derivatives along all coordinates are non-negative.
\end{Theorem}

The next steps are to show that for random realizations of $\cX$ and $\cE$,
%
\begin{enumerate}

\item[{\bf (a)}] Successive iterates lie in this non-expanding ball around the true parameters,

\item[{\bf (b)}] The procedures in \eqref{eqn:init-B} and \eqref{eqn:init-Theta} ensure starting values that lie inside the same ball,
\end{enumerate}
%
both with probability approaching 1 as $(p,q,n) \rightarrow \infty$.

To do so we break down the main problem into two subproblems. Take as $\bfbeta = (\ve (\bfB^1)^T, \ldots, \ve(\bfB^K)^T)^T$: any subscript or superscript on $\bfB$ being passed on to $\bfbeta$. Denote by $\widehat \bfbeta$ and $\widehat \Theta$ the generic estimators given by
%
\begin{align}
\widehat \bfbeta &= \argmin_{\bfbeta \in \BR^{pqK}} \left\{-2 \bfbeta^T \widehat \bfgamma + \bfbeta^T \widehat \bfGamma \bfbeta + \lambda \sum_{h \in \cH} \| \bfbeta^{[h]}  \| \right\} \label{eqn:EstEqn1}\\
\widehat \Theta_j &= \argmin_{\Theta_j \in \BM(q-1, K)} \left\{ \frac{1}{n} \sum_{k=1}^K \| \widehat \bfE^k_j - \widehat \bfE^k_{-j} \bftheta_j^k \|^2 + \gamma \sum_{j \neq j'} \sum_{g \in \cG_y^{jj'}} \| \bftheta_{jj'}^{[g]} \| \right\};
\quad j \in \cI_q \label{eqn:EstEqn2}
\end{align}
%
where
%
$$
\widehat \bfGamma = \begin{bmatrix}
(\widehat \bfT^1)^2 \otimes \frac{(\bfX^1)^T \bfX^1}{n} & &\\
& \ddots &\\
& & (\widehat \bfT^K)^2 \otimes \frac{(\bfX^K)^T \bfX^K}{n}
\end{bmatrix}; \quad
\widehat \bfgamma = \begin{bmatrix}
(\widehat \bfT^1)^2 \otimes \frac{(\bfX^1)^T}{n}\\
\vdots\\
(\widehat \bfT^K)^2 \otimes \frac{(\bfX^K)^T}{n}
\end{bmatrix}
\begin{bmatrix}
\ve (\bfY^1)\\
\vdots\\
\ve (\bfY^K)
\end{bmatrix}
$$
with 
%
\begin{align}\label{eqn:define-T}
\hat T_{jj'}^k = \begin{cases}
1 \text{ if } j = j'\\
- \hat \theta_{jj'}^k \text{ if } j \neq j'
\end{cases}
\end{align}
%
It is easy to see that solving for $\cB$ in \eqref{eqn:jmmle-objfun} given a fixed $\widehat \Theta$ is equivalent to solving \eqref{eqn:EstEqn1}.

We assume the following conditions on the true parameter versions $(\bfT_0^k)^2$, defined from $\Theta_0$ similarly as \eqref{eqn:define-T}:

\vspace{1em}
\noindent{\bf (T1)} The matrices $(\bfT^k)^2, k \in \cI_K$ are diagonally dominant, i.e. %
%
$$
|t_{0,jj}^k | > \sum_{j' \neq j} |t_{0,jj'}^k |
$$
%
for $j \in \cI_q, k \in \cI_K$.
\vspace{1em}
%

\noindent Now we are in a position to establish the estimation consistency for the solution of \eqref{eqn:EstEqn1}, given random $(\cX, \cE)$ and good enough estimators $\widehat \Theta$.

\begin{Theorem}\label{thm:thm-B}
Assume random $(\cX, \cE)$, and fixed $\widehat \Theta$ so that for $j \in \cI_q$,
%
\[
\| \widehat \Theta_j - \Theta_{0,j} \|_F \leq v_\Theta  = \eta_\Theta \sqrt{\frac{\log q}{n}}
\]
%
for some $\eta_\Theta > 0$ dependent on $\Theta$ only. Then, given the choice of tuning parameter
%
$$
\lambda_n \geq 4 \sqrt{| h_{\max} |} \BR_0 \sqrt{ \frac{ \log(pq)}{n}}; \quad 
\BR_0 := \max_{k \in \cI_K} \BR \left(v_\Theta, \Sigma_x^k, \Sigma_y^k \right)
$$
%
the following hold
%
\begin{align}
\| \widehat \bfbeta - \bfbeta_0 \|_1 & \leq 48 \sqrt{ | h_{\max} |} B \lambda/ \psi_* \label{eqn:BetaThmEqn1}\\
\| \widehat \bfbeta - \bfbeta_0 \| & \leq 12 \sqrt B \lambda/ \psi_* \label{eqn:BetaThmEqn2}\\
\sum_{h \in \cH} \| \bfbeta^{[h]} - \bfbeta_0^{[h]} \| & \leq 48 B \lambda/ \psi_* \label{eqn:BetaThmEqn3}\\
(\widehat \bfbeta - \bfbeta_0 )^T \widehat \bfGamma (\widehat \bfbeta - \bfbeta_0 ) & \leq
72 B \lambda^2 / \psi_* \label{eqn:BetaThmEqn4}
\end{align}
%
with probability $\geq 1 - 12 c_1 \exp(c_2 \log(pq) - 2 \exp( -c_3 n)$, where $|h_{\max}| = \max_{h \in \cH} |h|$ and
%
$$
\psi_*= \frac{1}{2} \min_k \left[ \Lambda_{\min} (\Sigma_{x 0}^k) \left( \min_j \psi_j^k - d_k v_\Theta \right) \right],
\text{ with }
\psi_j^k := t_{0,jj}^k - \sum_{j' \neq j} t_{0,jj'}^k
$$
%
and $d_k$ being the maximum degree of $(\bfT_0^k)^2$.
\end{Theorem}

%{\bf summary}
%\begin{itemize}
%\item Say what to do
%\item Define extra notations
%\item Define conditions
%\item State results
%\end{itemize}

To prove an equivalent result for the solution of \eqref{eqn:EstEqn2}, as well as the consistency of the final estimates $\widehat \Omega_y^k$ using their support sets, we need the following conditions.

\vspace{1em}
\noindent{\bf (T2)} For $k \in \cI_K$, $\Omega_{y0}^k$ is diagonally dominant, i.e.
%
$
|\omega_{y0,jj} | > \sum_{j' \neq j} |\omega_{y0,jj'} |
$
for $j \in \cI_q$;

\noindent{\bf (T3)} There exist constants $c_0, d_0$ such that for $k \in \cI_K$,
%
\[
0 < 1/c_0 \leq \Lambda_{\min} (\Sigma_{y 0}^k) \leq \Lambda_{\max} (\Sigma_{y 0}^k) \leq 1/d_0 < \infty
\]
%
%\vspace{1em}

\noindent Given these, we establish the required consistency results.

\begin{Theorem}\label{corollary:thm-Omega}
Consider any deterministic $\widehat \cB$ that satisfy the following bound
%
$$
\| \widehat \bfB^k - \bfB_0^k \|_1 \leq v_\beta = \eta_\beta \sqrt{ \frac{ \log(pq)}{n}}
$$
%
Then, for sample size $n \succsim \log (pq)$ and choice of tuning parameter $\gamma_n = 4 \sqrt{| g_{\max}|} \BQ_0$, there exist constants $ c_1, c_2, c_3, c_4, c_5 > 0$ such that the following holds
%
\begin{align}\label{eqn:OmegaBounds}
\frac{1}{K} \sum_{k=1}^K \| \widehat \Omega_y^k - \Omega_y^k \|_F \leq
O \left( \BQ_0 \sqrt{\frac{| g_{\max}| S}{K}} \right)
\end{align}
%
with probability $\geq 1 - 1/p^{\tau_1-2} - 6c_1 \exp [-c_2 \log(pq)] - 2 \exp (- c_3 n) - 6c_4 \exp [-c_5 \log(pq)]$.
\end{Theorem}

Finally we ensure that the starting values are good enough.

\begin{Theorem}
Consider the starting values as derived in \eqref{eqn:init-B} and \eqref{eqn:init-Theta}. For sample size $n \succsim \log(pq)$, there exist constants $d_1, d_2, d_3>0$ such that for
%
\[
\lambda \geq 4 d_2 \max_{k \in \cI_K} \left\{ [\Lambda_{\max} (\Sigma_{x 0}^k) \Lambda_{\max} (\Sigma_{y 0}^k)]^{1/2} \right\}
\sqrt{ \frac{\log (pq)}{n}}
\]
%
we have $\| \widehat \bfbeta^{(0)} - \bfbeta_0 \|_1 \leq 64 B \lambda/\psi^*$ with probability $\geq 1 - 6d_1 \exp( -d_2 \log(pq)) - 2 \exp(d_3 n)$. Further, for $\gamma \geq 4\sqrt{| g_{\max}|} \BQ_0$ we have
%
\begin{align}
\| \widehat \Theta_j - \Theta_{0,j} \|_F & \leq 12 \sqrt{s_j} \gamma / \psi \label{eqn:theta-norm-bound-1}\\
\sum_{j \neq j', g \in \cG_y^{jj'}} \| \hat \bftheta_{jj'}^{[g]} - \bftheta_{0,jj'}^{[g]} \| & \leq 48 s_j \gamma / \psi \label{eqn:theta-norm-bound-2}
\end{align}
%
with probability $\geq 1 - 1/p^{\tau_1-2} - 6c_1 \exp [-(c_2^2-1) \log(pq)] - 2 \exp (- c_3 n) - 6c_4 \exp [-(c_5^2-1) \log(pq)]$.
\end{Theorem}
%

Putting everything together, estimation consistency for the limit points of Algorithm~\ref{algo:jmmle-algo} given our choice of starting values is immediate.

\begin{Corollary}
Assume the conditions (T1)-(T3), and starting values $\{ \cB^{(0)}, \Theta^{(0)} \}$ obtained using \eqref{eqn:init-B} and \eqref{eqn:init-Theta}, respectively. Then, for random realizations of $\cX, \cE$,
%

\vspace{1em}
\noindent (I) For the choice of $\lambda$
%
$$
\lambda \geq 4 \max \left[ d_2 \max_{k \in \cI_K} \left\{ [\Lambda_{\max} (\Sigma_{x 0}^k) \Lambda_{\max} (\Sigma_{y 0}^k)]^{1/2} \right\}, \sqrt{| h_{\max}|} \BR_0 \right] \sqrt{\frac{\log(pq)}{n}}
$$
%
we have
%
$$
\| \widehat \bfbeta - \bfbeta_0 \|_1 \leq \max \left\{ 48 \sqrt{ | h_{\max} |} + 64 \right\} \frac{B \lambda}{\psi_*}
$$
%
with probability $\geq {\colrbf tbd}$.

\vspace{1em}
\noindent (II) For $\gamma \geq 4 \sqrt{ | g_{\max} |} \BQ_0 $, \eqref{eqn:theta-norm-bound-1} and \eqref{eqn:theta-norm-bound-2} hold with probability $\geq {\colrbf tbd}$.

\vspace{1em}
\noindent (III) For $\gamma = 4 \sqrt{ | g_{\max} |} \BQ_0 $, \eqref{eqn:OmegaBounds} holds with probability $\geq {\colrbf tbd}$.
\end{Corollary}