\section{Real data example}
\label{sec:secreal}
We now apply the proposed methodology to breast cancer data obtained from The Cancer Genome Atlas\footnote{\url{https://www.genome.gov/Funded-Programs-Projects/Cancer-Genome-Atlas}}. The data set consists of mRNA and RNAseq expression values for $3980$ genes, divided into 88 pathways, for $n_1 = 262$ estrogen receptor positive or ER-positive (ER+) and $n_2 = 76$ ER-negative (ER--) breast cancer patients. 
As preprocessing steps, we consider single-pathway genes, and fit coordinate-wise lasso models to each column of the log-transformed response matrices $\bfY^k$, with $\bfX^k$ as predictors (say Lasso$(\bfy_{j}^k \sim \bfX^k)$). We then take the top 100 columns of each $\bfY^k$ that have lowest prediction errors (say $\cS^k$), and take unions of these column indices (i.e. $\cS := \cS^1 \cup \cS^2$) to construct the final response matrices $\bfY^k \rightarrow \bfY^k_\cS$. This gives us the final response dimension as $q = 166$. To select columns of $\bfX^k$, we take the top 200 predictor indices that have the highest mean absolute coefficient values across the lasso models on the selected response indices, i.e. Lasso$(\bfy_{j}^k \sim \bfX^k)$ where $j \in \cS$, and take the union of these indices. The resulting predictor dimension is $p = 339$.

Our objective here is to (a) obtain mRNA-mRNA, mRNA-RNAseq and RNAseq-RNAseq networks for the ER+ and ER-- groups while incorporating pathway information, (b) test for differential strengths of mRNA-RNAseq connections between the two sample groups. To this end, we take mRNA and RNAseq expression data as the top and bott0m layers ($X$ and $Y$ in our nomeclature), respectively, and consider pathway-wise groups. Note that gene expression (data in the $Y$ layer) is controlled on two levels. First, transcription is controlled by limiting the amount of mRNA (data in the $X$ layer) that is produced from a particular gene. The second level of control is through post-transcriptional events that regulate the translation of mRNA into proteins. For comparison purposes, we apply JMMLE and the separate estimation method \citet{LinEtal16} for estimating $\cB$ and $\Omega_y$, and JSEM for estimating $\Omega_y$. For comparing estimation performances of the methods, we use the following performance metrics calculated over 100 random 80:20 train-test splits of samples within each group:
%
\begin{itemize}
\item Root Mean Squared Scaled Prediction Error-
%
$$ \text{RMSSPE}(\widehat \cB, \widehat \Omega_y) = \left[
\sum_{k=1}^K \frac{1}{n_k} \Tr \left(\bfY^k - \bfX^k \widehat \bfB^k)^T (\bfY^k - \bfX^k \widehat \bfB^k)
\widehat\Omega_y^k \right)
\right]^{1/2} $$

\item Proportion of non-zero coefficients in $\widehat \cB$-
%
$$ \text{NZ}(\widehat \bfB^k) = \frac{| \supp(\hat \bfB^k) |}{pq}; \quad
\text{NZ}(\widehat \cB) = \frac{1}{K} \sum_{k=1}^K \text{NZ}(\widehat \bfB^k) $$

\item Proportion of non-zero coefficients in off-diagonal entries of $\widehat \Omega_y$-
%
$$ \text{NZ}(\widehat \Omega_y^k) = \frac{| \supp(\hat \Omega_y^k) - q|}{q^2}; \quad
\text{NZ}(\widehat \Omega_y) = \frac{1}{K} \sum_{k=1}^K \text{NZ}(\widehat \Omega_y^k) $$
\end{itemize}

\begin{table}[t]
\centering
%\begin{scriptsize}
    \begin{tabular}{llll}
    \hline
~        & RMSSPE        & NZ$(\widehat \cB)$  & NZ$(\widehat \Omega_y)$  \\\hline
JMMLE    & 14.38 (3.27) & 0.014 (0.004) & 0.077 (0.009) \\
Separate & 17.1 (2.26)  & 8.8 $\times 10^{-5}$ (3.5 $\times 10^{-4}$) & 0.085 (0.077) \\
JSEM     & 18.19 (4.04) & 0 (0)             & 0.09 (0.002) \\\hline
    \end{tabular}
    \caption{Performance metric comparison over 100 random splits of the real data}
    \label{table:real-compare}
%\end{scriptsize}
\end{table}
%
Table~\ref{table:real-compare} presents the comparison results. JMMLE and the separate estimation procedure obtain about the same amount of non-zero coefficients in $\Omega_y$ on average. Estimation of only the lower layer coefficients detects the most entries in the precision matrices $\widehat \Omega_y$, but has the highest prediction errors (calculated using $\widehat \cB = 0$). Separate estimation also hardly detects any non-zero elements in $\cB$, while JMMLE detects around 1.4\% of the inter-layer connection as non-zero. As a result, prediction errors are much lower for JMMLE.

To summarize within-layer and between-layer interactions, we consider the 10 highest entries in $\widehat \bfB^k, \widehat \Omega_y^k; k = 1,2$ in terms of absolute value. Table \ref{table:top10} gives their magnitudes, as well as the corresponding mRNA-RNAseq and RNAseq-RNAseq pairs. For the sake of comparison, we also report the same numbers and mRNA-mRNA pairs from the analysis of only the top layer using JSEM~\citep{MaMichailidis15}. According to our findings, the mRNA SQRDL\_3 is involved in downregulation of a number of RNA sequences in both groups of samples. Expression of the SQRDL gene is positively associated with high macrophage activity \citep{LyonsEtal17}, and its lower expression has been associated with breast cancer \citep{LiuEtal07,PiresEtal18}. In the ER+ group, KCNN3\_3 seems to be more heavily involved in doing so than ER--, whereas this is the case for TAF9\_3 in the ER-- group vs. ER+. Considering the crucial roles of these genes in cancer cell migration, drug resistance (KCNN3, \citet{kcnn3}) and estrogen signalling (TAF9, \citet{taf9}), the evidence of differential expression is be significant knowledge in developing subtype-specific therapeutic targets.

\begin{table}[t!]
\centering
%\begin{scriptsize}
    \begin{tabular}{l|lll|lll}
    \hline
Sample group & \multicolumn{3}{c}{ER+ $(k=1)$} & \multicolumn{3}{c}{ER- $(k=2)$} \\\hline
    ~ & Value & mRNA      &  RNAseq   & Value  & mRNA      &  RNAseq   \\\cline{2-7}
    ~ & -5.87 & TAF9\_3   & TRA2B\_1  & -11.32 & TAF9\_3   & TRA2B\_1  \\
    ~ & -5.7  & KCNN3\_3  & THOC7\_1  & -6.01  & TAF9\_3   & UQCRQ\_1  \\
    ~ & 4.9   & SQRDL\_3  & COX6A1\_1 & -5.38  & TAF9\_3   & TAF9\_1   \\
    ~ & 4.35  & SQRDL\_3  & ATP5G3\_1 & 5.17   & SQRDL\_3  & COX6A1\_1 \\
Conections & -4.34 & KCNN3\_3  & PABPN1\_1 & 5.14   & SQRDL\_3  & ACTR3\_1  \\
in $\widehat \cB$ & 4.31  & SQRDL\_3  & ACTR3\_1  & 4.55   & SQRDL\_3  & SSU72\_1  \\
    ~ & -4.21 & KCNN3\_3  & SNRPD2\_1 & -4.52  & KCNN3\_3  & THOC7\_1  \\
    ~ & 3.98  & CYP7B1\_3 & ECH1\_1   & 4.4    & UNG\_3    & COX6A1\_1 \\
    ~ & 3.88  & SQRDL\_3  & SSU72\_1  & -4.18  & TAF9\_3   & ATP5J\_1  \\
    ~ & 3.87  & CYP7B1\_3 & FTH1\_1   & 4.17   & CYP7B1\_3 & FTH1\_1   \\ \hline
    \hline
    ~ & Value &    RNASeq1 &  RNAseq2  & Value &  RNAseq1 &  RNAseq2     \\\cline{2-7}
    ~ & -0.27 & ECH1\_1    & PIGY\_1   & -0.16 & THOC7\_1 & PABPN1\_1    \\
    ~ & -0.25 & THOC7\_1   & PABPN1\_1 & -0.12 & RBBP4\_1 & PABPN1\_1    \\
    ~ & -0.22 & COX6A1\_1  & SF3B5\_1  & -0.11 & NAPA\_1  & CD63\_1      \\
    ~ & -0.21 & PCBP1\_1   & SH3GL1\_1 & -0.11 & SOD1\_1  & SNRPD3\_1    \\
Conections & -0.21 & ECH1\_1    & DDX42\_1  & -0.1  & EIF3I\_1 & TXNL4A\_1    \\
in $\widehat \Omega_y$ & -0.19 & EXOSC2\_1  & QARS\_1   & -0.1  & PCBP1\_1 & SH3GL1\_1    \\
    ~ & -0.19 & QARS\_1    & PIGY\_1   & -0.1  & TAF9\_1  & COX7C\_1     \\
    ~ & -0.18 & ECH1\_1    & SDHC\_1   & -0.1  & ECH1\_1  & HNRNPA1L2\_1 \\
    ~ & -0.18 & PABPN1\_1  & VAMP8\_1  & -0.1  & KARS\_1  & FUNDC1\_1    \\
    ~ & -0.18 & EIF3I\_1   & TXNL4A\_1 & -0.1  & ECH1\_1  & PIGY\_1      \\ \hline
    \hline
    ~ & Value &    mRNA1 & mRNA2     & Value & mRNA1    & mRNA2     \\\cline{2-7}
    ~ & -0.32 & GP1BB\_3 & COX6A2\_3 & -0.19 & PTPRC\_3 & ITGAL\_3  \\
    ~ & -0.32 & PTTG1\_3 & PTTG2\_3  & -0.17 & PTPRC\_3 & CD2\_3    \\
    ~ & -0.3  & ABCA8\_3 & C7\_3     & -0.16 & PDCD1\_3 & ICOS\_3   \\
    ~ & -0.3  & PTPRC\_3 & CD2\_3    & -0.16 & PDCD1\_3 & CD2\_3    \\
Conections & -0.29 & ABCA8\_3 & FXYD1\_3  & -0.16 & PTPRC\_3 & CYBB\_3   \\
in $\widehat \Omega_x$ & -0.28 & PDCD1\_3 & ICOS\_3   & -0.15 & GP1BB\_3 & COX6A2\_3 \\
    ~ & -0.26 & PDCD1\_3 & CD2\_3    & -0.15 & PTPRC\_3 & IL2RG\_3  \\
    ~ & -0.25 & CD6\_3   & PDCD1\_3  & -0.15 & PTPRC\_3 & CTSS\_3   \\
    ~ & -0.25 & PTPRC\_3 & PTGER4\_3 & -0.14 & PTPRC\_3 & PTGER4\_3 \\
    ~ & -0.25 & LAT\_3   & PDCD1\_3  & -0.14 & LCP2\_3  & CYBB\_3   \\ \hline
    \end{tabular}
%\end{scriptsize}
\caption{Top 10 within-layer and between-layer connections obtained by JMMLE.}
\label{table:top10}
\end{table}

\begin{table}[t!]

\begin{minipage}{.5\linewidth}
\caption*{(a)}
\centering
\begin{tabular}{ll}
\hline mRNA & Statistic \\\hline
DCTN2\_3   & 17015.2   \\ 
ST8SIA1\_3 & 13514.6   \\ 
FUT5\_3    &  8315.7   \\ 
XPA\_3     &  7194.2   \\ 
RETSAT\_3  &  5676.0   \\ 
TAF4B\_3   &  5385.8   \\ 
CYP7B1\_3  &  4189.6   \\ 
UNG\_3     &  3709.1   \\ 
RAD23A\_3  &  2793.8   \\ 
TAF9\_3    &  2427.1   \\ 
%ARHGDIB\_3 &  1826.9   \\ 
%SQRDL\_3   &  1498.0   \\ 
%CH25H\_3   &  1045.4   \\ 
%MMP2\_3    &  1001.2   \\ 
%ST3GAL6\_3 &   957.6   \\ 
%KCNN3\_3   &   809.9   \\ 
%RDH16\_3   &   711.6   \\ 
%PTPLAD2\_3 &   626.4   \\ 
%PGM2\_3    &   605.3   \\ 
%MCAT\_3    &   353.3   \\ 
%HSD17B6\_3 &   230.1   \\ 
%BDH2\_3    &   223.4   \\ 
%NNMT\_3    &   217.2   \\
\hline
\end{tabular}
\end{minipage}
%\end{table}
% latex table generated in R 3.5.1 by xtable 1.8-2 package
% Sun Aug 09 13:54:31 2020
%\begin{table}[t!]
%\centering
\begin{minipage}{.5\linewidth}
\caption*{(b)}
\centering
\begin{tabular}{lll}
\hline
mRNA & RNAseq & Statistic \\\hline
DCTN2\_3   & EIF4A1\_1  & 2560.6    \\ 
DCTN2\_3   & ARPC4\_1   & 2021.8    \\ 
ST8SIA1\_3 & EIF4A1\_1  & 1948.2    \\ 
DCTN2\_3   & PAIP1\_1   & 1922.3    \\ 
DCTN2\_3   & SNX5\_1    & 1825.1    \\ 
DCTN2\_3   & SUMO3\_1   & 1817.6    \\ 
DCTN2\_3   & CETN2\_1   & 1779.2    \\ 
DCTN2\_3   & SF3B4\_1   & 1755.8    \\ 
ST8SIA1\_3 & ARPC4\_1   & 1516.5    \\ 
ST8SIA1\_3 & PAIP1\_1   & 1453.9    \\\hline
\end{tabular}
\end{minipage}
\caption{Hypothesis testing outputs from real data analysis: (a) top-10 mRNAs and their global test statistic ($D_i$) values, (b) top-10 mRNA-RNAseq pairs and their simultaneous test statistic ($d_{ij}$) values}
\label{table:realtesting}
\end{table}

After applying our debiasing procedure and performing the global test, 23 mRNA-s were determined to have significant differences in the corresponding rows across sample groups, i.e. between $\widehat \bfb_i^1$ and $\widehat \bfb_i^2$. Within connections of these mRNAs, 957 total mRNA-RNAseq connections were determined by the simultaneous testing procedure to have significant differences between their corresponding coefficients, i.e. between $\hat b_{ij}^1$ and $\hat b_{ij}^2$. Table~\ref{table:realtesting} summarizes the top-10 statistic values in each situation. The DCTN2\_3 mRNA shows significant differential interactions with a number of RNA sequences---up-regulation of the DCTN2 gene has previously been found to be associated with chemotherapy resistance in breast cancer patients \citep{dctn2}. 
