\section{Discussion}
\label{sec:sec5}
This work introduces an integrative framework for knowledge discovery in multiple multi-layer Gaussian Graphical Models. We exploit {\it a priori} known structural similarities across parameters of the multiple models to achieve estimation gains compared to separate estimation. More importantly, we derive results on the asymptotic distributions of generic estimates of the multiple regression coefficient matrices in this complex setup, and perform global and simultaneous testing for pairwise differences within the between-layer edges.

\paragraph{}
The JMMLE algorithm due to the incorporation of prior information about sparsity patterns improves on the theoretical convergence rates of the estimation method for {\it single} multi-layer GGMs (i.e. $K=1$) introduced in \citet{LinEtal16}. With our initial estimates, the method of \citet{LinEtal16} achieves the following convergence rates for the estimation of $\cB$ and $\Omega_y$, respectively (using Corollary 4 therein):
%
\begin{align*}
\| \widehat \bfbeta - \bfbeta_0 \|_F & \leq \sum_{k=1}^K O \left( \sqrt{ \frac{ b_k \log (p q)}{n}} \right), \\
\sum_{k=1}^K \| \widehat \Omega_y^k - \Omega_{y0}^k \|_F & \leq O \left( K \sqrt{ \frac{ (S + q) \log (pq)}{n}} \right).
\end{align*}
%
In comparison, JMMLE has the following rates:
%
\begin{align*}
\| \widehat \bfbeta - \bfbeta_0 \|_F & \leq O \left( \sqrt{ \frac{ |h_{\max}| B \log (p q)}{n}} \right), \\
\sum_{k=1}^K \| \widehat \Omega_y^k - \Omega_{y0}^k \|_F & \leq O \left( \sqrt{ \frac{ KS |g_{\max}| \log (p q)}{n}} \right).
\end{align*}
%
For $\cB$, joint estimation outperforms separate estimation when group sizes are small, so that $(| h_{\max}| B)^{1/2} < \sum_k b_k^{1/2}$. The estimation gain for $\Omega_y$ is more substantial, especially for higher values of $q$. This is corroborated by our simulation outputs (Tables \ref{table:simtable11} and \ref{table:simtable12}), where the joint estimates perform better for both sets of parameters, but the differences between RF errors obtained from joint and separate estimates tend to be lower for $\widehat \Omega_y$ than $\widehat \cB$.

\paragraph{}
We now discuss two immediate extensions of our hypothesis testing framework.

\vspace{1em}
\noindent (I) In recent work, \citet{Liu17} proposed a framework to test for structural similarities and differences across multiple {\it single layer} GGMs. For $K$ GGMs with precision matrices $\Omega^k = (\omega_{ii'}^k)_{i,i' \in \cI_p}$, a test for the partial correlation coefficients $\rho_{ii'}^{k} = - \omega_{ii'}^{k} / \sqrt{\omega_{ii}^{k} \omega_{i'i'}^{k}}$ using residuals from $pK$ separate penalized neighborhood regressions is developed, one for each variable of each GGM. To incorporate structured sparsity across $k$, our simultaneous regression techniques for all neighborhood coefficients (i.e. \eqref{eqn:jsem-model} and \eqref{eqn:EstEqn2}) can be used instead, to perform testing on the between-layer edges. Theoretical properties of this procedure can be derived using results in \citet{Liu17}, possibly with adjustments for our neighborhood estimates to adhere to the rate conditions for the constants $a_{n1}, a_{n2}$ therein to account for a diverging $(p,q,n)$ setup.

\noindent (II) For $K > 2$, detection of the following sets of inter-layer edges can be scientifically significant:
%
\begin{align*}
\cB_1 &= \left\{ (i,j): \sum_{1 \leq k < k' \leq K} \left(b_{0,ij}^k - b_{0,ij}^{k'} \right)^2 > 0; i \in \cI_p, j \in \cI_q \right\}\\
\cB_2 &= \left\{ (i,j): b_{0,ij}^1 = \cdots, b_{0,ij}^K \neq 0 \right\}\\
\cB_3 &= \left\{ (i,j): b_{0,ij}^1 = \cdots, b_{0,ij}^K = 0 \right\}
\end{align*}
%
e.g. detection of gene-protein interactions that are present, but may have different or same weights across $k$ ($\cB_1$ and $\cB_2$, respectively), and that are absent for all $k$ ($\cB_3$). The asymptotic result in Theorem~\ref{Thm:ThmTesting} continues to hold in this situation, and an extension of the global test (Algorithm~\ref{algo:AlgoGlobalTest}) is immediate. However, extending the FDR control procedure requires a technically more involved approach.

The strength of our proposed debiased estimator \eqref{eqn:DebiasedBeta} is that only generic estimates of relevant model parameters that satisfy general rate conditions are necessary to obtain a valid asymptotic distribution. This translates to a high degree of flexibility in choosing the method of estimation. Our formulation based on sparsity assumptions (Section~\ref{sec:algosection}) is a specific way (motivated by applications in Omics data integration) to obtain the necessary estimates. Sparsity may not be an assumption that is required or even valid in complex hierarchical structures from different domains of application. For different two-layer components in such multi-layer setups, low-rank, group-sparse or sparse methods (or a combination thereof) can be plugged into our alternating algorithm. Results analogous to those in Section~\ref{sec:jmmle-theory} need to be established for the corresponding estimators. However, as long as these estimators adhere to the convergence conditions (T1)-(T3), Theorem~\ref{Thm:ThmTesting} can be used to derive the asymptotic distributions of between-layer edges.

\paragraph{}
Finally, extending our framework to non-Gaussian data is of interest. As seen for the $K=1$ case in \citet{LinEtal16}, their alternating block algorithm continues to give comparable results under shrunken or truncated empirical distributions of Gaussian errors. Similar results may be possible in the general case, and improvements can come from modifying different parts of the estimation algorithm. For example, the estimation of the precision matrices based on restricted support sets using log-likelihoods in \eqref{eqn:omega-y-calc} can be replaced by methods like nonparanormal estimation \citep{LiuLaffertyWasserman09} or regularized score matching \citep{LinDrtonShojaie16}.