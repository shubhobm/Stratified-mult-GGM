%\documentclass[11pt,letterpaper]{article}
\documentclass[twoside,11pt]{article}
\usepackage{jmlr2e}

\pdfoutput=1

\usepackage{mycommands1,amssymb,amsmath,amsthm,color,outlines,cite,subfigure,epsfig}
\usepackage[small]{caption}
\usepackage{hyperref} % for linking references
\hypersetup{colorlinks = true, citecolor = blue, urlcolor = blue}

\usepackage{stackrel}

%\usepackage[round]{natbib}

% for algorithm
%\usepackage[noend]{algpseudocode}
%\usepackage{algorithm}

% DON'T change margins - should be 1 inch all around.
%\addtolength{\evensidemargin}{-.5in}
%\addtolength{\oddsidemargin}{-.5in}
%\addtolength{\textwidth}{0.9in}
%\addtolength{\textheight}{0.9in}
%\addtolength{\topmargin}{-.4in}

%\pagestyle{myheadings}
%\markboth{}{\underline{{\bf Notes: (do not circulate)} \hspace{4.5cm} {\sc  Ansu Chatterjee} \hspace{0.25cm}}}

\DeclareMathOperator*{\ve}{vec}
\DeclareMathOperator*{\diag}{diag }
\DeclareMathOperator*{\supp}{supp }
\DeclareMathOperator*{\Tr}{Tr}
\DeclareMathOperator*{\argmin}{arg\,min}
\DeclareMathOperator*{\argmax}{arg\,max}
\DeclareMathOperator*{\Th}{^{\text{th}}}

\makeatletter
\newcommand{\opnorm}{\@ifstar\@opnorms\@opnorm}
\newcommand{\@opnorms}[1]{%
  \left|\mkern-1.5mu\left|\mkern-1.5mu\left|
   #1
  \right|\mkern-1.5mu\right|\mkern-1.5mu\right|
}
\newcommand{\@opnorm}[2][]{%
  \mathopen{#1|\mkern-1.5mu#1|\mkern-1.5mu#1|}
  #2
  \mathclose{#1|\mkern-1.5mu#1|\mkern-1.5mu#1|}
}
\makeatother

%% Appendix theorem counter
\usepackage{chngcntr}
\usepackage{apptools}
\AtAppendix{\counterwithin{Theorem}{section}}
\numberwithin{equation}{section}

% Heading arguments are {volume}{year}{pages}{submitted}{published}{author-full-names}
% \jmlrheading{0}{2015}{0}{unknown}{unknown}{}

% Short headings should be running head and authors last names
\ShortHeadings{Joint Multiple Multi-layered Gaussian Graphical Models}{Majumdar and Michailidis}
\firstpageno{1}

\begin{document}

\newtheorem{Theorem}{theorem}[section]
\newtheorem{Lemma}[Theorem]{Lemma}
\newtheorem{Corollary}[Theorem]{Corollary}
\newtheorem{Proposition}[Theorem]{Proposition}
\newtheorem{Conjecture}[Theorem]{Conjecture}
\theoremstyle{definition} \newtheorem{Definition}[Theorem]{Definition}
\newtheorem{Example}{Example}[section]
\newtheorem{Algorithm}{Algorithm}
\newtheorem{Remark}{Remark}

\title{Joint Estimation and Inference for Data Integration Problems based on Multiple Multi-layered Gaussian Graphical Models}
\date{}
\author{\name Subhabrata Majumdar \email smajumdar@ufl.edu \\
       \addr University of Florida Informatics Institute\\
       Gainesville, FL, 32611, USA
      \AND
      \name George Michailidis\thanks{Corresponding Author. Post Address: 205 Griffin Floyd Hall, 1 University Ave, Gainesville, FL, 32611.} \email gmichail@ufl.edu \\
      \addr Department of Statistics and Computer \& Information Science \& Engineering \\
      University of Florida \\
      Gainesville, FL 32611, USA 
       }

\editor{}
\maketitle

\begin{abstract} %   <- trailing '%' for backward compatibility of .sty file
The rapid development of high-throughput technologies has enabled the generation of data from biological or disease processes that span multiple layers, like genomic, proteomic or metabolomic data, and further pertain to multiple sources, like disease subtypes or experimental conditions. In this work, we propose a general statistical framework based on Gaussian graphical models for horizontal (i.e. across conditions or subtypes) and vertical (i.e. across different layers containing data on molecular compartments) integration of information in such datasets. We start with decomposing the multi-layer problem into a series of two-layer problems. For each two-layer problem, we model the outcomes at a node in the lower layer as dependent on those of other nodes in that layer, as well as all nodes in the upper layer. We use a combination of neighborhood selection and group-penalized regression to obtain sparse estimates of all model parameters. Following this, we develop a debiasing technique and asymptotic distributions of inter-layer directed edge weights that utilize already computed neighborhood selection coefficients for nodes in the upper layer. Subsequently, we establish global and simultaneous testing procedures for these edge weights. Performance of the proposed methodology is evaluated on synthetic data.
\end{abstract}

\vspace{1em}
\begin{keywords}
Data integration; Gaussian Graphical Models; neighborhood selection; group lasso; high-dimensional asymptotics; multiple testing; false discovery rate
\end{keywords}
\newpage

%\section{Introduction}
The human body is a complex system, comprising of numerous regulatory networks that are connected within and between themselves. These networks have a natural hierarchy, for exmaple DNA variations (Copy Number variations, Single Nucleotide Polymorphisms) influence gene expressions, which in turn influence RNA and protein expressions. They can also come from multiple sources that have some similarity with each other, for example diseased and healthy individuals, different experimental conditions, or different subtypes of the same disease. Recent developments in high-throughput technologies have resulted in the generation of enormous amount of data covering each of the above and many more situations (e.g.The Cancer Genome Atlas (TCGA: \cite{Tomczak15})).

Figure~\ref{fig:multi2layer} gives a schematic representation of the horizontal and vertical structure of heterogeneous biological data as outlined above. A simultaneous analysis of all parameters in this complex layered structure is known as {\it data integration}. While it is common knowledge that this will result in a more comprehensive picture of the regulatory mechanisms behind diseases, phenotypes and biological processes in general, there is a dearth of rigorous methodologies that satisfactorily tackle all challenges that stem from attempts to perform data integration \citep{GomezCabreroEtal14,JoycePalsson06,GligPrzulj15}. A review of the present approaches towards achieving this goal, which are based mostly on specific case studies, can be found in \cite{GligPrzulj15} and \cite{ZhangOuyangZhao17}.

\paragraph{}
Gaussian Graphical Models (GGM) have been extensively used to model biological networks in the last few years. While the initial work on GGMs focused on estimating undirected edges within a single network through obtaining sparse estimates of the inverse covariance matrix, or the precision matrix from the data (e.g. see the references in \cite{BuhlmannvandeGeer11}), the attention has now shifted to estimating parameters from more complex structures, especially multiple related graphical models and hierarchical multilayer models with both directed and undirected edges. For the first problem, \cite{GuoEtal11} and \cite{XieLiuValdar16} assumed perturbations over a common underlying structure to model multiple precision matrices, while \cite{DanaherEtal14} proposed using fused lasso or group lasso penalties in a joint group lasso model for the same purpose. To incorporate prior information on the group structures across several graphs, \cite{MaMichailidis15} proposed the Joint Structural Estimation Method (JSEM), which uses group-penalized neighborhood regression and subsequent refitting for estimating precision matrices.

For the second problem, a two-layered structure can be modeled by interpreting directed edges between the two layers as elements of a multitask regression coefficient matrix, while undirected edges inside either layer correspond to the precision matrix of predictors in that layer. While several methods exist in the literature for joint estimation of both parameters \citep{RothmanEtal10, LeeLiu12, CaiEtal12}, only recently \cite{LinEtal16} made the observation that a multi-layer model can, in fact, be decomposed into a series of two-layer problems. Subsequently, they proposed an estimation algorithm and derived theoretical properties of the resulting estimators.

\begin{figure}
\centering
\includegraphics[]{multi2layer}
\caption{Multiple multilayer graphical models. The matrices $(\bfX^k, \bfY^k, \bfZ^k), k = 1,2,3$ indicate data for each layer and category $k$. Within-layer connections (black lines) are undirected, while between-layer connections (red lines) go from an upper layer to the successive lower layer. For each type of edges (i.e. within $\cX, \cY, \cZ$ and $\cX \rightarrow \cY, \cY \rightarrow \cZ$), there are common edges across some or all $k$. }
\label{fig:multi2layer}
\end{figure}

\paragraph{}
All the above approaches model either the horizontal or the vertical complexity in the full hierarchical structure of Figure~\ref{fig:multi2layer}. This means multiple related groups of heterogeneous datasets has to be modeled by analyzing all data in individual layers (i.e. models for $\{ \bfX^k \}$, $\{ \bfY^k \}$, $\{ \bfZ^k \}$), and then separately analyzing individual hierarchies of datasets (i.e. separate models for $(\bfX^k, \bfY^k, \bfZ^k), k = 1,2,3$). Although a recent paper \citep{ZhangOuyangZhao17} provides a model for the full structure in Figure~\ref{fig:multi2layer} using penalized log-likelihoods, they do not give theoretical guarantees for the estimates, and limit the numerical examples to small feature dimensions ($<70$) only. Thus, a truly rigorous and scalable model for data integration is yet to be proposed.

While there has been some progress for parameter estimation in multilayer models, little is known about the asymptotic properties of resulting estimates. Current research on asymptotic distributions and testing procedures for estimates from high-dimensional problems has been limited to single-response regression using lasso \citep{ZhangZhang14,JavanmardMontanari14,JavanmardMontanari18,vanDeGeerEtal14} or group lasso \citep{MitraZhang16} penalties, and partial correlations of single \citep{CaiLiu16} or multiple \citep{BelilovskyEtal16,Liu17} GGMs. From a systemic perspective, testing and identifying downstream interactions that differ across experimental conditions or disease subtypes can offer important insight on the underlying biological process \citep{MaoEtal17,LiEtal15}. In our framework, this can be done by developing a hypothesis testing procedure for entries in the within-layer regression matrices.

\paragraph{}
The contributions of this paper are two-fold. Firstly, we propose an integrative framework to conduct simultaneous inference for all parameters in multiple multilayer graphical models, essentially formalizing the structure in Figure~\ref{fig:multi2layer}. We decompose the multi-layer problem into a series of two-layer problems, incorporate prior information on structural dependencies through imposing group structures on the model parameters, propose an estimation algorithm for them and derive theoretical properties of the estimators. Secondly, we obtain debiased versions of within-layer regression coefficients in this two-layer model, and derive their asymptotic distributions using estimates of model parameters that satisfy generic convergence guarantees. Consequently, we formulate a global test, as well as a simultaneous testing procedure that controls for False Discovery Rate (FDR) to detect important pairwise differences among directed edges between layers.

Our proposed framework for knowledge discovery from heterogeneous data sources is highly flexible. The group sparsity assumptions in our estimation technique can be replaced by other structural restrictions, for example low-rank or low-rank-plus-sparse, as and when deemed appropriate by the prior dependency assumptions across parameters. As long as the resulting estimates converge to corresponding true parameters at certain rates, they can be plugged into the testing methodology.

\paragraph{Organization of paper.}
We start with the model formulation in Section~\ref{sec:sec2}, then introduce our computational algorithm for a two-layer model, and derive theoretical convergence properties of the algorithm and resulting estimates. In section~\ref{sec:sec3}, we start by introducing the debiased versions of rows of the regression coefficient matrix estimates in our model, then use already computed parameter estimates that satisfy some general consistency conditions to obtain its asymptotic distribution. We then move on to pairwise testing, and use sparse estimates from our algorithm to propose a global test to detect overall differences in rows of the coefficient matrices, as well as a multiple testing procedure to detect elementwise differences and perform within-row thresholding of estimates in presence of moderate misspecification of the group sparsity structure. Section~\ref{sec:sec4} is devoted to implementation of our methodology. We evaluate the performance of our estimation and testing procedure through several simulation settings, and give strategies to speed up the computational algorithm for high data dimensions.  We conclude the paper with a discussion in Section~\ref{sec:sec5}. Proofs of all theoretical results, as well as some auxiliary results, are given in the appendix.

\paragraph{Notations.}
We denote scalars by small letters, vectors by bold small letters and matrices by bold capital letters. For any matrix $\bfA$, $(\bfA)_{ij}$ denote its element in the $(i,j)^\text{th}$ position. For $a,b \in \BN$, we denote the set of all $a \times b$ real matrices by $\BM(a,b)$. For any positive integer $c$, define $\cI_c = \{ 1, \ldots, c\}$. For vectors $\bfv$ and matrices $\bfM$, $\| \bfv \|$, $\|\bfv \|_1$ or $\|\bfM \|_1$ and $\|\bfv \|_\infty$ or $\|\bfM \|_\infty$ denote euclidean, $\ell_1$ and $\ell_\infty$ norms, respectively. The notation $\supp(\bfA)$ indicates the non-zero edge set in a matrix (or vector) $\bfA$, i.e. $\supp(\bfA) = \{(i,j): (\bfA)_{ij} \neq 0\}$. For positive real numbers $A, B$ we write $A \succsim B$ if there exists $c>0$ independent of model parameters such that $A \geq cB$. We use the `$:=$' notation to define a quantity for the first time.
\section{Introduction}

Aberrations in complex biological systems develop in the background of diverse genetic and environmental factors and are associated with multiple complex molecular events. These include changes in the genome, transcriptome, proteome and metabolome, as well as epigenetic effects. Advances in high-throughput profiling techniques have enabled a systematic and comprehensive exploration of the genetic and epigenetic basis of various diseases, including cancer \citep{lee2016integrated, kaushik2016inhibition}, diabetes \citep{yuan2014integrated,sas2018shared}, chronic kidney disease \citep{atzler2014integrated}, etc. Further, such multi-Omics collections have become available for patients belonging to different, but related disease subtypes, with The Cancer Genome Atlas (TCGA: \cite{Tomczak15}) being a prototypical one. Hence, there is an increasing need for models that can {\em integrate} such complex data both {\em vertically} across multiple modalities and {\em horizontally} across different disease subtypes.

Figure~\ref{fig:multi2layer} provides a schematic representation of the horizontal and vertical structure of such heterogeneous multi-modal Omics data as outlined above. A simultaneous analysis of all components in this complex layered structure has been coined in the literature as {\it data integration}. While it is common knowledge that this will result in a more comprehensive picture of the regulatory mechanisms behind diseases, phenotypes and biological processes in general, there is a dearth of rigorous methodologies that satisfactorily tackle all challenges that stem from attempts to perform data integration \citep{GomezCabreroEtal14,JoycePalsson06,GligPrzulj15}. A review of the present approaches towards achieving this goal, which are based mostly on specific case studies, can be found in \cite{GligPrzulj15} and \cite{ZhangOuyangZhao17}.

\paragraph{}
Gaussian Graphical Models (GGM) have been extensively used to model biological networks in the last few years. While the initial work on GGMs focused on estimating undirected edges within a single network through obtaining sparse estimates of the inverse covariance matrix from high-dimensional data (e.g. see the references in \cite{BuhlmannvandeGeer11}), attention has shifted to estimating parameters from more complex structures, including multiple related graphical models and hierarchical multilayer models comprising of both directed and undirected edges. For the first class of problems, \cite{GuoEtal11} and \cite{XieLiuValdar16} assumed perturbations over a common underlying structure to model multiple precision matrices, while \cite{DanaherEtal14} proposed using fused/group lasso type penalties for the same task. To incorporate prior information on the group structures across several graphs, \cite{MaMichailidis15} proposed the Joint Structural Estimation Method (JSEM), which uses group-penalized neighborhood regression and subsequent refitting for estimating precision matrices.
For the second problem, a two-layered structure can be modeled by interpreting directed edges between the two layers as elements of a multitask regression coefficient matrix, while undirected edges inside either layer correspond to the precision matrix of predictors in that layer. While several methods exist in the literature for joint estimation of both parameters \citep{LeeLiu12, CaiEtal12}, only recently \cite{LinEtal16} made the observation that a multi-layer model can, in fact, be decomposed into a series of two-layer problems. Subsequently, they proposed an estimation algorithm and derived theoretical properties of the resulting estimators.

\begin{figure}
\centering
\includegraphics[]{multi2layer}
\caption{Multiple multilayer graphical models. The matrices $(\bfX^k, \bfY^k, \bfZ^k), k = 1,2,3$ indicate data for each layer and category $k$. Within-layer connections (black lines) are undirected, while between-layer connections (red lines) go from an upper layer to the successive lower layer. For each type of edges (i.e. within $\cX, \cY, \cZ$ and $\cX \rightarrow \cY, \cY \rightarrow \cZ$), there are common edges across some or all $k$. }
\label{fig:multi2layer}
\end{figure}

%\paragraph{}
All the above approaches focus either on the horizontal or the vertical dimensions of the full hierarchical structure depicted in Figure~\ref{fig:multi2layer}. Hence, multiple related groups of heterogeneous data sets have to be modeled by analyzing all data in individual layers (i.e. models for $\{ \bfX^k \}$, $\{ \bfY^k \}$, $\{ \bfZ^k \}$), and then separately analyzing individual hierarchies of datasets (i.e. separate models for $(\bfX^k, \bfY^k, \bfZ^k), k = 1,2,3$). In another line of work, \cite{KlingEtal15,ZhangOuyangZhao17} model all undirected edges within all nodes together using penalized log-likelihoods. The advantage of this approach is that it can incorporate feedback loops and connections between nodes in non-adjacent layers. However, it has two considerable caveats. Firstly, it does not distinguish between hierarchies, hence delineating the direction of a connection between two nodes is not possible in such models. Secondly, computation becomes difficult with increasing number of layers, as the number of estimable parameters increases at a faster late compared to a hierarchical model.% (as seen in the numerical examples of \cite{ZhangOuyangZhao17}, which are limited to small feature dimensions ($<$ 70) only {\colrbf is this needed?}).

While there has been some progress for parameter estimation in multilayer models, little is known about the sampling distributions of resulting estimates. Current research on such distributions and related testing procedures for estimates from high-dimensional problems has been limited to single-response regression using lasso \citep{ZhangZhang14,JavanmardMontanari14,JavanmardMontanari18,vanDeGeerEtal14} or group lasso \citep{MitraZhang16} penalties, and partial correlations of single \citep{CaiLiu16} or multiple \citep{BelilovskyEtal16,Liu17} GGMs. From a systemic perspective, testing and identifying downstream interactions that differ across experimental conditions or disease subtypes can offer important insights on the underlying biological process \citep{MaoEtal17,LiEtal15}. In the proposed integrative framework, this can be accomplished by developing a hypothesis testing procedure for entries in the within-layer regression matrices.

\paragraph{}
%The contributions of this paper are two-fold. Firstly, we propose an integrative framework to conduct simultaneous inference for all parameters in multiple multilayer graphical models, essentially formalizing the structure in Figure~\ref{fig:multi2layer}. We decompose the multi-layer problem into a series of two-layer problems, incorporate prior information on structural similarities through imposing group structures on the model parameters, propose an estimation algorithm for them and derive theoretical properties of the estimators. Secondly, we obtain {\em debiased} versions of within-layer regression coefficients in this two-layer model, and derive their asymptotic distributions using estimates of model parameters that satisfy generic convergence guarantees. Consequently, we formulate a global test, as well as a simultaneous testing procedure that controls for False Discovery Rate (FDR) to detect important pairwise differences among directed edges between layers.
%
%Our proposed framework for knowledge discovery from heterogeneous data sources is highly flexible. Imposing group structures on the model parameters allows us to incorporate prior information on within-layer or between-layer sub-graph components shared across some or all $k$. However, these group sparsity assumptions in our estimation technique can be replaced by other structural restrictions, for example low-rank or low-rank-plus-sparse, as and when deemed appropriate by the prior dependency assumptions across parameters. As long as the resulting estimates converge to corresponding true parameters at certain rates, they can be used by the developed testing methodology.

The contributions of this paper are two-fold. Firstly, we propose an integrative framework to conduct simultaneous inference for all parameters in multiple multi-layer graphical models, essentially formalizing the structure in Figure~\ref{fig:multi2layer}. We decompose the multi-layer problem into a series of two-layer problems, propose an estimation algorithm for them based on group penalization, and derive theoretical properties of the estimators. Imposing group structures on the model parameters allows us to incorporate prior information on within-layer or between-layer sub-graph components shared across some or all $k$. For biological processes, such information can stem from experimental or mechanistic knowledge (for example a pathway-based grouping of genes). Secondly, we obtain {\em debiased} versions of within-layer regression coefficients in this two-layer model, and derive their asymptotic distributions using estimates of model parameters that satisfy generic convergence guarantees. Consequently, we formulate a global test, as well as a simultaneous testing procedure that controls for False Discovery Rate (FDR) to detect important pairwise differences among directed edges between layers.

Our proposed framework for knowledge discovery from heterogeneous data sources is highly flexible. The group sparsity assumptions in our estimation technique can be replaced by other structural restrictions, for example low-rank or low-rank-plus-sparse, as and when deemed appropriate by the prior dependency assumptions across parameters. As long as the resulting estimates converge to corresponding true parameters at certain rates, they can be used by the developed testing methodology.

\paragraph{Organization of paper.}
We start with the model formulation in Section~\ref{sec:sec2}, then introduce our computational algorithm for a two-layer model, and derive theoretical convergence properties of the algorithm and resulting estimates. In section~\ref{sec:sec3}, we start by introducing the debiased versions of rows of the regression coefficient matrix estimates in our model, then use already computed parameter estimates that satisfy some general consistency conditions to obtain its asymptotic distribution. We then move on to pairwise testing, and use sparse estimates from our algorithm to propose a global test to detect overall differences in rows of the coefficient matrices, as well as a multiple testing procedure to detect elementwise differences and perform within-row thresholding of estimates in presence of moderate misspecification of the group sparsity structure. Section~\ref{sec:sec4} is devoted to implementation of our methodology. We evaluate the performance of our estimation and testing procedure through several simulation settings, and give strategies to speed up the computational algorithm for high data dimensions.  We conclude the paper with a discussion in Section~\ref{sec:sec5}. Proofs of all theoretical results, as well as some auxiliary results, are given in the appendix.

\paragraph{Notations.}
We denote scalars by small letters, vectors by bold small letters and matrices by bold capital letters. For any matrix $\bfA$, $(\bfA)_{ij}$ denote its element in the $(i,j)\Th$ position. For $a,b \in \BN$, we denote the set of all $a \times b$ real matrices by $\BM(a,b)$. For a positive semi-definite matrix $\bfP$, we denote its smallest and largest eigenvalues by $\Lambda_{\min} (\bfP)$ and $\Lambda_{\max} (\bfP)$, respectively. For any positive integer $c$, define $\cI_c = \{ 1, \ldots, c\}$. For vectors $\bfv$ and matrices $\bfM$, $\| \bfv \|$, $\|\bfv \|_1$ or $\|\bfM \|_1$ and $\|\bfv \|_\infty$ or $\|\bfM \|_\infty$ denote euclidean, $\ell_1$ and $\ell_\infty$ norms, respectively. The notation $\supp(\bfA)$ indicates the non-zero edge set in a matrix (or vector) $\bfA$, i.e. $\supp(\bfA) = \{(i,j): (\bfA)_{ij} \neq 0\}$. For any set $\cS$, $| \cS|$ denotes the number of elements in that set. For positive real numbers $A, B$ we write $A \succsim B$ if there exists $c>0$ independent of model parameters such that $A \geq cB$. We use the `$:=$' notation to define a quantity for the first time.
\section{The Joint Multiple Multilevel Estimation Framework}
\label{sec:sec2}

\subsection{Formulation}
Suppose there are $K$ independent datasets, each pertaining to an $M$-layered Gaussian Graphical Model (GGM). The $k^{\Th}$ model has the following structure:

\vspace{1em}
\begin{tabular}{ll}
{\it Layer 1}- &
%
$\BD_1^k = (D_{1 1}^k, \ldots, D^k_{1 p_1}) \sim
\cN (0, \Sigma_1^k); \quad k \in \cI_K,$\\
{\it Layer $m$} $(1< m \leq M)$-  &
%
$ \BD_m^k = \BD_{m-1}^k \bfB_m^k + \BE_m^k$, with $\bfB_m^k \in \BM(p_{m-1}, p_m) $\\
& and $\BE_m^k = (E_{m 1}^k, \ldots, E^k_{m p_m}) \sim
\cN (0, \Sigma_m^k); \quad k \in \cI_K $.\\
\end{tabular}
\vspace{1em}

We assume known structured sparsity patterns, denoted by $\cG_m$ and $\cH_m$, for the parameters of interest in the above model, i.e. the precision matrices $\Omega_m^k := (\Sigma_m^k)^{-1}$ and the regression coefficient matrices $\bfB_m^k$, respectively. These patterns provide information on horizontal dependencies across $k$ for the corresponding parameters, and our goal here is to leverage them to estimate the full hierarchical structure of the network- specifically to obtain the undirected edges inside nodes of a single layer, and the directed edges between two successive layers through jointly estimating $\{ \Omega_m^k \}$ and $\{ \bfB_m^k \}$.

Consider now a two-layer model, which is a special case of the above model with $M=2$:
%
\begin{eqnarray}
\BX^k = (X^k_1, \ldots, X^k_p)^T \sim \cN (0, \Sigma^k_{x});\\
\BY^k = \BX^k \bfB^k + \BE^k; \quad \BE^k = (E^k_1, \ldots, E^k_p)^T \sim \cN (0, \Sigma^k_{y});\\
\bfB^k \in \BM(p,q), \quad \Omega^k_{x} = (\Sigma^k_{x})^{-1}; \quad \Omega^k_y = (\Sigma^k_{y})^{-1};
\end{eqnarray}
%
where we want to estimate $\{ (\Omega^k_{x}, \Omega^k_{y}, \bfB^k); k \in \cI_K \}$ from data $\cZ^k = \{ (\bfY^k, \bfX^k); \bfY^k \in \BM(n,q), \bfX^k \in \BM(n,p), k \in \cI_K\}$ in presence of known grouping structures $\cG_x, \cG_y, \cH$ respectively and assuming $n_k = n$ for all $k \in \cI_K$ for simplicity. We focus the theoretical discussion in the rest of the paper on jointly estimating $\Omega_{y}:= \{ \Omega_{y}^k \}$ and $\cB := \{ \bfB^k \}$. This is because for $M>2$, within-layer undirected edges of any $m{\Th}$ layer $(m>1)$ and between-layer directed edges from the $(m-1){\Th}$ layer to the $m{\Th}$ layer can be estimated from the corresponding data matrices in a similar fashion. On the other hand, parameters in the very first layer are analogous to $\Omega_{x} := \{ \Omega_{x}^k \}$, and can be estimated from $\{ \bfX^k\}$ using any method for joint estimation of multiple graphical models (e.g. \cite{GuoEtal11, MaMichailidis15}). This provides all building blocks for recovering the full hierarchical structure of our $M$-layered multiple GGMs.

%Setting $M=1$ reduces the above model to joint estimation of GGMs with structured sparsity \citep{MaMichailidis15}, while setting $K=1$ reduces the model to a multi-layer GGM, which can be estimated by breaking it down to successive two-layer models and then minimizing a penalized conditional log-likelihood function \citep{LinEtal16}.

\subsection{Algorithm}
\label{sec:algosection}
We assume an element-wise group sparsity pattern over $k$ for the precision matrices $\Omega_x^k$:
%
\[
\cG_x = \{ \cG_x^{ii'}: i \neq i'; i, i' \in \cI_p \},
\]
%
where each $\cG_x^{ii'}$ is a partition of $\cI_K$, and consists of index groups $g$ such that $g \subseteq \cI_K, \cup_{g \in \cG_x^{ii'}} g = \cI_K$. Subsequently we use the Joint Structural Estimation Method (JSEM) \citep{MaMichailidis15} to estimate $\Omega_x$, which first uses the group structure given by $\cG_x$ in penalized nodewise regressions \citep{MeisenBuhlmann06} to obtain neighborhood coefficients $\zeta_i = (\bfzeta_i^1, \ldots, \bfzeta_i^K)$ of each variable $X_i, i \in \cI_p$, then fits a graphical lasso model over the combined support sets to obtain sparse estimates of the precision matrices:
%
\begin{align}\label{eqn:jsem-model}
\widehat \zeta_i &= \argmin_{\zeta_i} \left\{
\frac{1}{n} \sum_{k=1}^K \| \bfX_i^k - \bfX_{-i}^k \bfzeta_i^k \|^2 +
\sum_{i' \leq i} \sum_{g \in \cG_x^{ii'}} \eta_n \| \bfzeta_{ii'}^{[g]} \| \right\}, \notag\\
\widehat E_x^k &= \{(i,i'): 1 \leq i < i' \leq p, \hat \zeta_{ii'}^k \neq 0 \text{ OR } \hat \zeta_{i'i}^k \neq 0 \}, \notag\\
\widehat \Omega_x^k &= \argmin_{\Omega_x^k \in \BS_+ (\hat E_x^k)}
\left\{ \Tr (\widehat \bfS_x^k \Omega_x^k ) - \log \det (\Omega_x^k) \right\}.
\end{align}
%
where $\widehat \bfS_x^k := (\bfX^k)^T \bfX^k/n$ and $\eta_n$ is a tuning parameter.

For the precision matrices $\Omega_y^k$ we assume an element-wise sparsity pattern $\cG_y$ defined in a similar manner as $\cG_x$, while the sparsity pattern $\cH$ for $\cB$ is more general, each group $h \in \cH$ being defined as:
%
$$
h = \{ (\cS_p, \cS_q, \cS_K): \cS_p \subseteq \cI_p, \cS_q \subseteq \cI_q, \cS_K \subseteq \cI_K \}
; \quad \bigcup_{h \in \cH} h = \cI_p \times \cI_q \times \cI_K.
$$
%
Denote the neighborhood coefficients of the $j^{\Th}$ variable in the lower layer by $\bftheta_j^k$, and $\Theta_j := (\bftheta_j^1, \ldots, \bftheta_j^K), \Theta = \{ \Theta_j \}$. We obtain sparse estimates of $\cB, \Theta$, and subsequently $\Omega_y$, by solving the following group-penalized least square minimization problem that has the tuning parameters $\gamma_n$ and $\lambda_n$ and then refitting:
%
\begin{align}
\{ \widehat \cB, \widehat \Theta \} &= 
\argmin_{\cB, \Theta} \left\{ \frac{1}{n} \sum_{j=1}^q \sum_{k=1}^K \| \bfY^k_j - (\bfY_{-j}^k - \bfX^k \bfB_{-j}^k) \bftheta_j^k - \bfX^k \bfB_j^k \|^2 \right. \notag\\
& \left. + \sum_{j \neq j'} \sum_{g \in \cG_y^{jj'}} \gamma_n \| \bftheta_{jj'}^{[g]} \| + \sum_{h \in \cH} \lambda_n \| \bfB^{[h]} \| \right\}, \label{eqn:jmmle-objfun}\\
\widehat E_y^k &= \{(j,j'): 1 \leq j < j' \leq q, \hat \theta_{jj'}^k \neq 0 \text{ OR } \hat \theta_{j'j}^k \neq 0 \}, \notag\\
\widehat \Omega_y^k &= \argmin_{\Omega_y^k \in \BS_+ (\hat E_y^k)}
\left\{ \Tr (\widehat \bfS_y^k \Omega_y^k ) - \log \det (\Omega_y^k) \right\}. \label{eqn:omega-y-calc}
\end{align}
%&= \min \left\{ f ( \cY, \cX, \cB, \Theta) + P (\Theta) + Q (\cB) \right\} 
%
The outcome of a node in the lower layer is thus modeled using all other nodes in that layer using the neighborhood coefficients $\widehat \bfB_j^k$, {\it and} nodes in the immediate upper layer using the regression coefficients $\widehat \bftheta_j^k$.

\paragraph{Remark.} Common sparsity structures across the same layer are incorporated into the regression by the group penalties over the element-wise groups $\bftheta_{jj'}^{[g]}$, while sparsity pattern overlaps across the different regression matrices $\bfB^k$ are handled by the group penalties over $\bfB^{[h]}$, which denote the collection of elements in $\cB$ that are in $h$. Other kinds of structural assumptions on $\cB$ or $\Theta$ can be handled within the above structure by swapping out the group norms in favor of other appropriate norm-based penalties.

\subsubsection{Alternating algorithm}
The objective function in \eqref{eqn:jmmle-objfun} is bi-convex, i.e. convex in $\cB$ for fixed $\Theta$, and vice-versa, but not jointly convex in $\{ \cB, \Theta \}$. Consequently, we use an alternating iterative algorithm to solve for $\{ \cB, \Theta \}$ that minimizes \eqref{eqn:jmmle-objfun} by iteratively cycling between $\cB$ and $\Theta$, i.e. holding one set of parameters fixed and solving for the other, then alternating until convergence.

Choice of initial values plays a crucial role in the performance of this algorithm. % Following the analysis of \cite{LinEtal16}, who proposed an estimation framework for the special case of our multi-level structure for $K=1$ based on a similar algorithm, we choose the initial values $\{ \bfB^{k (0)} \}$ by first selecting a support set for the $j^{\Th}$ column, say $\tilde \cS_j^k$, as the support of the debiased lasso estimate of \cite{JavanmardMontanari14}, then fitting a Lasso regression model with small penalty value $\lambda_n^0$:
%
%\begin{align}\label{eqn:init-B}
%\widehat \bfB_j^{k (0)} = \argmin_{\supp(\bfB_j^k) \subseteq \tilde \cS_j^k} \|\bfY_j^k - \bfX^k \bfB_j^k \|^2 + \lambda_n^0 \| \bfB_j^k \|_1
%\end{align}
%
We choose the initial values $\{ \widehat \bfB^{k (0)} \}$ by fitting separate lasso regression models for each $j$ and $k$:
%
\begin{align}\label{eqn:init-B}
\widehat \bfB_j^{k (0)} = \argmin_{\bfB_j^k \in \BR^p} \|\bfY_j^k - \bfX^k \bfB_j^k \|^2 + \lambda_n \| \bfB_j^k \|_1; \quad
j \in \cI_q, k \in \cI_K.
\end{align}
%

We obtain initial estimates of $ \Theta_j, j \in \cI_q$ by performing group-penalized nodewise regression on the residuals $\widehat \bfE^{k (0)} := \bfY^k - \bfX^k \widehat \bfB_j^{k (0)}$:
%
\begin{align}\label{eqn:init-Theta}
\widehat \Theta_j^{(0)} = \argmin_{\Theta_j} \frac{1}{n} \sum_{k=1}^K \|
\widehat \bfE_j^{k (0)} - \widehat \bfE_{-j}^{k (0)} \bftheta_j^k \|^2
+ \gamma_n \sum_{j \neq j'} \sum_{g \in \cG_y^{jj'}} \| \bftheta_{jj'}^{[g]} \|.
\end{align}

The steps of our full estimation procedure, which we call the {\it Joint Multiple Multi-Layer Estimation} (JMMLE) method, can thus be summarized in Algorithm \ref{algo:jmmle-algo}.

\begin{Algorithm}
(The JMMLE Algorithm)
\label{algo:jmmle-algo}

\noindent 1. Initialize $\widehat \cB$ using \eqref{eqn:init-B}.

\noindent 2. Initialize $\widehat \Theta$ using \eqref{eqn:init-Theta}.

\noindent 3. Update $\widehat \cB$ as:
%
\begin{align}\label{eqn:update-B}
\widehat \cB^{(t+1)} &= \argmin_{\substack{\bfB^k \in \BM(p,q)\\k \in \cI_K}} \left\{ \frac{1}{n} \sum_{j=1}^q \sum_{k=1}^K \| \bfY^k_j - (\bfY_{-j}^k - \bfX^k \bfB_{-j}^k) \widehat \bftheta_j^{k (t)} - \bfX^k \bfB_j^{k } \|^2
+ \lambda_n \sum_{h \in \cH} \| \bfB^{[h]} \| \right\}
\end{align}

\noindent 4. Obtain $\widehat \bfE^{k (t+1)} := \bfY^k - \bfX^k \bfB_j^{k (t)}, k \in \cI_K$. Update $\widehat \Theta$ as:
%
\begin{align}\label{eqn:update-Theta}
\widehat \Theta_j^{(t+1)} = \argmin_{\Theta_j \in \BM(q-1, K)}
\left\{ \frac{1}{n} \sum_{k=1}^K
\| \widehat \bfE_j^{k (t+1)} - \widehat \bfE_{-j}^{k (t+1)} \bftheta_j^k \|^2
+ \gamma_n \sum_{j \neq j'} \sum_{g \in \cG_y^{jj'}} \| \bftheta_{jj'}^{[g]} \| \right\}
\end{align}

\noindent 5. Continue till convergence.

\noindent 6. Calculate $\widehat \Omega_y^k, k \in \cI_K$ using \eqref{eqn:omega-y-calc}.
\end{Algorithm}

\subsubsection{Tuning parameter selection}
The nodewise regression step in the JSEM model \eqref{eqn:jsem-model} uses Bayesian Information Criterion (BIC) for tuning parameter selection. The step for updating $\Theta$, i.e. \eqref{eqn:update-Theta}, in our JMMLE algorithm is analogous to this procedure, hence we use BIC to select the penalty parameter $\gamma_n$. In our setting the BIC for a given $\gamma_n$ and fixed $\cB$ is given by:
%
\begin{align*}
\text{BIC} (\gamma_n; \cB) &=
\Tr \left( \bfS_y^k \widehat \Omega_{y,\gamma_n}^k \right) - \log \det \left( \widehat \Omega_{y,\gamma_n}^k \right) +
\frac{\log n}{n} \sum_{k=1}^K | \widehat E_{y,\gamma_n}^k |
\end{align*}
%
where $\gamma_n$ in subscript indicates the corresponding quantity is calculated taking $\gamma_n$ as the tuning parameter, and $\bfS_y^k := (\bfY^k - \bfX^k \bfB^k)^T (\bfY^k - \bfX^k \bfB^k)/n$. Every time $\widehat \Theta$ is updated in the JMMLE algorithm, we choose the optimal $\gamma_n$ as the one with the smallest BIC over a fixed set of values $\cC_n$. Thus for a fixed $\lambda_n$, our final choice of $\gamma_n$ will be 
$
\gamma_n^* (\lambda) = \argmin_{\gamma_n \in \cC_n} \text{BIC} (\gamma_n; \widehat \cB_{\lambda_n})
$.

We use the High-dimensional BIC (HBIC) to select the other tuning parameter, $\lambda$:
%
\begin{align*}
\text{HBIC} (\lambda_n; \Theta) &=
\frac{1}{n} \sum_{j=1}^q \sum_{k=1}^K \| \bfY^k_j - (\bfY_{-j}^k - \bfX^k \widehat \bfB_{-j,\lambda_n}^k ) \bftheta_j^{k } - \bfX^k \widehat \bfB_{j,\lambda_n}^k \|^2 +\\
& \log (\log n) \frac{\log (pq)}{n} \sum_{k=1}^K
\left( \| \bfB^k \|_0 + | \widehat E_{y, \gamma_n^* (\lambda_n)}^k| \right)
\end{align*}
%
We choose an optimal $\lambda_n$ as the minimizer of HBIC by training multiple JMMLE models using Algorithm \ref{algo:jmmle-algo} over a finite set of values $\lambda_n \in \cD_n$: 
$
\lambda^* = \argmin_{\lambda_n \in \cD_n} \text{HBIC} (\lambda, \widehat \Theta_{\gamma_n^*(\lambda_n)})
$.

%\begin{enumerate}
%\item Run neighborhood selection on $y$-network incorporating effects of $x$-data and an additional blockwise group penalty:
%%
%
%%
%where $\Theta = \{ \Theta_i \}, \cB = \{ \bfB^k \}, \cY = \{ \bfY^k \}, \cX = \{ \bfX^k \}, \cE = \{ \bfE^k \}$.
%
%This estimates $\cB$ { \colrbf (possibly refit and/or within-group threshold) }.
%
%\item Step I part 2 and step II of JSEM (see 15-656 pg 6) follows to estimate $\{ \Omega_y^k \}$.
%\end{enumerate}

%The objective function is bi-convex, so we are going to do the following in step 1-
%
%\begin{itemize}
%\item Start with initial estimates of $\cB$ and $\Theta$, say $\cB^{(0)}, \Theta^{(0)}$.
%\item Iterate:
%%
%\begin{align}
%\Theta^{(t+1)} &= \argmin \left\{ f ( \cY, \cX, \cB^{(t)}, \Theta^{(t)}) + P (\Theta^{(t)}) \right\}\\
%\cB^{(t+1)} &= \argmin \left\{ f ( \cY, \cX, \cB^{(t)}, \Theta^{(t+1)}) + Q (\cB^{(t)}) \right\}
%\end{align}
%\item Continue till convergence.
%\end{itemize}
%%

\subsection{Properties of JMMLE estimators}
\label{sec:jmmle-theory}
We now provide theoretical results ensuring the convergence of our alternating algorithm, as well as the consistency of estimators obtained from the algorithm. We present statements of theorems in the main paper, giving detailed proofs and auxiliary results in the Appendix.

We introduce some notations and conditions that help establish the theorems that follow in this section. Denote the true values of the parameters as $\Omega_{x 0} = \{ \Omega_{x 0}^k \}, \Omega_{y 0} = \{ \Omega_{y 0}^k \}, \Theta_0 = \{ \Theta_{0 j} \}, \cB_0 = \{ \bfB_0^k \}$. Sparsity levels of individual true parameters are indicated by $s_j := | \supp (\Theta_{0j})|, b_k := | \supp(\bfB^k_0) |$. Also define $S := \sum_{j=1}^q s_j, B:= \sum_{k=1}^K b_k, s:= \max_{j \in \cI_q } s_j$, and $\cX := \{ \bfX^k \}_{k=1}^K, \cE := \{ \bfE^k \}_{k=1}^K$.

\vspace{1em}
\noindent{\bf Condition 1} (Bounded eigenvalues). A positive definite matrix $\Sigma \in \BM(b,b)$ is said to have bounded eigenvalues with constants $(c_0, d_0)$ if
%
\[
0 < 1/c_0 \leq \Lambda_{\min} (\Sigma) \leq \Lambda_{\max} (\Sigma) \leq 1/d_0 < \infty
\]

\noindent{\bf Condition 2} (Diagonal dominance). A matrix $\bfM  \in \BM(b,b)$ is said to be strictly diagonally dominant if for all $a \in \cI_b$,
%
$$
| (\bfM)_{aa} | > \sum_{a' \neq a} |(\bfM)_{aa'} |
$$
%
Also denote $\Delta_0 (\bfM) = \min_a \{ | (\bfM)_{aa} | - \sum_{a' \neq a} |(\bfM)_{aa'} | \}$.

\paragraph{}
Our first result establishes the convergence of Algorithm~\ref{algo:jmmle-algo} for any fixed realization of $\cX$ and $\cE$.

\begin{Theorem}
\label{thm:algo-convergence}
Suppose for any fixed $(\cX, \cE)$, estimates in each iterate of Algorithm~\ref{algo:jmmle-algo} are uniformly bounded by some quantity dependent on only $p, q$ and $n$:
%
\begin{align}
\left\| (\widehat \cB^{(t)}, \widehat \Theta_y^{(t)}) - ( \cB_0, \Theta_{y 0}) \right\|_F
\leq R(p,q,n);
\quad t \geq 1
\end{align}
%
Then any limit point $(\cB^\infty, \Theta_y^\infty)$ of the algorithm is a stationary point of the objective function, i.e. a point where partial derivatives along all coordinates are non-negative.
\end{Theorem}

The next steps are to show that for random realizations of $\cX$ and $\cE$,
%

\vspace{1em}
\noindent {\bf (a)} successive iterates lie in this non-expanding ball around the true parameters, and

\noindent {\bf (b)} the procedures in \eqref{eqn:init-B} and \eqref{eqn:init-Theta} ensure starting values that lie inside the same ball,
%

\vspace{1em}
\noindent both with probability approaching 1 as $(p,q,n) \rightarrow \infty$.

To do so we break down the main problem into two subproblems. Take as $\bfbeta = (\ve (\bfB^1)^T, \ldots, \ve(\bfB^K)^T)^T$: any subscript or superscript on $\bfB$ being passed on to $\bfbeta$. Denote by $\widehat \Theta$ and $\widehat \bfbeta$ the generic estimators given by
%
\begin{align}
\widehat \Theta_j &= \argmin_{\Theta_j \in \BM(q-1, K)} \left\{ \frac{1}{n} \sum_{k=1}^K \| \widehat \bfE^k_j - \widehat \bfE^k_{-j} \bftheta_j^k \|^2 + \gamma_n \sum_{j \neq j'} \sum_{g \in \cG_y^{jj'}} \| \bftheta_{jj'}^{[g]} \| \right\};
\quad j \in \cI_q, \label{eqn:EstEqn2}\\
\widehat \bfbeta &= \argmin_{\bfbeta \in \BR^{pqK}} \left\{-2 \bfbeta^T \widehat \bfgamma + \bfbeta^T \widehat \bfGamma \bfbeta + \lambda_n \sum_{h \in \cH} \| \bfbeta^{[h]}  \| \right\}, \label{eqn:EstEqn1}
\end{align}
%
where
%
$$
\widehat \bfGamma = \begin{bmatrix}
(\widehat \bfT^1)^2 \otimes \frac{(\bfX^1)^T \bfX^1}{n} & &\\
& \ddots &\\
& & (\widehat \bfT^K)^2 \otimes \frac{(\bfX^K)^T \bfX^K}{n}
\end{bmatrix}; \quad
\widehat \bfgamma = \begin{bmatrix}
(\widehat \bfT^1)^2 \otimes \frac{(\bfX^1)^T}{n}\\
\vdots\\
(\widehat \bfT^K)^2 \otimes \frac{(\bfX^K)^T}{n}
\end{bmatrix}
\begin{bmatrix}
\ve (\bfY^1)\\
\vdots\\
\ve (\bfY^K)
\end{bmatrix},
$$
with 
%
\begin{align}\label{eqn:define-T}
\hat T_{jj'}^k = \begin{cases}
1 \text{ if } j = j'\\
- \hat \theta_{jj'}^k \text{ if } j \neq j'
\end{cases}.
\end{align}
%
Using matrix algebra it is easy to see that solving for $\cB$ in \eqref{eqn:jmmle-objfun} given a fixed $\widehat \Theta$ is equivalent to solving \eqref{eqn:EstEqn1}.

We assume the following conditions on the true parameter versions $(\bfT_0^k)^2$, defined from $\Theta_0$ similarly as \eqref{eqn:define-T}:
%

\vspace{1em}
\noindent{\bf (E1)} The matrices $\Omega_{y0}^k, k \in \cI_K$ are diagonally dominant,

\noindent{\bf (E2)} The matrices $\Sigma_{y0}^k, k \in \cI_K$ have bounded eigenvalues with constants $(c_y, d_y)$ that are common across $k$.
\vspace{1em}
%

\noindent Now we are in a position to establish the estimation consistency for \eqref{eqn:EstEqn2}, as well as the consistency of the final estimates $\widehat \Omega_y^k$ using their support sets.

\begin{Theorem}\label{thm:thm-Theta}
Consider random $(\cX, \cE)$, any deterministic $\widehat \cB$ that satisfy the following bound
%
$$
\| \widehat \bfB^k - \bfB_0^k \|_1 \leq C_\beta \sqrt{ \frac{ \log(pq)}{n}},
$$
%
where $C_\beta = O(1)$ depends only on $\cB_0$. Then, for sample size $n \succsim \log (pq)$ the following hold:

\noindent (I) Denote $|g_{\max}| = \max_{g \in \cG_y} |g|$. Then for the choice of tuning parameter
%
$$
\gamma_n \geq 4 \sqrt{| g_{\max}|} \BQ_0 \sqrt {\frac{ \log (p q)}{n}},
$$
%
where $\BQ_0 = O(1)$ depends on the model parameters only, we have
%
\begin{align}
\| \widehat \Theta_j - \Theta_{0,j} \|_F & \leq 12 \sqrt{s_j} \gamma_n / \psi, \label{eqn:theta-norm-bound-1}\\
\sum_{j \neq j', g \in \cG_y^{jj'}} \| \hat \bftheta_{jj'}^{[g]} - \bftheta_{0,jj'}^{[g]} \| & \leq 48 s_j \gamma_n / \psi. \label{eqn:theta-norm-bound-2}
\end{align}
%

\noindent (II) For the choice of tuning parameter $\gamma_n = 4 \sqrt{| g_{\max}|} \BQ_0 \sqrt{\log (p q)/n}$,
%
\begin{align}\label{eqn:OmegaBounds0}
\frac{1}{K} \sum_{k=1}^K \| \widehat \Omega_y^k - \Omega_{y0}^k \|_F \leq
O \left( \BQ_0 \sqrt{\frac{| g_{\max}| S}{K}} 
\sqrt {\frac{ \log (p q)}{n}} \right),
\end{align}
%
both with probability $\geq 1 - 1/p^{\tau_1-2} - 12 c_1 \exp [-(c_2^2-1) \log(pq)] - 2 \exp (- c_3 n) - 6c_4 \exp [-(c_5^2-1) \log(pq)]$, for some constants $c_1, c_3, c_4 > 0, c_2, c_5 > 1, \tau_1 > 2$.
\end{Theorem}

%We now put both the pieces together, and prove that our alternating algorithm results in a solution sequence $\{ \widehat \cB^{(r)}, \widehat \Theta^{(r)} \}, r = 1, 2, \ldots$ that lies uniformly within a non-expanding ball around the true parameter values.

To prove an equivalent result for the solution of \eqref{eqn:EstEqn1}, we need the following conditions.

\vspace{1em}
\noindent{\bf (E3)} The matrices $(\bfT^k)^2, k \in \cI_K$ are diagonally dominant,

\noindent{\bf (E4)} The matrices $\Sigma_{x0}^k, k \in \cI_K$ have bounded eigenvalues with common constants $(c_x, d_x)$.
\vspace{1em}
%

\noindent Given these, we establish the required consistency results.

\begin{Theorem}\label{thm:thm-B}
Assume random $(\cX, \cE)$, and fixed $\widehat \Theta$ so that for $j \in \cI_q$,
%
\[
\| \widehat \Theta_j - \Theta_{0,j} \|_F \leq C_\Theta \sqrt{\frac{\log q}{n}}
\]
%
for some $C_\Theta = O(1)$ dependent on $\Theta_0$ only. Then, given the choice of tuning parameter
%
$$
\lambda_n \geq 4 \sqrt{| h_{\max} |} \BR_0 \sqrt{ \frac{ \log(pq)}{n}},
$$
%
where $\BR_0 = O(1)$ depends on the population parameters only, the following hold
%
\begin{align}
\| \widehat \bfbeta - \bfbeta_0 \|_1 & \leq 48 \sqrt{ | h_{\max} |} B \lambda_n / \psi_* \label{eqn:BetaThmEqn1}\\
\| \widehat \bfbeta - \bfbeta_0 \| & \leq 12 \sqrt B \lambda_n / \psi_* \label{eqn:BetaThmEqn2}\\
\sum_{h \in \cH} \| \bfbeta^{[h]} - \bfbeta_0^{[h]} \| & \leq 48 B \lambda_n / \psi_* \label{eqn:BetaThmEqn3}\\
(\widehat \bfbeta - \bfbeta_0 )^T \widehat \bfGamma (\widehat \bfbeta - \bfbeta_0 ) & \leq
72 B \lambda_n^2 / \psi_* \label{eqn:BetaThmEqn4}
\end{align}
%
with probability $\geq 1 - 12 c_1 \exp[-(c_2^2-1) \log(pq)] - 2 \exp( -c_3 n)$, where $|h_{\max}| = \max_{h \in \cH} |h|$ and
%
$$
\psi_*= \frac{1}{2} \min_k \left[ \Lambda_{\min} (\Sigma_{x 0}^k) \left( \Delta_0 ( (\bfT_0^k)^2)
- d_k C_\Theta \sqrt{ \frac{\log (pq)}{n}} \right) \right],
$$
%
with $d_k$ being the maximum degree $(\bfT_0^k)^2$.
\end{Theorem}

Following the choice of tuning parameters in Theorems \ref{thm:thm-Theta} and \ref{thm:thm-B}, $S = o(n/\log (pq))$ and $B = o(n/\log (pq))$ are sufficient conditions on the sparsity of corresponding parameters for the JMMLE estimators to be consistent.

Finally we ensure that the starting values are good enough.

\begin{Theorem}\label{thm:starting-values}
Consider the starting values as derived in \eqref{eqn:init-B} and \eqref{eqn:init-Theta}. For sample size $n \succsim \log(pq)$, and the choice of tuning parameter
%
\[
\lambda_n \geq 4 c_2 \max_{k \in \cI_K} \left\{ [\Lambda_{\max} (\Sigma_{x 0}^k) \Lambda_{\max} (\Sigma_{y 0}^k)]^{1/2} \right\}
\sqrt{ \frac{\log (pq)}{n}},
\]
%
we have $\| \widehat \bfbeta^{(0)} - \bfbeta_0 \|_1 \leq 64 B \lambda_n /\psi^*$ with probability $\geq 1 - 6c_1 \exp( -(c_2^2-1) \log(pq)) - 2 \exp(c_3 n)$. Also, for $\gamma_n \geq 4\sqrt{| g_{\max}|} \BQ_0 \sqrt{ \log (pq)/n}$ we have
%
\begin{align*}
\| \widehat \Theta_j^{(0)} - \Theta_{0,j} \|_F & \leq 12 \sqrt{s_j} \gamma_n / \psi,\\
\sum_{j \neq j', g \in \cG_y^{jj'}} \| \hat \bftheta_{jj'}^{[g](0)} - \bftheta_{0,jj'}^{[g]} \| & \leq 48 s_j \gamma_n / \psi,
\end{align*}
%
with probability $\geq 1 - 1/p^{\tau_1-2} - 12 c_1 \exp [-(c_2^2-1) \log(pq)] - 2 \exp (- c_3 n) - 6c_4 \exp [-(c_5^2-1) \log(pq)]$.
\end{Theorem}
%

Putting everything together, estimation consistency for the limit points of Algorithm~\ref{algo:jmmle-algo} given our choice of starting values is immediate.

\begin{Corollary}\label{corollary:jmmle-final}
Assume the conditions (E1)-(E4), and starting values $\{ \cB^{(0)}, \Theta^{(0)} \}$ obtained using \eqref{eqn:init-B} and \eqref{eqn:init-Theta}, respectively. Then, for random realizations of $\cX, \cE$,
%

\vspace{1em}
\noindent (I) For the choice of $\lambda_n$
%
$$
\lambda_n \geq 4 \max \left[ c_2 \max_{k \in \cI_K} \left\{ [\Lambda_{\max} (\Sigma_{x 0}^k) \Lambda_{\max} (\Sigma_{y 0}^k)]^{1/2} \right\}, \sqrt{| h_{\max}|} \BR_0 \right] \sqrt{\frac{\log(pq)}{n}},
$$
%
we have
%
$$
\| \widehat \bfbeta - \bfbeta_0 \|_1 \leq \max \left\{ 48 \sqrt{ | h_{\max} |}, 64 \right\} \frac{B \lambda_n}{\psi_*}
$$
%
with probability $\geq 1 - 18 c_1 \exp[-(c_2^2-1) \log(pq)] - 4 \exp( -c_3 n)$.

\vspace{1em}
\noindent (II) For the choice of $\gamma_n$
%
$$
\gamma_n \geq 4 \sqrt{ | g_{\max} |} \BQ_0 \sqrt{\frac{\log(pq)}{n}},
$$
%
\eqref{eqn:theta-norm-bound-1} and \eqref{eqn:theta-norm-bound-2} hold, while for $\gamma_n = 4 \sqrt{ | g_{\max} |} \BQ_0 \sqrt{ \log (pq)/n}$, \eqref{eqn:OmegaBounds0} holds, both with probability $\geq 1 - 2/p^{\tau_1-2} - 24 c_1 \exp [-(c_2^2-1) \log(pq)] - 4 \exp (- c_3 n) - 12 c_4 \exp [-(c_5^2-1) \log(pq)]$.
\end{Corollary}

\paragraph{Remark.}
To save computation time for high data dimensions, an initial screening step, e.g. the debiased lasso procedure of \cite{JavanmardMontanari14}, can be used to first restrict the support set of $\bfB_j^k$ before obtaining the initial estimates using \eqref{eqn:init-B}. The consistency properties of resulting initial and final estimates follow along the lines of the special case $K=1$ discussed in \cite{LinEtal16}, in conjunction with Theorem~\ref{thm:starting-values} and Corollary~\ref{corollary:jmmle-final}, respectively. We leave the details to the reader.
\section{Hypothesis testing in multilayer models}
\label{sec:sec3}
In this section, we lay out a framework for hypothesis testing in our proposed joint multilayer structure. Present literature in high-dimensional hypothesis testing either focuses on testing for simularities in the within-layer connections of single-layer networks \citep{CaiLiu16,Liu17}, or coefficients of single response penalized regression \citep{vanDeGeerEtal14,ZhangZhang14,MitraZhang16}. However, to our knowledge no method is available in the literature to perform testing for {\it between-layer} connections in a two-layer (or multilayer) setup.

There are two main challenges in doing the above: firstly the need to mitigate the bias of estimators that are obtained from lasso and group lasso-based procedures and assumptions on the design matrix required for the same, and secondly the dependency among response nodes translating into the need for controlling false discovery rate  while simultaneously testing for such hypotheses. In Section~\ref{sec:testing-subsec-1} we propose a debiased estimator for rows of the coefficient matrix estimates $\bfB^k$ that makes use of already computed (using JSEM) nodewise regression coefficients in the upper layer, and establish asymptotic properties of scaled version of them. Section~\ref{sec:testing-subsec-2} is devoted to pairwise testing, where we asssume $K=2$, and propose asymptotic global tests for detecting differential effects of a variable in the upper layer, as well as pairwise simultaneous tests for detecting elementwise difference in the coefficient matrices across $k$.

%Suppose there are two disease subtypes: $k = 1,2$, and we are interested in testing whether the downstream effect of a predictor in X-data is same across both subtypes, i.e. if $\bfb_{0i}^1 = \bfb_{0i}^2$ for some fixed $i \in \cI_p$. For this we consider the modified optimization problem:
%%
%\begin{align}
%& \min_{\cB, \Theta} \frac{1}{n} \left\{ \sum_{j=1}^q \sum_{k=1}^2 \| \bfY_j^k - \bfY_{-j}^k \bftheta_j^k - \bfX^k \bfB_{j}^k \|^2 + \sum_{j \neq j'} \lambda_{jj'} \| \bftheta_{jj'}^* \| + \sum_{i=1}^p \eta_i \| \bfB_{i*}^* \| \right\} \notag\\
%&= \min \left\{ f ( \cY, \cX, \cB, \Theta) + P (\Theta) + Q (\cB) \right\} 
%\end{align}
%%
%with $n_1 = n_2 = n$ for simplicity; and $\bfB^k = (\bfb_1^k, \ldots, \bfb_q^k), (\bfB_{i*}^*) \in \BR^{ q \times K}$.

\subsection{Debiased estimators and asymptotic normality}
\label{sec:testing-subsec-1}
\cite{ZhangZhang14} proposed a debiasing procedure for lasso estimates and subsequently calculate confidence intervals for individual coefficients $\beta_j$ in high-dimensional linear regression: $\bfy = \bfX \bfbeta + \bfepsilon, \bfy \in \BR^n, \bfX \in \BM(n,p)$ and $\epsilon_r \sim N(0,\sigma^2), r \in \cI_n$ for some $\sigma>0$. Given an initial lasso estimate $\widehat \bfbeta^{\text{(init)}} \in \BR^p$ their debiased estimator was defined as:
%
$$
\hat \beta_j^{(\text{deb})} = \hat \beta_j^{(\text{init})} + \frac{\bfz_j^T ( \bfy - \bfX \bfbeta^{(\text{init})})}{\bfz^T \bfx_j}
$$
%
where $\bfz_j$ is the vector of residuals from $\ell_1$-penalized regression of $\bfx_j$ on $\bfX_{-j}$. With centering around the true parameter value, say $\beta_j^0$, and proper scaling this has an asymptotic normal distribution:
%
$$
\frac{\hat \beta_j^{(\text{deb})} - \beta_j^0}{\| \bfz_j \|/| \bfz_j^T \bfx_j |} \sim N(0, \sigma^2)
$$
%
Essentially, they obtain the debiasing factor for the $j^{\Th}$ coefficient by taking residuals from the regularized regression and scale them using the projection of $\bfx_j$ onto a space approximately orthogonal to it. \cite{MitraZhang16} later generalized this idea to group lasso estimates. Further, \cite{vanDeGeerEtal14} and \cite{JavanmardMontanari14} performed debiasing on the entire coefficient vectors.

We start off by defining debiased estimates for individual rows of the coefficient matrices $\bfB^k$ in our two-layer model:
%
\begin{align}\label{eqn:DebiasedBeta}
\widehat \bfc_i^k = \widehat \bfb_i^k + \frac{1}{n t_i^k} \left( \bfX_i^k - \bfX_{-i}^k \widehat \bfzeta_i^k \right)^T (\bfY^k - \bfX^k \widehat \bfB^k )
; \quad i \in \cI_p, k \in \cI_K
\end{align}
%
where $\widehat \bfb_i^k$ denotes the $i\Th$ row of $\widehat \bfB^k$, and $t_i^k = ( \bfX_i^k - \bfX_{-i}^k \widehat \bfzeta_i^k )^T \bfX_{i}^k/n$, and $\widehat \bfzeta_i^k, \widehat \bfB^k$ are {\it generic estimators} of the neighborhood coefficient matrices in the upper layer and within-layer coefficient matrices, respectively. By structure this is similar to the proposal of \cite{ZhangZhang14}. However, as we see shortly, minimal conditions need to be imposed on the parameter estimates used in \eqref{eqn:DebiasedBeta} for the asymptotic results based on a scaled version of the debiased estimator to go thorugh, and they continue to hold for arbitrary sparsity patterns over $k$ in all of the parameters.

Present methods of debiasing coefficients from regularized regression require specific assumptions on the regularization structure of the main regression, as well as on how to calculate the debiasing factor. While \cite{ZhangZhang14}, \cite{JavanmardMontanari14} and \cite{vanDeGeerEtal14} work on coefficients from lasso regressions, \cite{MitraZhang16} debias the coefficients of pre-specified groups in the coefficient vector from a group lasso. Current proposals for obtaining the debiasing factor available in the literature include nodewise lasso \citep{ZhangZhang14} and a variance minimization scheme with $\ell_\infty$-constraints \citep{JavanmardMontanari14}. In comparison, we only assume the following generic constraints on the parameter estimates used in our procedure.

\vspace{1em}
\noindent{\bf (T1)} For the upper layer neighborhood coefficients, the following holds for all $k \in \cI_K$:
%
$$
\| \widehat \bfzeta^k - \bfzeta_0^k \|_1 \leq D_\zeta  = O \left( \sqrt { \frac{\log p}{n}} \right)
$$
%
where $D_\zeta$ depends only on the true values, i.e. $\{ \zeta^k_0 \}$.

\noindent{\bf (T2)} The lower layer precision matrix estimators satisfy for all $k \in \cI_K$
%
$$
\| \widehat \Omega_y^k - \Omega_{y0}^k \|_\infty \leq D_\Omega
= O \left( \sqrt { \frac{\log (pq)}{n}} \right)$$
%
where $D_\Omega$ depends only on $\Omega_{y 0}$.

\noindent{\bf (T3)} For the regression coefficient matrices, the following holds for all $k \in \cI_K$:
%
$$
\| \widehat \bfB^k - \bfB^k_0 \|_1 \leq D_\beta = O \left( \sqrt { \frac{\log (p q)}{n}} \right)
$$
%
where $D_\beta$ depends on $\cB_0$ only.
\vspace{1em}

Given these conditions, the following theorem provides the asymptotic joint distribution of a scaled version of the debiased coefficients. A similar result for fixed design in the context of single-response linear regression can be found in the preprint by \cite{StuckyVandeGeer17}. However they use nuclear norm as the loss function while obtaining the debiasing factors and use the resulting Karush-Kuhn-Tucker (KKT) conditions to derive their results, whereas we leverage bounds on generic parameter estimates combined with the sub-gaussianity of our design matrices.

\begin{Theorem}\label{Thm:ThmTesting}
Define $\widehat s_i^k = \sqrt{\| \bfX_i^k - \bfX_{-i}^k \widehat \bfzeta_i^k \|^2/n}$, and $m_i^k = \sqrt n t_i^k / \widehat s_i^k$. Consider parameter estimates that satisfy conditions (T1)-(T3). Define the following:
%
\begin{align*}
\widehat \Omega_y &= \diag(\widehat \Omega_y^1, \ldots, \widehat \Omega_y^K)\\
\bfM_i &= \diag(m_i^1, \ldots, m_i^K)\\
\widehat \bfC_i &= \ve(\widehat \bfc_i^1, \ldots, \widehat \bfc_i^K)^T\\
\bfD_i &= \ve(\bfb_{0i}^1, \ldots, \bfb_{0i}^K)^T\\
\end{align*}
%
Also assume the conditions (E1), (E2) and (E4). Then for sample size satisfying $\log p = o(n^{1/2}), \log q = o(n^{1/2})$ we have
%
\begin{align}\label{eqn:ThmTestingEqn}
\widehat \Omega_y^{1/2} \bfM_i (\widehat \bfC_i - \bfD_i) \sim
\cN_{Kq} ({\bf 0}, \bfI) + \bfR_n
\end{align}
%
where $\| \bfR_n \|_\infty = o_P(1)$.
\end{Theorem}
%

\subsection{Test formulation}
\label{sec:testing-subsec-2}
We now simply plug in estimators from the JMMLE algorithm in Theorem~\ref{Thm:ThmTesting}. Doing so is fairly straightforward. Condition (T1) is ensured by the JSEM penalized neighborhood estimators in \eqref{eqn:jsem-model} (immediate from Proposition A.1 in \cite{MaMichailidis15}). On the other hand, bounds on total sparsity of the true coefficient matrices: $B = o(\sqrt{n} / \log(pq))$, and lower layer precision matrices: $S = o(\sqrt n/ \log (pq)$, in conjunction with Corollary~\ref{corollary:jmmle-final}, ensure conditions (T2) and (T3), respectively- all with probability approaching 1 as $(n,p,q) \rightarrow \infty$.

An asymptotic joint distribution of debiased versions of the JMMLE regression estimates is now immediate.
%
\begin{Corollary}\label{corollary:CorTesting}
Consider the estimates $\widehat \cB$ and $\widehat \Omega_y $ obtained from Algorithm~\ref{algo:jmmle-algo}, and upper layer neighborhood coefficients from solving the nodewise regression in \eqref{eqn:jsem-model}. Suppose that $\log (pq) /\sqrt n \rightarrow 0$, and the sparsity conditions $B = o(\sqrt{n} / \log(pq)), S = o(\sqrt{n} / \log(pq))$ are satisfied. Then, with the same notations in Theorem~\ref{Thm:ThmTesting} we have
%
\begin{align}\label{eqn:CorTestingEqn}
\widehat \Omega_y^{1/2} \bfM_i (\widehat \bfC_i - \bfD_i) \sim
\cN_{Kq} ({\bf 0}, \bfI) + \bfR_{1n}
\end{align}
%
where $\| \bfR_{1n} \|_\infty = o_P(1)$.
\end{Corollary}

We are now ready to formulate asymptotic global and simultaneous testing procedures based on Corollary~\ref{corollary:CorTesting}. In this paper we restrict our attention to testing for pairwise differences only. Specifically, we set $K=2$, and are interested in testing whether there are overall and elementwise differences between individual rows of the true coefficient matrices, i.e. $\bfb_{0i}^1$ and $\bfb_{0i}^2$.

When $\bfb_{0 i}^1 = \bfb_{0 i}^2$, it is immediate from Corollary~\ref{corollary:CorTesting} that a scaled version of the vector of estimated differences $\widehat \bfc_i^1 - \widehat \bfc_i^2$ follows a $q$-variate multinormal distribution. Consequently we formulate a global test for detecting differential overall downstream effect of the $i^{\Th}$ covariate in the upper layer.

\begin{Algorithm}\label{algo:AlgoGlobalTest}
(Global test for $H_0^i: \bfb_{0 i}^1 = \bfb_{0 i}^2$ at level $\alpha, 0< \alpha< 1$)

\noindent 1. Obtain the debiased estimators $\widehat \bfc_i^1, \widehat \bfc_i^2$ using \eqref{eqn:DebiasedBeta}.

\noindent 2. Calculate the test statistic
%
$$
D_i = (\widehat \bfc_i^1 - \widehat \bfc_i^2)^T
\left( \frac{ \widehat \Sigma_y^1}{(m_i^1)^2} +
\frac{\widehat \Sigma_y^2}{(m_i^2)^2} \right)^{-1} (\widehat \bfc_i^1 - \widehat \bfc_i^2)
$$
%
where $\widehat \Sigma_y^k = (\widehat \Omega_y^k)^{-1}, k = 1,2$.

\noindent 3. Reject $H_0^i$ if $D_i \geq \chi^2_{q, 1-\alpha}$.
\end{Algorithm}

Besides controlling the type-I error at a specified level, the above testing procedure maintains rate optimal power.

\begin{Theorem}\label{thm:PowerThm}
Consider the global test given in Algorithm~\ref{algo:AlgoGlobalTest}, performed using parameter estimates satisfying conditions (T1)-(T3). Say $\bfdelta := \bfb_{0 i}^1 - \bfb_{0 i}^2$. Assume that either of the following sufficient conditions are satisfied.

\begin{itemize}
\item[(I)] The following bound holds: $D_\Omega \leq \Delta_0 (\Omega_{y0}^k), k \in \cI_K$;

\item[(II)] For every $j \in \cI_q, k \in \cI_K$, we have
$
\sum_{j' = 1}^q | \sigma_{y0,jj'}^k |^q \leq c_0 (p)
$ for some $q \in [0,1)$ and positive-valued function $c_0(\cdot)$.
\end{itemize}

Then the power of the global test is given by
%
$$
K_q \left( \chi^2_{q,1-\alpha} + n \bfdelta^T 
\left( \frac{ \Sigma_{y 0}^1}{\sigma_{x0, i,-i}^1} + \frac{\Sigma_{y 0}^2}{\sigma_{x0, i,-i}^2} \right)^{-1} \bfdelta \right) + o(1)
$$
%
where $K_q$ is the cumulative distribution function of the $\chi^2_q$ distribution. Consequently, for $\| \bfdelta \| \geq O(n^{-1/2})$, $P( H_0^i \text{ is rejected }) \rightarrow 1$ as $(n,p,q) \rightarrow \infty$.
\end{Theorem}

The conditions (I) or (II) above are needed to derive upper bounds for $\| \widehat \Sigma_y^k - \Sigma_{y0}^k \|_\infty$ using those for $\| \widehat \Omega_y^k - \Omega_{y0}^k \|_\infty$. While (I) imposes a potentially more stringent bound on the estimation error of $\Omega_y$, (II) restricts the power calculations to a uniformity class of covariance matrices \citep{BickelLevina08,CaiLiuLuo12}.
%

\subsection{Control of False Discovery Rate}
Given the null hypothesis is rejected, we consider the multiple testing problem of simultaneously testing for all entrywise differences, i.e. testing
%
$$
H_0^{ij}: b_{0 ij}^1 = b_{0ij}^2 \quad \text{vs.} \quad H_1^{ij}: b_{0 ij}^1 \neq b_{0 ij}^2 
$$
%
for all $j \in \cI_q$. Here we use the test statistic
%
\begin{align}\label{eqn:PairwiseStatistic}
d_{ij} &= \frac{\widehat c_{ij}^1 - \widehat c_{ij}^2}{\sqrt{\hat \sigma_{jj}^1/ (m_i^1)^2 + \hat \sigma_{jj}^2/ (m_i^2)^2}}
\end{align}
%
with $\hat \sigma_{jj}^k$ being the $j^{\Th}$ diagonal element of $\widehat \Sigma_y^k, k = 1,2$.

For the purpose of simultaneous testing, we consider tests with a common rejection threshold $\tau$, i.e. for $j \in \cI_q$, $H_0^{ij}$ is rejected if $| d_{ij} | > \tau$. We denote $\cH_0^i = \{ j: b_{0,ij}^1 = b_{0,ij}^2 \}$ and define the False Discovery Proportion (FDP) and False Discovery Rate (FDR) for these tests as follows:
%
$$
FDP (\tau) = \frac{\sum_{j \in \cH_0^i} \BI( |d_{ij}| \geq \tau)}{\max\left\{
\sum_{j \in \cI_q} \BI( |d_{ij}| \geq \tau), 1\right\} }\quad
FDR (\tau) = \BE [ FDP (\tau) ]
$$
%
For a pre-specified level $\alpha$, we choose a threshold that ensures both FDP and FDR $\leq \alpha$ using the Benjamini-Hochberg (BH) procedure. % To do this we define the following:
%%
%\begin{align*}
%P_0 = 2 \Phi (1) - 1; \quad
%\hat P_0 = \frac{1}{q} \sum_{j \in \cH_0^i} \BI (|d_{ij}| \leq 1); \quad
%Q_0 = \sqrt 2 \phi (1);\\
%A = \frac{P_0 - \hat P_0}{Q_0}; \quad A(t) = \left[ 1 + |A| \frac{|t| \phi(t)}{\sqrt 2 (1 - \Phi(t))} \right]^{-1}
%\end{align*}
%%
%where $\Phi(\cdot)$ and $\phi(\cdot)$ are the standard normal distribution and density functions, respectively.   
The procedure for FDR control is now given by Algorithm \ref{algo:AlgoFDR}.

\begin{Algorithm}\label{algo:AlgoFDR}
(Simultaneous tests for $H_0^{ij}: b_{0 ij}^1 = b_{0 ij}^2$ at level $\alpha, 0< \alpha< 1$)

\noindent 1. Calculate the pairwise test statistics $d_{ij}$ using \eqref{algo:AlgoFDR} for $j \in \cI_q$.

\noindent 2. Obtain the threshold
%
$$
\hat \tau = \inf \left\{\tau \in \BR: 1 - \Phi(\tau) \leq \frac{\alpha}{2 q}
\max \left( \sum_{j \in \cI_q} \BI( |d_{ij}| \geq \tau), 1 \right) \right\}
$$
%

\noindent 3. For $j \in \cI_q$, reject $H_0^{ij}$ if $|d_{ij}| \geq \hat \tau$.
\end{Algorithm}

%Denote $\hat \cH_0^i = \{ j\ in \cI_q: |d_{ij}| \geq \hat \tau \}$.
To ensure that this procedure maintains FDR and FDP asymptotically at a pre-specified level $\alpha \in (0,1)$, we need some dependence conditions on true correlation matrices in the lower layer. Following \cite{LiuShao14}, we consider the following two types of dependencies:

\noindent {\bf (D1)} Define $r_{jj'}^k = \sigma_{y0,jj'}^k /\sqrt{\sigma_{y0,jj}^k \sigma_{y0,j'j'}^k}$ for $j,j' \in \cI_q, k = 1,2$. Suppose there exists $0 < r < 1$ such that $\max_{1 \leq j < j' \leq q} | r_{jj'}^k | \leq r$, and for every $j \in \cI_q$,
%
$$
\sum_{j'=1}^q \BI \left\{ |r_{jj'}^k| \geq \frac{1}{(\log q)^{2 + \theta}} \right\} \leq O(q^\rho)
$$
%
for some $\theta > 0$ and $0 < \rho < (1-r)/(1+r)$.

\noindent {\bf (D1*)} Suppose there exists $0 < r < 1$ such that $\max_{1 \leq j < j' \leq q} | r_{jj'}^k | \leq r$, and for every $j \in \cI_q$,
%
$$
\sum_{j'=1}^q \BI \left\{ |r_{jj'}^k| > 0 \right\} \leq O(q^\rho)
$$
%
for some $0 < \rho < (1-r)/(1+r)$.

Originally proposed by \cite{LiuShao14}, the above dependency conditions are meant to control the amount of correlation among the test statistics. Condition (D1) allows each variable to be highly correlated with at most $O(q^\rho)$ other variables and weakly correlated with others, while (D1*) limits the number of variables to have {\it any} correlation with it to $O(q^\rho)$. Note that (D1*) is a stronger condition, and can be seen as the limiting condition of (D1) as $q \rightarrow \infty$.

\begin{Theorem}\label{thm:FDRthm}
Suppose $\mu_j = b_{0,ij}^1 - b_{0,ij}^2, \sigma_j^2 = \sigma_{y0,jj}^1/ \sigma_{x0,i,-i}^1 + \sigma_{y0,jj}^2/ \sigma_{x0,i,-i}^2$. Assume the following holds as $(n,q) \rightarrow \infty$
%
\begin{align}\label{eqn:FDRthmEqn1}
\left| \left\{ j \in \cI_q: |\mu_j / \sigma_j | \geq
4 \sqrt{ \log q/n} \right\} \right| \rightarrow \infty
\end{align}
%
Now Consider the conditions (D1) and (D1*). If (D1) is satisfied, then the following holds when $\log q = O(n^{\xi}), 0 < \xi < 3/23$:
%
\begin{align}\label{eqn:FDReqn}
\frac{FDP( \hat \tau)}{(| \cH_0^i|/q) \alpha} \stackrel{P}{\rightarrow} 1; \quad
\lim_{n, q \rightarrow \infty} \frac{FDR( \hat \tau)}{(| \cH_0^i|/q) \alpha} = 1
\end{align}
%
Further, if (D1*) is satisfied, then \eqref{eqn:FDReqn} holds for $\log q = o(n^{1/3})$.
\end{Theorem}
%
%Theorem~\ref{thm:FDRthm} is essentially a restatement of Theorem 4.1 in \cite{LiuShao14}.
The condition \eqref{eqn:FDRthmEqn1} is essential for FDR control in a diverging parameter space \citep{LiuShao14, Liu17}.

\begin{Remark}
Following \eqref{eqn:FDRthmEqn1}, a sufficient condition on the sparsity of $\cB_0$ for FDR to be asymptotically controlled at some specified level is $B = o(n^\zeta/ \log q)$ if (D1) is satisfied, and $B = o(n^{1/3}/ \log q)$ if (D1*) is satisfied. In comparison, our results for the global testing procedure require $B = o(\sqrt n/ \log (pq))$, and point estimation requires $B = o(n/ \log (pq))$. In finite samples settings, the stricter sparsity requirements translate to higher sample sizes being needed (given the same $(p,q)$) for our testing procedures to have satisfactory performances compared to estimation only (See Sections \ref{sec:eval-jmmle} and \ref{sec:eval-testing}).

In recent work, \cite{JavanmardMontanari18} showed that the $o(\sqrt n/ \log p)$ bound on the sparsity of the true coefficient vector required to construct confidence intervals from debiased lasso coefficient estimates \citep{vanDeGeerEtal14, ZhangZhang14,JavanmardMontanari14} can be weakened to $o(n/ (\log p)^2)$ when the random design precision matrix is known, or is unknown but satisfies certain sparsity assumptions. Similar relaxations may be possible in our case. For example, the machineries in \cite{Liu17}, which performs simultaneous testing in multiple (single layer) GGMs using slightly modified FDR thresholds, can be useful in obtaining \eqref{eqn:FDReqn} for $ \log q = o(n^{1/2})$ under (D1), (D1*) or other suitable dependency assumptions.% However, this requires a significant amount of theoretical analysis, and we leave it for future research.
\end{Remark}

\begin{Remark}
Based on the FDR control procedure in Algorithm~\ref{algo:AlgoFDR}, we can perform {\it within-row thresholding} in the matrices $\widehat \bfB^k$ to tackle group misspecification.
%
\begin{align}
& \hat \tau_i^k := \inf \left\{\tau \in \BR: 1 - \Phi(\tau) \leq \frac{\alpha}{2 q}
\max \left( \sum_{j \in \cI_q} \BI( | \sqrt{\hat \omega_{jj}^k} m_i^k \hat c_{ij}^k | \geq \tau), 1 \right) \right\} \notag\\
& \hat b_{ij}^{k, \text{thr}} =  \hat b_{ij}^k \BI \left(
|\sqrt{\hat \omega_{jj}^k} m_i^k \hat c_{ij}^k | \geq \hat \tau_i^k \right)
\label{eqn:fdr-threshold}
\end{align}
%
Even without group misspecification, this helps identify directed edges between layers that have high nonzero values. Similar post-estimation thresholdings have been proposed in the context of multitask regression \citep{ObozinskiEtal11,MajumdarChatterjeeStat} and neighborhood selection \citep{MaMichailidis15}. However, our procedure is the first one to provide explicit guarantees on the amount of false discoveries while doing so.
\end{Remark}
\section{Numerical performance}
\label{sec:sec4}
In this section, we evaluate the performance of our proposed JMMLE algorithm and the hypothesis testing framework in a two-layer simulation setup (Sections \ref{sec:eval-jmmle} and \ref{sec:eval-testing}), and also introduce some computational techniques that significantly accelerates computation for high data dimensions (Section~\ref{sec:tricks-jmmle}).

\subsection{Simulation 1: estimation}
\label{sec:eval-jmmle}
As a first step towards obtaining a two-layer structure with horizontal (across $k$) complexity and inter-layer directed edges, we generate the precision matrices $\{ \Omega_{x0}^k \}$ and $\{ \Omega_{y0}^k \}$ using a dependency structure across $k$ that was first used in the simulation study of \cite{MaMichailidis15}. We set $K=5$, and set different shared sparsity patterns across $k$ inside the lower $p/2 \times p/2$ block of the upper layer precision matrices, and outside the block. In our notation, this means the following elementwise group structure:
%
$$
\cG_{x,ii'} = \begin{cases}
\{ (1,2),(3,4), 5 \} \text{ if } i \leq p/2 \text{ or } j \leq p/2\\
\{ (1,3,5),(2,4) \} \text{ otherwise }
\end{cases}
$$
%
The schematic in Figure~\ref{fig:sim-structure} illustrates this structure. We set an off-diagonal element inside each of these common blocks (i.e. $A,B,C$ and $\alpha, \beta$ in the figure) to be non-zero with probability $\pi_x \in \{ 5/p, 30/p \}$, then generate the values of all non-zero elements independently from the uniform distribution in the interval $[-1, 0.5] \cup [0.5, 1]$. The precision matrices $\Omega_{x0}^k$ are generated by putting together the corresponding common blocks, their positive definiteness ensured by setting all diagonal elements to be $1 + |\Lambda_{\min} (\Omega_{x0}^k)|$. We get elements in the covariance matrix as
%
$$
\sigma_{x0,ii'}^k = (\omega_{x0,ii'}^k)^{-1} / \sqrt{(\omega_{x0,ii}^k)^{-1} (\omega_{x0,i'i'}^k)^{-1} },
$$
and generate rows of $\bfX^k$ independently from $\cN(0, \Sigma_{x0}^k)$. We obtain $\Sigma_{y0}^k$ and then $\bfE^k$ using the same setup but with the number of variables being $q$ and setting off-diagonal elements non-zero with probability $\pi_y \in \{ 5/q, 30/q \}$. To obtain the matrices $\bfB_0^k$, for a fixed $(i,j), i \in \cI_p, j \in \cI_q$, we set $\bfb_{ij}^k$ non-zero across all $k$ with probability $\pi \in \{ 5/p, 5/q \}$, generate the non-zero groups independently from $\text{Unif} \{ [-1, 0.5] \cup [0.5, 1] \}$, and set $\bfY^k = \bfX^k \bfB_0^k + \bfE^k, k \in \cI_K$. Finally, we generate 50 such independent two-layer datasets for each of the following model settings:

\begin{itemize}
\item Set $\pi_x = \pi = 5/p, \pi_y = 5/q$, and
%
$$
(p,q,n) \in \{ (60,30,100), (30,60,100), (200,200,150), (300,300,150) \};
$$

\item Set $\pi_x = \pi = 30/p, \pi_y = 30/q$, and $(p,q,n) \in  \{ (200,200,100), (200,200,200) \}$.
\end{itemize}
%

\begin{figure}
\centering
\includegraphics{omega-structure}
\caption{Shared sparsity patterns across $k$ for the precision matrices $\{ \Omega_{x0}^k\}$ and $\{ \Omega_{y0}^k\}$}
\label{fig:sim-structure}
\end{figure}

We use the following array of tuning parameters to train Algorithm~\ref{algo:jmmle-algo}:
%
$$
\gamma \in \left\{ 0.3, 0.4, ..., 1 \right\} \sqrt{\frac{\log q}{n}}; \quad
\lambda \in \left\{ 0.4, 0.6, ..., 1.8 \right\} \sqrt{\frac{\log p}{n}}
$$
%
using the one-step version (Section~\ref{sec:tricks-jmmle}) instead of the full algorithm to save computation time. We compare the performance of our joint estimation method with separate estimates of the parameters using the method of \cite{LinEtal16}, using the following performance metrics to evaluate estimates $\tilde \cB = \{ \tilde \bfB^k \}$:

\begin{itemize}
\item True positive Rate-
%
\[
\text{TP}(\tilde \bfB_k) = | \supp(\hat \bfB^k) \cap \supp (\bfB_0^k)|; \quad
\text{TPR} (\tilde \cB) = \frac{\sum_k \text{TP}(\tilde \bfB_k) }{\sum_k | \supp (\bfB_0^k)| }
\]
\item True negative Rate-
%
\[
\text{TN}(\tilde \bfB_k) = | {\supp}^c (\hat \bfB^k) \cap {\supp}^c (\bfB_0^k)|; \quad
\text{TNR} (\tilde \cB) = \frac{\sum_k \text{TN}(\tilde \bfB_k) }{\sum_k | \supp^c (\bfB_0^k)| }
\]
%
\item F1 measure-
%
$$
\text{FP}(\tilde \bfB_k) = | {\supp}^c (\bfB^k_0) | - \text{TN}(\tilde \bfB_k); \quad
\text{FN}(\tilde \bfB_k) = | {\supp} (\bfB^k_0) | - \text{TP}(\tilde \bfB_k) $$
$$ \text{F1} (\tilde \cB) = \frac{ \sum_k \text{TP}(\tilde \bfB_k)}
{2 \sum_k \text{TP}(\tilde \bfB_k) + \text{FP}(\tilde \bfB_k) + \text{FN}(\tilde \bfB_k)}
$$
%
\item Relative error in Frobenius norm-
%
\[
\text{RF} (\tilde \cB) = \sum_{k=1}^K \frac{\| \hat \bfB^k - \bfB_0^k \|_F}{\| \bfB_0^k \|_F}
\]
%
\end{itemize}
%
We use the same metrics to evaluate the precision matrix estimates $\tilde \Omega_y^k$ as well.

Tables \ref{table:simtable11} and \ref{table:simtable12} summarize the results. Joint estimation vastly outperforms separate estimation for $\cB_0$ across all metrics. JMMLE tends to be conservative for the estimation of $\Omega_{y0}^k$, producing sparse estimates and very high true negative proportions, although they are still far more accurate than those obtained from separate estimation, as evident by the low average RF scores.
%

\begin{scriptsize}
\begin{table}
    \begin{tabular}{ccccccc}
    \hline
    $(\pi_x, \pi_y)$ & $(p,q,n)$   & Method   & TP            & TN             & F1 & RF            \\ \hline
    $(5/p, 5/q)$   & (60,30,100)   & JMMLE    & 0.97 (0.045) & 0.99 (0.006)   & 0.97 (0.03)  & 0.24 (0.073) \\
    ~              & ~             & Separate & 0.95 (0.018) & 0.99 (0.002)  & ~   & 0.27 (0.031) \\
    ~              & (30,60,100)   & JMMLE    & 0.97 (0.029) & 0.99 (0.005)  & 0.96 (0.015)   & 0.27 (0.053) \\
    ~              & ~             & Separate & 0.66 (0.038) & 0.99 (0.001) & ~   & 0.59 (0.033) \\
    ~              & (200,200,150) & JMMLE    & 0.98 (0.025) & 1.0 (0)      & 0.99 (0.012)   & 0.16 (0.056) \\
    ~              & ~             & Separate & ~             & ~              & ~   & ~             \\
    ~              & (300,300,150) & JMMLE    & 0.99 (0.015) & 1.0 (0)      & 0.99 (0.008)   & 0.14 (0.033)\\
    ~              & ~             & Separate & ~             & ~              & ~   & ~             \\\hline
    $(30/p, 30/q)$ & (200,200,100) & JMMLE    & 0.97 (0.039) & 1.0 (0)      & 0.98 (0.017)   & 0.21 (0.07) \\
    ~              & ~             & Separate & ~             & ~              & ~   & ~             \\
    ~              & (200,200,200) & JMMLE    & 0.99 (0.01)  & 1.0 (0)      & 0.99 (0.017)   & 0.13 (0.036) \\
    ~              & ~             & Separate & ~             & ~              & ~   & ~             \\ \hline
    \end{tabular}
    \caption{Table of outputs for joint and separate estimation of regression matrices, giving empirical mean and standard deviation (in brackets) of each evaluation metric over 50 replications.}
    \label{table:simtable11}
\end{table}
%
\begin{table}
    \begin{tabular}{ccccccc}
    \hline
    $(\pi_x, \pi_y)$ & $(p,q,n)$   & Method   & TP            & TN             & F1 & RF            \\ \hline
    $(5/p, 5/q)$   & (60,30,100)   & JMMLE    & 0.76 (0.041) & 0.90 (0.014)   & 0.68 (0.048)   & 0.32 (0.018) \\
    ~              & ~             & Separate & 0.89 (0.018) & 0.63 (0.014)  & ~   & 0.77 (0.044) \\
    ~              & (30,60,100)   & JMMLE    & 0.7 (0.035) & 0.94 (0.003)  & 0.58 (0.039)   & 0.3 (0.01) \\
    ~              & ~             & Separate & 0.62 (0.027) & 0.81 (0.007)  & ~   & 0.43 (0.011) \\
    ~              & (200,200,150) & JMMLE    & 0.68 (0.037) & 0.98 (0.001)  & 0.47 (0.031)   & 0.26 (0.005) \\
    ~              & ~             & Separate & ~             & ~              & ~   & ~             \\
    ~              & (300,300,150) & JMMLE    & 0.71 (0.03)  & 0.98 (0)      & 0.41 (0.019)   & 0.25 (0.004) \\
    ~              & ~             & Separate & ~             & ~              & ~   & ~             \\\hline
    $(30/p, 30/q)$ & (200,200,100) & JMMLE    & 0.77 (0.037) & 0.98 (0.001)  & 0.42 (0.034)   & 0.31 (0.008) \\
    ~              & ~             & Separate & ~             & ~              & ~   & ~             \\
    ~              & (200,200,200) & JMMLE    & 0.76 (0.04)  & 0.98 (0.001)  & 0.54 (0.035)   & 0.27 (0.008) \\
    ~              & ~             & Separate & ~             & ~              & ~   & ~             \\ \hline
    \end{tabular}
    \caption{Table of outputs for joint and separate estimation of lower layer precision matrices over 50 replications.}
    \label{table:simtable12}
\end{table}
\end{scriptsize}

\subsubsection{Effect of heterogeneity}
We repeat the above setups to check the performance of JMMLE in presence of within-group misspecification. For this we set individual elements in a non-zero group $\{ b_{ij}^k, k \in \cI_K \}$ to be non-zero with probability 0.2, then pass JMMLE estimates of $\bfB_0^k$ through the FDR controlling thresholds as given in \eqref{eqn:fdr-threshold}. The results are summarized in Tables \ref{table:simtable2} and \ref{table:simtable22}. For each setting, TP and TN rates are almost identical to the fully specified rates in Table~\ref{table:simtable11}, thus the thresholding step is effective. In all cases the empirical FDR for estimating $\cB$ is below 0.2. The estimation quality of $\Theta$, however, suffers more. This is expected, as the estimates $\widehat \Omega_y^k$ are obtained from neighborhood coefficients that are calculated based on the pre-thresholding coefficient estimates.
\begin{scriptsize}
\begin{table}
\centering
    \begin{tabular}{cccccc}
    \hline
    $(\pi_x, \pi_y)$ & $(p,q,n)$   & TP$(\widehat \cB)$            & TN$(\widehat \cB)$             & F1$(\widehat \cB)$ & RF$(\widehat \cB)$    \\ \hline
    $(5/p, 5/q)$   & (60,30,100)   & 0.98 (0.022)  & 0.99 (0.005)   & 0.89 (0.04)   & 0.29 (0.031) \\
    ~              & (30,60,100)   & 0.94 (0.048)  & 0.99 (0.008)   & 0.93 (0.033)  & 0.31 (0.062) \\
    ~              & (200,200,150) & 0.99 (0.004)  & 0.99 (0)       & 0.98 (0.008)  & 0.17 (0.015) \\
    ~              & (300,300,150) & 0.99 (0.002)  & 1 (0)          & 0.99 (0.004)  & 0.15 (0.013) \\
    $(30/p, 30/q)$ & (200,200,100) & 0.99 (0.021)  & 1 (0)          & 0.98 (0.011)  & 0.2 (0.031)  \\
    ~              & (200,200,200) & 0.99 (0.02)   & 1 (0)          & 0.98 (0.011)  & 0.15 (0.037) \\\hline
    \hline
    $(\pi_x, \pi_y)$ & $(p,q,n)$   & TP$(\widehat \Omega)$            & TN$(\widehat \Omega)$             & F1$(\widehat \Omega)$ & RF$(\widehat \Omega)$            \\ \hline
    $(5/p, 5/q)$   & (60,30,100)   & 0.71 (0.054)  & 0.90 (0.012)   & 0.64 (0.044)  & 0.34 (0.018)\\
    ~              & (30,60,100)   & 0.7 (0.043)   & 0.94 (0.004)   & 0.59 (0.03)   & 0.3 (0.01)  \\
    ~              & (200,200,150) & 0.62 (0.027)  & 0.98 (0.001)   & 0.43 (0.021)  & 0.27 (0.006)\\
    ~              & (300,300,150) & 0.69 (0.03)   & 0.98 (0)       & 0.39 (0.02)   & 0.26 (0.05) \\
    $(30/p, 30/q)$ & (200,200,100) & 0.78 (0.054)  & 0.98 (0.001)   & 0.43 (0.037)  & 0.31 0.008) \\
    ~              & (200,200,200) & 0.69 (0.059)  & 0.98 (0.001)   & 0.5 (0.044)   & 0.29 (0.009)\\\hline
    \end{tabular}
    \caption{Table of outputs for joint estimation in presence of group misspecification}
    \label{table:simtable2}
\end{table}
\end{scriptsize}

\begin{scriptsize}
\begin{table}[t]
\centering
    \begin{tabular}{ccc}
    \hline
    $(\pi_x, \pi_y)$ & $(p,q,n)$   & FDR          \\\hline
    $(5/p, 5/q)$   & (60,30,100)   & 0.19 (0.077) \\
    ~              & (30,60,100)   & 0.08 (0.064) \\
    ~              & (200,200,150) & 0.04 (0.016) \\
    ~              & (300,300,150) & 0.02 (0.007) \\\hline
    $(30/p, 30/q)$ & (200,200,100) & 0.03 (0.019) \\
    ~              & (200,200,200) & 0.03 (0.016) \\\hline
    \end{tabular}
    \caption{Table of outputs giving empirical FDR for estimating $\cB$ using JMMLE in presence of group misspecification}
    \label{table:simtable22}
\end{table}
\end{scriptsize}

\subsection{Simulation 2: testing}
\label{sec:eval-testing}
We slightly change our data generating model to evaluate our proposed global testing and FDR control procedure. We set $K=2$, then generate the $\bfB_0^1$ by first randomly assigning each of its element to be non-zero with probability $\pi$, then drawing values of those elements from $\text{Unif}\{ [ -1, -0.5] \cup [0.5,1]\}$ independently. After this we generate a matrix of differences $\bfD$, with $(\bfD)_{ij}, i \in \cI_p, j \in \cI_q$ taking values -1, 1, 0 with probabilities 0.1, 0.1 and 0.8, respectively. Finally we set $\bfB_0^2 = \bfB_0^1 + \bfD$. We set the same sparsity structures for the pairs of precision matrices $\{ \Omega_{x0}^1, \Omega_{x0}^2 \}$ and $\{ \Omega_{y0}^1, \Omega_{y0}^2 \}$. We use 50 replications of the above setup to calculate empirical power of global tests, as well as empirical power and FDR of simultaneous tests, while to get size of global tests we use JMMLE estimators from a separate set of data generated setting all elements of $\bfD$ to 0. The type-I error of global tests is taken as 0.05, while FDR is set at 0.2 while calculating the respective thresholds.

Table~\ref{table:simtable3} reports the empirical mean and standard deviations (in brackets) of all relevant quantities. We report outputs for all combinations of data dimensions and sparsity used in Section~\ref{sec:eval-jmmle}, and also for increased sample sizes in each setting until a satisfactory FDR is reached. As expected, higher sample sizes result in increased power for both global and simultaneous tests, and decreased size and FDR for all but one ($p=30, q=60$) of the settings.

%
\begin{scriptsize}
\begin{table}
    \begin{tabular}{ccccccc}
    \hline
$(\pi_x, \pi_y)$ & $(p,q)$   & $n$ & \multicolumn{2}{c}{Global test} & \multicolumn{2}{c}{Simultaneous tests}\\\cline{4-7}
 & & & Power     & Size			   & Power         & FDR           \\ \hline
    $(5/p, 5/q)$ & (60,30)   & 100 & 0.977 (0.018) & 0.058 (0.035) & 0.937 (0.021) & 0.237 (0.028) \\
    ~            & ~         & 200 & 0.987 (0.016) & 0.046 (0.032) & 0.968 (0.013) & 0.218 (0.032) \\
    ~            & (30,60)   & 100 & 0.985 (0.018) & 0.097 (0.069) & 0.925 (0.022) & 0.24 (0.034)  \\
    ~            & ~         & 200 & 0.990 (0.02)  & 0.119 (0.059) & 0.958 (0.024) & 0.245 (0.041) \\
    ~            & (200,200) & 150 & 0.987 (0.005) & 0.004 (0.004) & 0.841 (0.13)  & 0.213 (0.007) \\
    ~            & (300,300) & 150 & 0.988 (0.002) & 0.002 (0.003) & 0.546 (0.035) & 0.347 (0.017) \\
    ~            & ~         & 300 & 0.998 (0.003) & 0.000 (0.001) & 0.989 (0.003) & 0.117 (0.006) \\ \hline
  $(30/p, 30/q)$ & (200,200) & 100 & 0.994 (0.005) & 0.262 (0.06)  & 0.479 (0.01)  & 0.557 (0.006) \\
    ~            & ~         & 200 & 0.998 (0.004) & 0.020 (0.01)  & 0.962 (0.003) & 0.266 (0.007) \\
    ~            & ~         & 300 & 0.999 (0.002) & 0.011 (0.008) & 0.990 (0.004) & 0.185 (0.009) \\ \hline
    \end{tabular}
    \caption{Table of outputs for hypothesis testing.}
    \label{table:simtable3}
\end{table}
\end{scriptsize}

\subsection{Computation}
\label{sec:tricks-jmmle}
We now discuss some observations and strategies that speed up the JMMLE algorithm and reduces computations time significantly, especially for higher number of features in either layer.

\paragraph{Block update and refit $\bfB^k$ in each iteration.} Similar to the case of $K=1$ \citep{LinEtal16}, we use block coordinate descent {\it within} each $\bfB^k$. This means instead of the full update step \eqref{eqn:update-B} we perform the following steps in each iteration to speed up convergence:
%
$$
\left\{\widehat \bfB^{k (t+1)}_j \right\}_{k=1}^K =
\argmin_{\substack{\bfb_j^k \in \BR^p\\k \in \cI_K}} \left\{ \frac{1}{n} \sum_{j=1}^q \sum_{k=1}^K \| \bfY^k_j + \bfr_j^{k (t)} - \bfX^k \bfB_j^{k } \|^2
+ \lambda \sum_{h \in \cH} \| \bfB_j^{[h]} \| \right\}
$$
%
where $\bfr_1^{k (t)} = \widehat \bfE_{-1}^{k (t)} \widehat \bftheta_1^{k (t)}$, and
%
$$
\bfr_j^{k (t)} = \sum_{j'=1}^{j-1} \hat \bfe_j^{k (t+1)} \hat \theta_{jj'}^{k (t)} +
\sum_{j'=j+1}^{q} \hat \bfe_j^{k (t)} \hat \theta_{jj'}^{k (t)}
$$
%
for $j \geq 2$. Further, when starting from the initializer of the coefficient matrix given in \eqref{eqn:init-B}, the support set of coefficient estimates becomes constant after only a few ($\sim 10$) iterations of our algorithm, after which it refines the values inside the same support until overall convergence. This process speeds up significantly if a refitting step is added {\it in each iteration} after the matrices $\widehat \bfB^k$ are updated:
%
\begin{align*}
\left\{\widetilde \bfB^{k (t+1)}_j \right\}_{k=1}^K &=
\argmin_{\substack{\bfb^k \in \BR^p\\k \in \cI_K}} \left\{ \frac{1}{n} \sum_{j=1}^q \sum_{k=1}^K \| \bfY^k_j + \bfr_j^{k (t)} - \bfX^k \bfB_j^{k } \|^2
+ \lambda \sum_{h \in \cH} \| \bfB^{[h]}_{-j} \| \right\}; \\
\widehat \bfB^{k (t+1)}_j &= \left[ (\bfX_{\cS_{jk}}^k)^T (\bfX_{\cS_{jk}}^k) \right]^- (\bfX_{\cS_{jk}}^k)^T \bfY_j^k
\end{align*}
%
where $\cS_{jk} = \supp(\widetilde \bfB^{k (t+1)}_j)$.

\paragraph{One-step estimator.} Algorithm~\ref{algo:jmmle-algo}, even after the above modifications, is conputation-intensive. The reason behind this is the full tuning and updating of the lower layer neighborhood estimates $\{ \widehat \Theta_j \}$ in each iteration. In practice, the algorithm speeds up significantly without compromising on estimation accuracy if we dispense of the $\Theta_j$ updation step in all but the last iteration. More precisely, we consider the following one-step version of the original algorithm.

\begin{Algorithm}
(The one-step JMMLE Algorithm)
\label{algo:jmmle-algo-1step}

\noindent 1. Initialize $\widehat \cB$ using \eqref{eqn:init-B}.

\noindent 2. Initialize $\widehat \Theta$ using \eqref{eqn:init-Theta}.

\noindent 3. Update $\widehat \cB$ as:
%
\begin{align*}
\widehat \cB^{(t+1)} &= \argmin_{\substack{\bfB^k \in \BM(p,q)\\k \in \cI_K}} \left\{ \frac{1}{n} \sum_{j=1}^q \sum_{k=1}^K \| \bfY^k_j - (\bfY_{-j}^k - \bfX^k \bfB_{-j}^k) \widehat \bftheta_j^{k (0)} - \bfX^k \bfB_j^{k } \|^2
+ \lambda_n \sum_{h \in \cH} \| \bfB^{[h]} \| \right\}
\end{align*}

\noindent 4. Continue till convergence to obtain $\widehat \cB = \{ \widehat \bfB^k \}$.

\noindent 5. Obtain $\widehat \bfE^k := \bfY^k - \bfX^k \widehat \bfB^k, k \in \cI_K$. Update $\widehat \Theta$ as:
%
\begin{align*}
\widehat \Theta_j = \argmin_{\Theta_j \in \BM(q-1, K)}
\left\{ \frac{1}{n} \sum_{k=1}^K
\| \widehat \bfE_j^k - \widehat \bfE_{-j}^k \bftheta_j^k \|^2
+ \gamma \sum_{j \neq j'} \sum_{g \in \cG_y^{jj'}} \| \bftheta_{jj'}^{[g]} \| \right\}
\end{align*}

\noindent 6. Calculate $\widehat \Omega_y^k, k \in \cI_K$ using \eqref{eqn:omega-y-calc}.
\end{Algorithm}

Note that compared to early stopping rules of iterative algorithms \citep{}, we let $\cB$ converge completely, then use these solutions to recover the support set of the precision matrices. The estimation accuracy of $\Omega_y$ depends on the solution $\widehat \cB$ used to solve the sub-problem \eqref{eqn:EstEqn2} (Theorem~\ref{thm:thm-Theta} and Lemmas \ref{thm:ThetaThm} and \ref{prop:ErrorRE}). Letting $\cB$ converge first ensures that the solutions $\widehat \Theta$ and $\widehat \Omega_y$ obtained subsequently are of a better quality compared to a simple early stopping of the JMMLE algorithm.

Table~\ref{table:simtable41} compares performances the full algorithm and the one-step version for the two data settings with smaller feature dimensions. The performances are indistinguishable across metrics, but the one-step algorithm saves computation time in orders of magnitude (Table~\ref{table:simtable42}).

% p3=, q=60: full- 118552, 1-step- 18772 secs for 100 replications
\begin{table}
\centering
    \begin{tabular}{cccccc}
    \hline
    $(p,q,n)$     & Method         & TP$(\widehat \cB)$            & TN$(\widehat \cB)$            & F1$(\widehat \cB)$ & RF$(\widehat \cB)$             \\\hline
    (60,30,100) & Full         & 0.999 (0.002) & 0.992 (0.009) & ~   & 0.195 (0.021)                \\
    ~           & One step     & 0.999 (0.002) &  0.993 (0.01) & ~   & 0.190 (0.019)                 \\
    (30,60,100) & Full          & 0.997 (0.004) & 0.986 (0.007) & ~   & 0.205 (0.014)               \\
    ~           & One step & 0.996 (0.004) & 0.988 (0.006) & ~   & 0.206 (0.014)             \\ \hline
    \hline
    $(p,q,n)$     & Method         & TP$(\widehat \Omega)$            & TN$(\widehat \Omega)$            & F1$(\widehat \Omega)$ & RF$(\widehat \Omega)$            \\\hline
    (60,30,100) & Full         & 0.671 (0.052) & 0.949 (0.01) & ~   & 0.327 (0.015)                \\
    ~           & One step     & 0.663 (0.058) & 0.95 (0.01)   & ~   & 0.328 (0.018)               \\
    (30,60,100) & Full          & 0.58 (0.039) & 0.982 (0.003) & ~   & 0.32 (0.009)             \\
    ~           & One step & 0.577 (0.035) & 0.981 (0.003)  & ~   & 0.321 (0.008)            \\ \hline
    \end{tabular}
    \caption{Comaprison of evaluation metrics for full and one-step versions of the JMMLE algorithm.}
    \label{table:simtable41}
\end{table}

\begin{table}[t]
\centering
  \begin{tabular}{ccc}
    \hline
    $(p,q,n)$     & Method   & Comp. time (min) \\ \hline
    (60,30,100) & Full     & 5.8              \\ 
    ~           & One-step & 1.0              \\ \hline
    (30,60,100) & Full     & ~             \\ 
    ~           & One-step & 3.2              \\ \hline
    \end{tabular}
    \caption{Comaprison of computation times (averaged over 50 replications) for full and one-step versions of the JMMLE algorithm.}
    \label{table:simtable42}
\end{table}

%\paragraph{Parallelization.} While training JMMLE models for multiple values for $\lambda$, it is possible to speed up 


\section{Discussion}

\newpage
\vskip 0.1in
\bibliography{GGMbib}

\newpage
\appendix
\section*{Appendix}
\section{Proofs of main results}
%{\colrbf need modification}

\begin{proof}[Proof of Theorem~\ref{thm:algo-convergence}]
The theorem is a generalization of Theorem 1 in \cite{LinEtal16}. The proof follows directly from the proof of that theorem, substituting $( \widehat B^{(k)}, \widehat \Theta_\epsilon^{(k)}), (B^*, \Theta_\epsilon^*)$ and $ (B^\infty, \Theta_\epsilon^\infty)$ therein with $(\widehat \cB^{(t)}, \widehat \Theta_y^{(t)}), ( \cB_0, \Theta_{y 0})$ and $ (\cB^\infty, \Theta_y^\infty)$, and corresponding variations as required.

\end{proof}

We use the following condition extensively while deriving the results that follow.

\vspace{1em}
\noindent {\bf Condition 3} (Restricted eigenvalues). A symmetric matrix $\bfM \in \BM(b,b)$ is said to satisfy the restricted eigenvalue or RE condition with parameters $\psi, \phi >0$, denoted as curvature and tolerance, respectively, if
%
$$
\bftheta^T \bfM \bftheta \geq \psi \| \bftheta \|^2 - \phi \| \bftheta \|_1^2
$$
%
for all $\bftheta \in \BR^b$. In short, this is denoted by $\bfM \sim RE(\psi, \phi)$.

Starting from \cite{BickelRitovTsybakov09}, different versions of the RE conditions have been proposed and used in high-dimensional analysis \citep{LohWainwright12,BasuMichailidis15,MaMichailidis15,vandeGeerBuhlmann09} to ensure that a covariance matrix satisfies a somewhat relaxed positive-definitess condition.

\begin{proof}[Proof of Theorem~\ref{thm:thm-Theta}]
The proof strategy is as follows. We first show that given fixed $(\cX,\cE)$, and some conditions on $\widehat \bfE^k := \bfY^k - \bfX^k \widehat \bfB^k, k \in \cI_K$, the bounds in in Theorem~\ref{thm:thm-Theta} hold. We then show that for random $(\cX,\cE)$, those conditions hold with probability approaching 1.

\begin{Lemma}\label{thm:ThetaThm}
Assume fixed $\cX, \cE$ and deterministic $\widehat \cB = \{ \widehat \bfB^k \}$, and the following conditions.

\noindent{\bf(A1)} For $k \in \cI_K$,
%
$$
\| \widehat \bfB^k - \bfB^k_0 \|_1 \leq C_\beta \sqrt{\frac{\log (pq)}{n}}
$$
%
with $C_\beta = O(1)$ is non-negative and depends on $\cB_0$ only.

%\noindent{\bf(T2)} $\| \bfX^k (\widehat \bfB^k - \bfB^k_0 ) \|_\infty \leq c(v_\beta)$, where $c(v_\beta)$ is $O(1)$ and depends on $v_\beta$.

\noindent{\bf(A2)} For all $j \in \cI_q$,
%
$$
\frac{1}{n} \left\| (\widehat \bfE_{-j}^k)^T \widehat \bfE^k \bfT_{0,j}^k \right\|_\infty \leq
\BQ \left(C_\beta, \Sigma_{x0}^k, \Sigma_{y0}^k \right) \sqrt {\frac{ \log (p q)}{n}},
$$
%
where $\BQ \left(C_\beta, \Sigma_x^k, \Sigma_y^k \right) = O(1)$ is non-negative and depends on $\cB_0, \Sigma_{x0}^k$ and $\Sigma_{y0}^k$ only.

\noindent{\bf(A3)} Denote $\widehat \bfS^k = (\widehat \bfE^k)^T \widehat \bfE^k/n$. Then $\widehat \bfS^k \sim RE(\psi^k, \phi^k)$ with $Kq \phi \leq \psi/2$ where $ \psi = \min_k \psi^k, \phi = \max_k \phi^k $.

%\noindent{\bf(T4)} Assumption (A2) holds for $\Sigma_y^k$.

Then the following hold

\noindent (I) Given the choice of tuning parameter
%
$$
\gamma_n \geq 4 \sqrt{| g_{\max}|} \BQ_0 \sqrt {\frac{ \log (p q)}{n}}; \quad
\BQ_0 := \max_{k \in \cI_K} \BQ \left(C_\beta, \Sigma_{x0}^k, \Sigma_{y0}^k  \right)
$$
%
\begin{align}
\| \widehat \Theta_j - \Theta_{0,j} \|_F & \leq 12 \sqrt{s_j} \gamma_n / \psi, \label{eqn:theta-norm-bound-1a}\\
\sum_{j \neq j', g \in \cG_y^{jj'}} \| \hat \bftheta_{jj'}^{[g]} - \bftheta_{0,jj'}^{[g]} \| & \leq 48 s_j \gamma_n / \psi. \label{eqn:theta-norm-bound-2a}\\
| \supp (\widehat \Theta_j) | & \leq 128 s_j / \psi
\end{align}
%

\noindent (II) For the choice of tuning parameter $\gamma_n = 4 \sqrt{| g_{\max}|} \BQ_0 \sqrt{\log (p q)/n}$,
%
\begin{align}\label{eqn:OmegaBounds0a}
\frac{1}{K} \sum_{k=1}^K \| \widehat \Omega_y^k - \Omega_{y0}^k \|_F \leq
O \left( \BQ_0 \sqrt{\frac{| g_{\max}| S}{K}}
\sqrt {\frac{ \log (p q)}{n}} \right)
\end{align}
%
%Further if A1 holds with $s = s_0$, and A3 is satisfied then
%
%(II) Direction consistency.
\end{Lemma}

Condition (A1) holds by assumption. When $\cX$ and $\cE$ are random, the following proposition ensures that (A2) and (A3) hold with probabilities approaching to 1.

\begin{Lemma}\label{prop:ErrorRE}
Consider deterministic $\widehat \cB$ satisfying assumption (A1), and conditions (E1), (E2) from the main paper. Then for sample size $n \succsim \log (pq)$ and $k \in \cI_K$,

\begin{enumerate}
\item $\widehat \bfS^k$ satisfies the RE condition: $ \widehat \bfS^k \sim RE (\psi^k, \phi^k)$, where 
%
$$
\psi^k = \frac{ \Lambda_{\min} (\Sigma_{x0}^k)}{2}; \quad \phi^k = \frac{ \psi^k \log p}{n} + 2 C_\beta c_2 [ \Lambda_{\max} (\Sigma_{x0}^k) \Lambda_{\max} (\Sigma_{y0}^k) ]^{1/2} \frac{ \log(pq)}{n}
$$
%
with probability $\geq 1 - 6c_1 \exp [-(c_2^2-1) \log(pq)] - 2 \exp (- c_3 n), c_1, c_3 > 0, c_2 > 1$.
%
\item The following deviation bound is satisfied for any $j \in \cI_q$
%
$$
\left\|\frac{1}{n} (\widehat \bfE_{-j}^k)^T \widehat \bfE^k \bfT_{0,j}^k \right\|_\infty \leq \BQ \left(C_\beta, \Sigma_{x0}^k, \Sigma_{y0}^k \right) \sqrt {\frac{ \log (p q)}{n}}
$$
%

with probability $\geq 1 - 1/p^{\tau_1-2} - 12 c_1 \exp [-(c_2^2-1) \log(pq)] - 6c_4 \exp [-(c_5^2-1) \log(pq)], c_4 > 0, c_5 > 1, \tau_1 > 2$, where
%
\begin{align*}
\BQ \left(C_\beta, \Sigma_{x0}^k, \Sigma_{y0}^k \right) &=
\left[ 2 C_\beta^2 V_x^k + 4 C_\beta c_2 [ \Lambda_{\max} (\Sigma_{x0}^k) \Lambda_{\max} (\Sigma_{y0}^k) ]^{1/2} \right] \sqrt{\frac{ \log(pq)}{n}} +\\
& c_5 \left[ \Lambda_{\max} ( \Sigma_{y0}^k) \sigma_{y0,j,-j}^k \right]^{1/2} \sqrt{\frac{\log q}{\log (pq)}}
\end{align*}
%
with $\sigma_{y0,j,-j}^k = \BV( E_j - \BE_{-j} \bftheta_{0,j})$, and 
%
$$
V_x^k = \sqrt{ \frac{ \log 4 + \tau_1 \log p}{c_x^k n}}; \quad
c_x^k = \left[ 128 ( 1 + 4 \Lambda_{\max} (\Sigma_{x0}^k)  )^2 \max_i (\sigma_{x0,ii}^k)^2 \right]^{-1}
$$
\end{enumerate} 
\end{Lemma}

\noindent We prove the main theorem by putting together Lemma~\ref{thm:ThetaThm} and Lemma~\ref{prop:ErrorRE}.
\end{proof}

\begin{proof}[Proof of Theorem~\ref{thm:thm-B}]
The strategy is same as Theorem~\ref{thm:thm-Theta}. We first show the theorem statements hold for fixed $\cX, \cE$ in presence of some conditions, then show that those conditions are satisfied with probability approaching 1 when $\cX$ and $\cE$ are random.

%The proof strategy is as follows. We first show that given fixed $(\cX,\cE)$, and some conditions on the quantities $\widehat \bfgamma, \widehat \bfGamma$, the bounds in in \ref{thm:thm-B} hold. We then show that for random $(\cX,\cE)$, these conditions hold with probability approaching 1.

\begin{Lemma}\label{thm:BetaThm}
Assume fixed $(\cX, \cE)$, and deterministic $\widehat \Theta = \{ \widehat \Theta_j \}$, so that

\noindent{\bf(B1)} For $j \in \cI_q$,
%
$$
\| \widehat \Theta_j - \Theta_{0,j} \|_F \leq C_\Theta \sqrt{\frac{\log q}{n}},
$$
%
for some $C_\Theta = O(1)$ dependent on $\Theta_0$ only.

\noindent{\bf(B2)} Denote $\widehat \bfGamma^k = (\widehat \bfT^k)^2 \otimes (\bfX^k)^T \bfX^k/n, \widehat \bfgamma^k = (\widehat \bfT^k)^2 \otimes (\bfX^k)^T \bfY^k/n$. Then the deviation bound holds:
%
$$
\left\| \widehat \bfgamma^k - \widehat \bfGamma^k \bfbeta_0 \right\|_\infty \leq \BR( C_\Theta, \Sigma_{x0}^k, \Sigma_{y0}^k) \sqrt{ \frac{ \log(pq)}{n}}.
$$
%
where $\BR (C_\Theta, \Sigma_{x0}^k, \Sigma_{y0}^k ) = O(1)$ depends on $\Theta_0, \Sigma_{x0}^k$ and $\Sigma_{y0}^k$ only.

\noindent{\bf(B3)} $\widehat \bfGamma \sim RE(\psi_*, \phi_*)$ with $Kpq \phi_* \leq \psi_*/2$.

Then, given the choice of tuning parameter
%
$$
\lambda_n \geq 4 \sqrt{| h_{\max} |} \BR_0 \sqrt{ \frac{ \log(pq)}{n}}; \quad 
\BR_0 := \max_{k \in \cI_K} \BR \left(C_\Theta, \Sigma_{x0}^k, \Sigma_{y0}^k \right)
$$
%
the following holds
%
\begin{align}
\| \widehat \bfbeta - \bfbeta_0 \|_1 & \leq 48 \sqrt{ | h_{\max} |} s_\beta \lambda_n / \psi^* \label{eqn:BetaThmEqn1}\\
\| \widehat \bfbeta - \bfbeta_0 \| & \leq 12 \sqrt s_\beta \lambda_n / \psi^* \label{eqn:BetaThmEqn2}\\
\sum_{h \in \cH} \| \bfbeta^{[h]} - \bfbeta_0^{[h]} \| & \leq 48 s_\beta \lambda_n / \psi^* \label{eqn:BetaThmEqn3}\\
(\widehat \bfbeta - \bfbeta_0 )^T \widehat \bfGamma (\widehat \bfbeta - \bfbeta_0 ) & \leq
72 s_\beta \lambda_n^2 / \psi^* \label{eqn:BetaThmEqn4}
\end{align}
\end{Lemma}

Condition (B1) holds by assumption. Next we verify that conditions (B2) and (B3) hold with high probability given fixed $\widehat \Theta$.

\begin{Lemma}\label{prop:ThmBetaRE}
Consider deterministic $\widehat \Theta$ satisfying assumption (B1). Also assume conditions (E3), (E4) from the main paper. Then for sample size $n \succsim \log (pq)$,

\begin{enumerate}
\item $\widehat \bfGamma$ satisfies the RE condition: $ \widehat \bfGamma \sim RE (\psi_*, \phi_*)$, where 
%
$$
\psi_* = \min_k \psi^k \left( \min_i \psi_k^j - d_k C_\Theta \sqrt{ \frac{ \log (pq)}{n}}\right), 
\phi_* = \max_k \phi^k \left( \min_i \phi_k^j + d_k C_\Theta \sqrt{ \frac{ \log (pq)}{n}}\right)
$$
%
with probability $\geq 1 - 2 \exp(c_3 n), c_3>0$.
%
\item The deviation bound in (A2) is satisfied with probability $ \geq 1 - 12 c_1 \exp[ (c_2^2-1) \log (pq)]$, where
%
$$
\BR \left(C_\Theta, \Sigma_{x0}^k, \Sigma_{y0}^k \right) =
c_2 \left\{ d_k C_\Theta \sqrt{ \frac{ \log (pq)}{n}}
[ \Lambda_{\max} (\Sigma_x^k) \Lambda_{\max} (\Sigma_y^k)]^{1/2} +
\left[ \frac{\Lambda_{\max} (\Sigma_x^k)}{\Lambda_{\min} (\Sigma_y^k) } \right]^{1/2} \right\}
$$
%
\end{enumerate}
\end{Lemma}

The theorem is immediate by putting together Lemma~\ref{thm:BetaThm} and Lemma~\ref{prop:ThmBetaRE}.
\end{proof}

\begin{proof}[Proof of Theorem~\ref{thm:starting-values}]
The first part is immediate from the proof of part I of Theorem 4 in \cite{LinEtal16}. By choice of $\lambda_n$, we now have
%
$$
\| \widehat \bfB^{k (0)} - \bfB^k_0 \|_1 = O\left( \sqrt{ \frac{ \log (pq)}{n}} \right),
$$
%
so we can apply Theorem~\ref{thm:thm-Theta} to prove the bounds on $\{\widehat \Theta_j^{(0)} \}$.
\end{proof}

%{\colrbf Discuss tighter bound compared to vanilla JSEM}
%
%After providing the error bounds for solutions to the subproblem \eqref{eqn:EstEqn2}, we concentrate on the subproblem \eqref{eqn:EstEqn1}. Following a similar strategy, we first get error bounds for $\widehat \bfbeta$ assuming everything else fixed.

%with
%$$
%\bfbeta = \begin{bmatrix}
%\ve (\bfB^1)\\
%\vdots\\
%\ve (\bfB^K)\\
%\end{bmatrix}; \quad
%\bfGamma = \begin{bmatrix}
%I_q \otimes (\bfX^1) TX^1 / n) & &\\
%& \ddots &\\
%& & I_q \otimes (\bfX^K)^T X^K / n)
%\end{bmatrix} 
%$$


%%
%\begin{align}
%\| \widehat \bfS^k \bfT_{0,j}^k \|_\infty & \leq
%\| \widehat \bfS^k \|_\infty \| \bfT_{0,j}^k  \|_1 \notag\\
%& \leq \left[ \| \widehat \bfS^k - \bfS^k \|_\infty + \| \bfS^k \|_\infty \right] \| \bfT_{0,j}^k  \|_1 \label{eqn:ErrorRElemma2eqn1}
%\end{align}
%%
%Since $\Sigma_y^k$ is diagonally dominant, (see proof of Proposition~\ref{lemma:LemmaE2}) we have
%%
%\begin{align}\label{eqn:ErrorRElemma2eqn2}
% \| \bfT_{0,j}^k \|_1 \leq 2
%\end{align}
%%
%Moving on to $\| \widehat \bfS^k - \bfS^k \|_\infty$:
%%
%\begin{align}
%\| \widehat \bfS^k - \bfS^k \|_\infty & \leq \left\| \frac{2}{n} (\bfE^k)^T \bfX^k (\bfB^k_0 - \widehat \bfB^k) \right\|_\infty + 
%\left\| (\widehat \bfB^k - \bfB_0^k)^T \left( \frac{(\bfX^k)^T (\bfX^k)}{n} \right) (\widehat \bfB^k) - \bfB_0^k)) \right\|_\infty \notag\\
%& \leq 2 \left\| \frac{(\bfX^k)^T \bfE^k}{n} \right\|_\infty \left\| \bfB^k_0 - \widehat \bfB^k) \right\|_1 + 
%\left\| (\widehat \bfB^k - \bfB_0^k)^T \left( \frac{(\bfX^k)^T (\bfX^k)}{n} \right) (\widehat \bfB^k) - \bfB_0^k)) \right\|_\infty \label{eqn:ErrorRElemma2eqn3}
%\end{align}
%%
%By applying Lemma~\ref{lemma:ErrorRElemma2} on $\bfX^k, \bfE^k$, and assumption (T1) we have for $n \succsim \log (pq)$ and with probability $\geq 1 - 6c_1 \exp [-(c_2^2-1) \log(pq)]$,
%%
%$$
%2 \left\| \frac{(\bfX^k)^T \bfE^k}{n} \right\|_\infty \left\| \bfB^k_0 - \widehat \bfB^k) \right\|_1 \leq
%2 v_\beta c_2 [ \Lambda_{\max} (\Sigma_x^k) \Lambda_{\max} (\Sigma_y^k) ]^{1/2} \sqrt{\frac{ \log(pq)}{n}}
%$$
%%
%The second term is bounded by substituting $\bfX, \hat \bfB, \bfB_0$ with $\bfX^k, \hat \bfB^k, \hat \bfB_0^k$ respectively in (\ref{eqn:ErrorREeqn2}).
%
%For a bound on the $\ell_\infty$-norm of $\bfS^k = (\bfX^k)^T \bfX^k/n$, we apply Lemma 8 of \cite{RavikumarEtal11} on the (sub-gaussian) design matrix $\bfX^k$ to obtain that for sample size
%$$
%n \geq 512 ( 1 + 4 \Lambda_{\max} (\Sigma_x^k))^4 \max_j (\sigma_{x,jj}^k )^4 \log (4p^{\tau_1})
%$$
%%
%we get that with probability $ \geq 1 - 1/p^{\tau_1-2}$,
%%
%\begin{align}\label{eqn:ErrorRElemma2eqn4}
%\| \bfS^k \|_\infty & \leq \sqrt{ \frac{ \log 4 + \tau_1 \log p}{c_x^k n}} + \max_j \sigma_{x,jj}^k; \quad
%c_x^k = \left[ 128 ( 1 + 4 \Lambda_{\max} (\Sigma_x^k)  )^2 \max_j (\sigma_{x,jj}^k)^2 \right]^{-1}
%\end{align}
%%
%We now bound the right hand side of (\ref{eqn:ErrorRElemma2eqn1}) using (\ref{eqn:ErrorRElemma2eqn2}), (\ref{eqn:ErrorRElemma2eqn3}) and (\ref{eqn:ErrorRElemma2eqn4}), which proves part 3.

%, with an extra step to prove that $(\bfT^k)^2$ is diagonally dominant. For this, denote the elements of $(\bfT^k)^2$ by $\tau_{jj'}, j; j' \in \cI_q$. Then
%%
%\begin{align*}
%\tau_{jj} - \sum_{j \neq j'} | \tau_{jj'}| & = \sum_{l=1}^q \left[
%\frac{\omega_{jl}^2}{\omega_{jj}^2 } - \sum_{j \neq j'} \frac{| \omega_{jl} \omega_{j'l} |}{\omega_{jj} \omega_{j'j'} } \right]
%\end{align*}
%%
%For each summand,
%%
%\begin{align*}
%\frac{\omega_{jl}^2}{\omega_{jj}^2 } - \sum_{j \neq j'} \frac{| \omega_{jl} \omega_{j'l} |}{\omega_{jj} \omega_{j'j'}} & = \frac{| \omega_{jl} |}{\omega_{jj} } \left[ \frac{| \omega_{jl}| }{\omega_{jj} } - \sum_{j' \neq j} \frac{| \omega_{j'l} |}{\omega_{j'j'}} \right] \\
%\end{align*}
%

\begin{proof}[Proof of Theorem~\ref{Thm:ThmTesting}]
Define the following:
%
$$
\widehat \bfD_i = \ve(\widehat \bfb^1_i, \ldots, \widehat \bfb^K_i); \quad
\bfR_i^k = \bfX_i^k - \bfX_{-i}^k \widehat \bfzeta_i^k; k \in \cI_K
$$
%
Then from \eqref{eqn:DebiasedBeta} we have
%
\begin{align}\label{eqn:ThmTestingProofeq1}
\bfM_i ( \widehat \bfC_i - \widehat \bfD_i )^T &= \frac{1}{\sqrt n}
\begin{bmatrix}
\frac{1}{\widehat s^1_i} (\bfR^1_i)^T \widehat \bfE^1 \\
\vdots\\
\frac{1}{\widehat s^K_i} (\bfR^K_i)^T \widehat \bfE^K
\end{bmatrix}
\end{align}
%
We now decompose $\widehat \bfE^k:$
%
\begin{align*}
\widehat \bfE^k &= \bfY^k - \bfX^k \widehat \bfB^k \\
&= \bfE^k + \bfX^k (\bfB_0^k - \widehat \bfB^k)\\
&= \bfE^k + \bfX_i^k (\bfb_{0i}^k - \widehat \bfb_i^k) + \bfX_{-i}^k (\bfB_{0,-i}^k - \widehat \bfB_{-i}^k)
\end{align*}
%
Putting them back in \eqref{eqn:ThmTestingProofeq1} and using $t_i^k = (\bfR_i^k)^T \bfX_i^k/n$,
%
\begin{align}
\bfM_i ( \widehat \bfC_i - \widehat \bfD_i)^T &= \frac{1}{\sqrt n}
\begin{bmatrix}
\frac{1}{\widehat s^1_i} (\bfR^1_i)^T \bfE^1 \\
\vdots\\
\frac{1}{\widehat s^K_i} (\bfR^K_i)^T \bfE^K
\end{bmatrix} +
\bfM_i (\bfD_i - \widehat \bfD_i )^T \notag\\
& + \frac{1}{\sqrt n}
\begin{bmatrix}
\frac{1}{\widehat s^1_i} (\bfR^1_i)^T \bfX_{-i}^1 (\bfB_{0,-i}^1 - \widehat \bfB_{-i}^1) \\
\vdots\\
\frac{1}{\widehat s^K_i} (\bfR^K_i)^T \bfX_{-i}^K (\bfB_{0,-i}^K - \widehat \bfB_{-i}^K)
\end{bmatrix} \notag\\
\Rightarrow
\widehat \Omega_y^{1/2} \bfM_i ( \widehat \bfC_i - \bfD_i)^T &=
\frac{\widehat \Omega_y^{1/2}}{\sqrt n}
\begin{bmatrix}
\frac{1}{\widehat s^1} (\bfR^1_i)^T \bfE^1 \\
\vdots\\
\frac{1}{\widehat s^K} (\bfR^K_i)^T \bfE^K
\end{bmatrix} +
\frac{\widehat \Omega_y^{1/2}}{\sqrt n}
\begin{bmatrix}
\frac{1}{\widehat s^1_i} (\bfR^1_i)^T \bfX_{-i}^1 (\bfB_{0,-i}^1 - \widehat \bfB_{-i}^1) \\
\vdots\\
\frac{1}{\widehat s^K_i} (\bfR^K_i)^T \bfX_{-i}^K (\bfB_{0,-i}^K - \widehat \bfB_{-i}^K)
\end{bmatrix}\label{eqn:ThmTestingProofeq2}
\end{align}

At this point, we drop $k$ and 0 in the subscripts, and prove the following:

\begin{Lemma}\label{Lemma:ThmTestingLemma}
Given conditions (T1) and (T2), the following holds for sample size $n$ such that $n \succsim \log (pq)$:
%
$$
\frac{1}{\sqrt n \widehat s_i}  \widehat \Omega_y^{1/2} \bfE^T \bfR_i \sim
\cN_q ({\bf 0}, \bfI) + \bfS_{1n};
$$
%
\begin{align}\label{eqn:ThmTestingProofeq3}
\| \bfS_{1n} \|_\infty & \leq 
\frac{D_\Omega^{1/2} (2 + D_\zeta) c_2 [ \Lambda_{\max} (\Sigma_x) \Lambda_{\max} (\Sigma_e) ]^{1/2} \sqrt{ \log(pq)}}{\sqrt{ \sigma_{x,i,-i}} - n^{-1/4} - D_\zeta \sqrt{V_x}} =
O \left( \frac{ \log(pq)}{\sqrt n} \right)
\end{align}
%
with probability $\geq 1 - 6c_1 \exp[-(c_2^2-1) \log (p q)] - 1/p^{\tau_1-2} - \kappa_i/\sqrt n$, where $\kappa_i := \BV [(X_i - \BX_{-i} \bfzeta_{0,-i})^2]$.

Additionally, given condition (T3)
%
\begin{align}\label{eqn:ThmTestingProofeq4}
& \left\| \frac{1}{\sqrt n \widehat s_i} \bfR_i^T \bfX_{-i} (\bfB_{-i} - \widehat \bfB_{-i})
\widehat \Omega_y^{1/2} \right\|_\infty \notag\\
& \leq
\frac{D_\beta ( \Lambda_{\min} (\Sigma_y)^{1/2} + D_\Omega^{1/2})}{\sigma_{x,i,-i} - n^{-1/2} - D_\zeta \sqrt{V_x}} 
\left[ c_{7} \sqrt{ (\sqrt{ \sigma_{x,i,-i}} \Lambda_{\max} (\Sigma_{x, -i}) ) \log p} + \sqrt n D_\zeta V_x \right] =
O \left( \frac{ \log(pq)}{\sqrt n} \right)
\end{align}
%
holds with probability $\geq 1 - 6c_6 \exp[-(c_7^2-1) \log (p q)] - 1/p^{\tau_1-2} - \kappa_i/\sqrt n$ for some $c_6 > 0, c_7 > 1$.
\end{Lemma}

Given Lemma~\ref{Lemma:ThmTestingLemma}, the first and second summands on the right hand side of \eqref{eqn:ThmTestingProofeq2} are bounded above by applying each of \eqref{eqn:ThmTestingProofeq3} and \eqref{eqn:ThmTestingProofeq4} $K$ times. This completes our proof.
\end{proof}

\begin{proof}[Proof of Theorem~\ref{thm:PowerThm}]
From \eqref{eqn:ThmTestingProofeq2} and Lemma~\ref{Lemma:ThmTestingLemma} we have that
%
\begin{align}
(\widehat \Omega_y^k)^{1/2} m_i^k ( \widehat \bfc_i^k - \bfb_{0i}^k) \sim \cN_q ({\bf 0}, \bfI) + \bfS_{2n}^k,
\end{align}
%
where $\| \bfS_{2n}^k \|_\infty = o_P(1)$. We now have the following lemma:
%
\begin{Lemma}\label{Lemma:PowerThmLemma}
Drop $k$ in superscripts and 0 in subscripts. Given condition (T1), the following holds with probability $\geq 1 - 6c_{6} \exp [-(c_{7}^2-1) \log (p-1)] - 1/p^{\tau_2-2} - \kappa_i / \sqrt n, \tau_2 > 2$:
%
\begin{align}
\left| \frac{m_i}{\sqrt n} - \sqrt{ \sigma_{x,i,-i}} \right| & \leq
\delta_i := \sqrt{ \frac{ \log 4 + \tau_2}{c_i n}} +
\frac{D_\zeta + \| \bfzeta_{0i} \|_1 }{\sqrt{ \sigma_{x,i,-i}} - n^{-1/2} - D_\zeta \sqrt{V_x}} \times \notag\\
& \left[ c_{7} [(\sigma_{x,i,-i} \Lambda_{\max} (\Sigma_{x, -i})]^{1/2} \sqrt{ \frac{\log p}{n}} + D_\zeta V_x \right], \label{eqn:PowerThmLemmaEqn}
\end{align}
%
where $c_i = [ 128 (1 + 4 \sigma_{x,i,-i})^2 (\sigma_{x,i,-i})^2 ]^{-1}$, and the sample size satisfies $n \succsim \log p$.
\end{Lemma}
%

We also have a general result:

\begin{Lemma}\label{lemma:omega-diff}
Consider two positive definite matrices $\bfA, \bfA_1 \in \BM(a,a)$. Then for $\delta > 0$,
%
$$
\| \bfA  -\bfA_1 \|_\infty \leq \delta \Rightarrow \| \bfA^{1/2} - \bfA_1^{1/2} \|_\infty \leq \sqrt{ \delta}.
$$
\end{Lemma}
%

\noindent Applying Lemma~\ref{lemma:omega-diff} it is immediate from assumption (T2) that
%
\begin{align}\label{eqn:omega-sqrt-bound}
\left\| \widehat \Omega_y^{1/2} - \Omega_y^{1/2} \right\|_\infty \leq \sqrt{ D_{\Omega}}
\end{align}

Using Lemma~\ref{Lemma:PowerThmLemma} in conjunction with \eqref{eqn:omega-sqrt-bound} we now have
%
\begin{align}
\sqrt n (\Omega_{y0}^k)^{1/2} \sqrt{\sigma_{x0,i,-i}^k} (\widehat \bfc_i^k - \bfb_{0i}^k)  \sim
\cN_q ( {\bf 0}, \bfI) + \bfS_{3n}^k \notag\\
\Rightarrow \sqrt n \Sigma_i^{-1/2} (\widehat \bfc_i^1 - \widehat \bfc_i^2 - \bfdelta) \sim
\cN_q \left( {\bf 0}, \bfI \right) + \bfS_{3n},
\label{eqn:PowerThmProofEqn1}
\end{align}
%
where $\Sigma_i := \Sigma_{y0}^1/ \sigma_{x0,i,-i}^1 + \Sigma_{y0}^2/ \sigma_{x0,i,-i}^2$ and $\bfS_{3n} = \bfS_{3n}^1 + \bfS_{3n}^2, \|\bfS_{3n}^k\|_\infty = o_P(1)$. We now break down the left hand side above as
%
\begin{align}
\sqrt n \Sigma_i^{-1/2} (\widehat \bfc_i^1 - \widehat \bfc_i^2 - \bfdelta) &=
\sqrt n \Sigma_i^{-1/2} \widehat \Sigma_i^{1/2} \widehat \Sigma_i^{-1/2} (\widehat \bfc_i^1 - \widehat \bfc_i^2) - \sqrt n \Sigma_i^{-1/2} \bfdelta \notag\\
&= (\Sigma_i^{-1/2} \widehat \Sigma_i^{1/2} - \bfI) . \sqrt n \widehat \Sigma_i^{-1/2} (\widehat \bfc_i^1 - \widehat \bfc_i^2) + \notag\\
& \sqrt n \widehat \Sigma_i^{-1/2} (\widehat \bfc_i^1 - \widehat \bfc_i^2) - \sqrt n \Sigma_i^{-1/2} \bfdelta,
\label{eqn:PowerThmProofEqn2}
\end{align}
%
with
%
$$
\widehat \Sigma_i :=
\frac{ n \widehat \Sigma_y^1}{(m_i^1)^2} + \frac{n \widehat \Sigma_y^2}{(m_i^2)^2}.
$$
%
Now we have the following lemma:
%
\begin{Lemma}\label{Lemma:PowerThmLemma2}
Given conditions (T1) and (T2), for the pooled covariance matrix estimate $\widehat \Sigma_i$, we have
%
$$
\left\| \widehat \Sigma_i - \Sigma_i \right\|_\infty = o(1).
$$
%
for sample size $n \succsim \log p$.
\end{Lemma}
%
Lemma~\ref{lemma:omega-diff} now implies that $\| \widehat \Sigma_i^{1/2} - \Sigma_i^{1/2} \|_\infty = o(1)$. Putting this in the first summand of \eqref{eqn:PowerThmProofEqn2}, then using \eqref{eqn:PowerThmProofEqn1} we get
%
$$
\sqrt n \widehat \Sigma_i^{-1/2} (\widehat \bfc_i^1 - \widehat \bfc_i^2) - \sqrt n \Sigma_i^{-1/2} \bfdelta
\sim \cN_q \left( {\bf 0}, \bfI \right) + \bfS_{4n},
$$
%
with $\| \bfS_{4n} \|_\infty = o_P(1)$. The power of the global test follows as a consequence. Finally, the lower bound on the order of $\| \bfdelta \|$ holds because $n \bfdelta^T \Sigma_i^{-1} \bfdelta \geq n \| \bfdelta \|^2 \Lambda_{\min} (\Sigma_i^{-1}) $, and
%
$$
\Lambda_{\min} (\Sigma_i^{-1}) = \frac{ \Lambda_{\max} (\Sigma_{y0}^1)}{\sigma_{x,i,-i}^1} +
\frac{ \Lambda_{\max} (\Sigma_{y0}^2)}{\sigma_{x,i,-i}^2}
$$
%
\end{proof}

\begin{proof}[Proof of Theorem~\ref{thm:FDRthm}]
The proof follows the general structure of Theorem 4.1 in \cite{LiuShao14}, with two modifications. Firstly we replace the bound in equation (12) of \cite{LiuShao14} by a new deviation bound
%
$$
P \left( \left| d_{ij} - \frac{\mu_j}{\sigma_j} \right| \geq t \right) = (1 - \Phi(t))(1 + o(1))
$$
%
for any $t$, since $(d_{ij} - \mu_j)/\sigma_j \sim N(0,1) + o_P(1)$ from Corollary~\ref{corollary:CorTesting}. We replace $G_\kappa(t)$ in all following calculations in \cite{LiuShao14} with $1 - \Phi(t)$. Secondly, we need to ensure that given both $\Sigma_{y0}^1$ and $\Sigma_{y0}^2$ satisfy the condition (D1) or (D1*), the pooled covariance matrix $\Sigma_{y0}^1/ \sigma_{x0,i,-i}^1 + \Sigma_{y0}^2/  \sigma_{x0,i,-i}^2$ also does so.

For this, denote $c_k =  \sigma_{x0,i,-i}^k, k = 1,2$. Notice that for any $C_1, C_2 > 0$,
%
\begin{align*}
r_{jj'}^k \geq C_k & \Rightarrow \sigma_{y0,jj'}^k \geq (\sigma_{y0,jj}^k \sigma_{y0,j'j'}^k)^{1/2} C_k\\
& \Rightarrow \frac{\sigma_{y0,jj'}^1}{c_1} + \frac{\sigma_{y0,jj'}^2}{c_2} \geq
\frac{ (\sigma_{y0,jj}^1 \sigma_{y0,j'j'}^1)^{1/2} C_1}{c_1} +
\frac{ (\sigma_{y0,jj}^2 \sigma_{y0,j'j'}^2)^{1/2} C_2}{c_2}\\
& \Rightarrow \frac{ \sigma_{y0,jj'}^1/c_1 + \sigma_{y0,jj'}^2/c_2}
{ (\sigma_{y0,jj}^1 \sigma_{y0,j'j'}^1)^{1/2}/ c_1 + (\sigma_{y0,jj}^2 \sigma_{y0,j'j'}^2)^{1/2}/ c_2}
\geq \min \{ C_1, C_2 \}
\end{align*}
%
It is now immediate that (D1) or (D1*) holds for the pooled covariance matrices.
\end{proof}



\section{Proofs of auxiliary results}

\begin{proof}[Proof of Lemma~\ref{thm:ThetaThm}]
The proof has the same structure as the proof of Theorem 1 in \cite{MaMichailidis15}, where they prove consistency of the (single layer) JSEM estimates. Part (I) is analogous to part A.1 therein, but the proof strategy is completely different, which we show in detail. Our part (II) follows similar lines as their parts A.2 and A.3, incorporating the updated quantities from the first part. For this we provide outlines and leave the details to the reader.

\paragraph{Proof of part (I).}
%The following proposition establishes error bounds for estimated neighborhood coefficients in the Y-network.
%
%\begin{Proposition}\label{prop:Thm1ProofProp2}
%Consider the estimation problem in (\ref{eqn:EstEqn2}) and take $\gamma_n \geq 4 \sqrt{| g_{\max}|} \BQ_0$. Given the conditions (T2) and (T3) hold, for any solution of (\ref{eqn:EstEqn2}) we shall have
%%
%\begin{align}
%\| \widehat \Theta_j - \Theta_{0,j} \|_F & \leq 12 \sqrt{s_j} \gamma_n / \psi \label{eqn:Thm1ProofProp2Bd1}\\
%\sum_{j \neq j', g \in \cG_y^{jj'}} \| \hat \bftheta_{jj'}^{[g]} - \bftheta_{0,jj'}^{[g]} \| & \leq 48 s_j \gamma_n / \psi \label{eqn:Thm1ProofProp2Bd2}
%\end{align}
%%
%Also denote the non-zero support of $\widehat \Theta_j$ by $\widehat \cS_j$, i.e. $\widehat \cS_j = \{ (j',g): \hat \bftheta_{jj'}^{[g]} \neq {\bf 0} \}$. Then
%%
%\begin{align}
%| \widehat \cS_j| \leq 128 s_j / \psi \label{eqn:Thm1ProofProp2Bd3}
%\end{align}
%\end{Proposition}
In its reparametrized version, (\ref{eqn:EstEqn2}) becomes
%
\begin{align}
\widehat \bfT_j = \argmin_{\bfT_j} \left\{ \frac{1}{n} \sum_{k=1}^K \| (\bfY^k - \bfX^k \widehat \bfB^k) \bfT_j^k \|^2 + \gamma_n \sum_{j \neq j', g \in \cG_y^{jj'}} \| \bfT_{jj'}^{[g]} \| \right\}
\end{align}
%
with $\bfT_{jj'}^{[g]} := (T_{jj'}^k)_{k \in g}$. Now for any $\bfT_j \in \BM(q, K)$ we have
%
$$
\frac{1}{n} \sum_{k=1}^K \| (\bfY^k - \bfX^k \widehat \bfB^k) \widehat \bfT_j^k \|^2 + \gamma_n \sum_{j \neq j', g \in \cG_y^{jj'}} \| \widehat \bfT_{jj'}^{[g]} \| \leq
\frac{1}{n} \sum_{k=1}^K \| (\bfY^k - \bfX^k \widehat \bfB^k) \bfT_j^k \|^2 + \gamma_n \sum_{j \neq j', g \in \cG_y^{jj'}} \| \bfT_{jj'}^{[g]} \|
$$
%
For $\bfT_j = \bfT_{0,j}$ this reduces to
%
\begin{align}\label{eqn::Thm1ProofProp2Eqn1}
\sum_{k=1}^K (\bfd_j^k)^T \widehat \bfS^k \bfd_j^k & \leq - 2 \sum_{k=1}^K (\bfd_j^k)^T \widehat \bfS^k \bfT_{0,j}^k + \gamma_n \sum_{j \neq j', g \in \cG_y^{jj'}} \left( \| \bfT_{jj'}^{[g]} \| -  \| \bfT_{jj'}^{[g]} + \bfd_{jj'}^{[g]}\| \right)
\end{align}
%
with $\bfd_{j}^k := \widehat \bfT_j^k - \bfT_{0,j}^k$ etc. For the $k^\text{th}$ summand in the first term on the right hand side, since $d_{jj}^k = 0$, $\widehat \bfE^k \bfd_j^k = \widehat \bfE_{-j}^k \bfd_{-j}^k$. Thus
%
\begin{align*}
\sum_{k=1}^K \left| (\bfd_j^k)^T \widehat \bfS^k \bfT_{0,j}^k \right| &=
\sum_{k=1}^K \left| \bfd_j^k. \frac{1}{n} (\widehat \bfE^k)^T \widehat \bfE^k \bfT_{0,j}^k \right| \\
& \leq \sum_{k=1}^K \left\| \frac{1}{n} (\widehat \bfE_{-j}^k)^T \widehat \bfE^k \bfT_{0,j}^k \right\|_\infty \| \bfd_{-j}^k \|_1 \\
& \leq \left[ \sum_{j \neq j', g \in \cG_y^{jj'}} \| \bfd_{jj'}^{[g]} \| \right]
\BQ_0 \sqrt {| g_{\max}|} \sqrt{ \frac{\log (pq)}{n}}
\end{align*}
%
by assumption (A2). For the second term, suppose $\cS_{0,j}$ is the support of $\Theta_{0,j}$, i.e. $\cS_{0,j} = \{ (j',g): \bftheta_{jj'}^{[g]} \neq 0 \}$. Then
%
\begin{align*}
\sum_{j \neq j', g \in \cG_y^{jj'}} \left( \| \bfT_{jj'}^{[g]} \| -  \| \bfT_{jj'}^{[g]} + \bfd_{jj'}^{[g]}\| \right) & \leq
\sum_{(j',g) \in \cS_{0,j}} \left( \| \bfT_{jj'}^{[g]} \| -  \| \bfT_{jj'}^{[g]} + \bfd_{jj'}^{[g]}\| \right) -
\sum_{(j',g) \notin \cS_{0,j}} \| \bfd_{jj'}^{[g]} \|\\
& \leq \sum_{(j',g) \in \cS_{0,j}} \| \bfd_{jj'}^{[g]} \| - \sum_{(j',g) \notin \cS_{0,j}} \| \bfd_{jj'}^{[g]} \|
\end{align*}
%
so that by choice of $\gamma_n$, (\ref{eqn::Thm1ProofProp2Eqn1}) reduces to
%
\begin{align}\label{eqn:Thm1ProofProp2Eqn3}
\sum_{k=1}^K (\bfd_j^k)^T \widehat \bfS^k \bfd_j^k & \leq 
\frac{\gamma_n}{2} \left[ \sum_{(j',g) \in \cS_{0,j}} \| \bfd_{jj'}^{[g]} \| + \sum_{(j',g) \notin \cS_{0,j}} \| \bfd_{jj'}^{[g]} \| \right] +
\gamma_n \left[ \sum_{(j',g) \in \cS_{0,j}} \| \bfd_{jj'}^{[g]} \| - \sum_{(j',g) \notin \cS_{0,j}} \| \bfd_{jj'}^{[g]} \| \right] \notag\\
& = \frac{3 \gamma_n}{2} \sum_{(j',g) \in \cS_{0,j}} \| \bfd_{jj'}^{[g]} \| - \frac{\gamma_n}{2} \sum_{(j',g) \notin \cS_{0,j}} \| \bfd_{jj'}^{[g]} \| \notag\\
& \leq \frac{3 \gamma_n}{2} \sum_{j \neq j', g \in \cG_y^{jj'}} \| \bfd_{jj'}^{[g]} \|
\end{align}
%
Since the left hand side is $\geq 0$, this also implies
%
$$
\sum_{(j',g) \notin \cS_{0,j}} \| \bfd_{jj'}^{[g]} \| \leq 3 \sum_{(j',g) \in \cS_{0,j}} \| \bfd_{jj'}^{[g]} \| \quad \Rightarrow
\sum_{j \neq j', g \in \cG_y^{jj'}} \| \bfd_{jj'}^{[g]} \| \leq
4 \sum_{(j',g) \in \cS_{0,j}} \| \bfd_{jj'}^{[g]} \| \leq 4 \sqrt{s_j} \| \bfD_j \|_F
$$
% 
with $\bfD_j = (\bfd_j^1, \ldots, \bfd_j^K)$. Now the RE condition on $\widehat \bfS^k$ means that
%
$$
\sum_{k=1}^K (\bfd_j^k)^T \widehat \bfS^k \bfd_j^k \geq 
\sum_{k=1}^K \left( \psi_k \| \bfd_j^k \|^2 - \phi_k \| \bfd_j^k \|_1^2 \right) \geq
\psi \| \bfD_j \|_F^2 - \phi \| \bfD_j \|_1^2 \geq 
(\psi - Kq \phi ) \| \bfD_j \|_F^2 \geq \frac{\psi}{2}  \| \bfD_j \|_F^2
$$
%
by assumption (A3).

Combining the above with \eqref{eqn:Thm1ProofProp2Eqn3}, we finally have
%
\begin{align}\label{eqn:Thm1ProofProp2Eqn2}
\frac{\psi}{3} \| \bfD_j \|_F^2 & \leq
\gamma_n \sum_{j \neq j', g \in \cG_y^{jj'}} \| \bfd_{jj'}^{[g]} \| \leq
4 \gamma_n \sqrt{s_j} \| \bfD_j\|_F
\end{align}
%
Since
%
$$
(\bfD_j)_{j',k} = \widehat T_{jj'}^k - T_{0,jj'}^k = \begin{cases}
0 \text{ if } j = j'\\
-(\widehat \theta_{jj'}^k - \theta_{0,jj'}^k ) \text{ if } j \neq j'
\end{cases}
$$
%
The bounds in \eqref{eqn:theta-norm-bound-1a} and \eqref{eqn:theta-norm-bound-2a} are obtained by replacing the corresponding elements in (\ref{eqn:Thm1ProofProp2Eqn2}).

For the bound on $| \widehat \cS_j| := |\supp( \widehat \Theta_j)|$, notice that if $\hat \bftheta_{jj'}^{[g]} \neq 0$ for some $(j',g)$,
%
\begin{align*}
\frac{1}{n} \sum_{k \in g} \left| ((\widehat \bfE_{-j}^k)^T \widehat \bfE^k ( \widehat \bfT_j^k - \bfT_{0,j}^k ))^{j'} \right| & \geq
\frac{1}{n} \sum_{k \in g} \left| ((\widehat \bfE_{-j}^k)^T \widehat \bfE^k \widehat \bfT_j^k )^{j'} \right| - \frac{1}{n} \sum_{k \in g} \left| ((\widehat \bfE_{-j}^k)^T \widehat \bfE^k \bfT_{0,j}^k )^{j'} \right|\\
& \geq |g| \gamma_n - \sum_{k \in g} \BQ ( C_\beta, \Sigma_x^k, \Sigma_y^k ) \sqrt{ \frac{ \log (pq)}{n}}
\end{align*}
%
using the KKT condition for (\ref{eqn:EstEqn2}) and assumption (A2). The choice of $\gamma_n$ now ensures that the right hand side is $\geq 3|g| \gamma_n / 4$. Hence
%
\begin{align*}
| \hat \cS_j| & \leq \sum_{(j',g) \in \widehat \cS_j} \frac{16}{9 n^2 |g|^2 \gamma_n^2 } \sum_{k \in g} \left| ((\widehat \bfE_{-j}^k)^T \widehat \bfE^k ( \widehat \bfT_j^k - \bfT_{0,j}^k ))^{j'} \right|^2\\
& \leq \frac{16}{9 \gamma_n^2} \sum_{k=1}^K \frac{1}{n} \left\| (\widehat \bfE_{-j}^k)^T \widehat \bfE^k ( \widehat \bfT_j^k - \bfT_{0,j}^k ) \right\|^2 \\
& = \frac{16}{9 \gamma_n^2} \sum_{k=1}^K (\bfd_j^k)^T \widehat \bfS^k \bfd_j^k \\
& \leq \frac{8}{3 \gamma_n} \sum_{j \neq j', g \in \cG_y^{jj'}} \| \bfd_{jj'}^{[g]} \| \leq \frac{128 s_j}{\psi} 
\end{align*}
%
using (\ref{eqn:Thm1ProofProp2Eqn3}) and (\ref{eqn:Thm1ProofProp2Eqn2}).

\paragraph{Proof of part (II).}
We denote the selected edge set for the $k^\text{th}$ Y-network by $\hat E^k$. Denote its population version by $E_0^k$. Further, let
%
$$
\tilde \Omega_y^k = \diag (\Omega_{y0}^k) + \Omega_{y, E_0^k \cap \hat E^k}^k
$$
%
With similar derivations to the proof of Corollary A.1 in \cite{MaMichailidis15}, The following two upper bounds can be established:
%
\begin{align}
| \hat E^k | \leq \frac{ 128 S }{\psi} \label{eqn:Thm1ProofProp2Bd4}\\
\frac{1}{K} \sum_{k=1}^K \| \tilde \Omega_y^k - \Omega_{y0}^k \|_F \leq
\frac{12 c_y \sqrt{S} \gamma_n} {\sqrt K \psi} \label{eqn:Thm1ProofProp2Bd5}
\end{align}
%
following which, taking $\gamma_n = 4 \sqrt{| g_{\max}|} \BQ_0 \sqrt{ \log (pq)/ n}$,
%
\begin{align}
\Lambda_{\min} ( \tilde \Omega_y^k) \geq d_y - \frac{48 c_y \BQ_0 \sqrt{| g_{\max}| S}}{ \psi} \geq (1 - t_1) d_y > 0 \label{eqn:Thm1ProofProp2Bd6}\\
\Lambda_{\max} ( \tilde \Omega_y^k) \leq c_y + \frac{48 c_y \BQ_0 \sqrt{| g_{\max}| S}}{ \psi} \leq c_y + t_1 d_y < \infty \label{eqn:Thm1ProofProp2Bd7}
\end{align}
%
with $0 < t_1 < 1$, and the sample size $n$ satisfying
%
$$
n \geq | g_{\max}| S \BQ_0 \left[ \frac{48 c_y}{\psi t_1 d_y} \right]^2 \sqrt{ \log (pq)}.
$$

Following the same steps as part A.3 in the proof of Theorem 4.1 in \cite{MaMichailidis15}, it can be proven using \eqref{eqn:Thm1ProofProp2Bd4}--\eqref{eqn:Thm1ProofProp2Bd7} that
%
$$
\sum_{k=1}^K \| \widehat \Omega_y^k - \tilde \Omega_y^k \|^2 \leq O \left( \BQ_0^2 | g_{\max}| S \right)
$$
%
The proof is now complete by combining this with \eqref{eqn:Thm1ProofProp2Bd5} then applying Cauchy-Schwarz inequality and triangle inequality.
\end{proof}

\begin{proof}[Proof of Lemma~\ref{prop:ErrorRE}]

%For any sub-gaussian zero-mean design matrix $\bfX \in \BM(n,p)$ with parameters $(\Sigma_x, \sigma_x^2)$, and any $\hat \bfB, \bfB_0 \in \BM(p,q)$ such that $\| \hat \bfB - \bfB_0 \|_F \leq v_\beta$, we follow the proof of Proposition 3 in \cite{LinEtal16} to obtain that for sample size
%%
%\begin{align}\label{eqn:ErrorREeqn1}
%n & \geq 512 ( 1 + 4 \sigma_x^2)^4 \max_j (\Sigma_{x,jj})^4 \log (4p^{\tau_1})
%\end{align}
%the following holds
%%
%\begin{align}\label{eqn:ErrorREeqn2}
%\left\| (\widehat \bfB - \bfB_0)^T \left( \frac{\bfX^T \bfX}{n} \right) (\widehat \bfB - \bfB_0) \right\|_\infty & \leq
%v_\beta^2 \left[ \sqrt{ \frac{ \log 4 + \tau_1 \log p}{c_x n}} + \max_j \Sigma_{x,jj} \right]
%\end{align}
%%
%with probability $\geq 1 - 1/p^{\tau_1-2}$ for some $\tau_1>2$, where
%%
%$$
%c_x = \left[ 128 ( 1 + 4 \sigma_x^2)^2 \max_j (\Sigma_{x,jj})^2 \right]^{-1}
%$$
%%
%Here we substitute $\bfX, \hat \bfB, \bfB_0$ with $\bfX^k, \hat \bfB^k, \hat \bfB_0^k$ respectively. Since rows of $\bfX^k$ come independently from $\cN( {\bf 0}, \Sigma_x^k)$, $\sigma_x^2$ in our case is the spectral norm of $\Sigma_x^k$ \citep{LohWainwright12}, which is $\Lambda_{\max} (\Sigma_x^k)$. Finally
%%
%$$
%\| \bfX^k ( \hat \bfB^k - \bfB_0^k ) \|_\infty \leq 
%\sqrt{\left\| (\widehat \bfB^k - \bfB_0^k)^T (\bfX^k)^T \bfX^k (\widehat \bfB^k - \bfB_0^k) \right\|_\infty}
%$$
%%
%The expression of part 1 is immediate now, and (\ref{eqn:ErrorREeqn1}) ensures that part 1 holds when the leading term of the sample size requirement is $n \succsim \log (pq)$.

We drop the subscript 0 for true values and the superscript $k$ since there is no scope of ambiguity. For part 1, we start with an auxiliary lemma:
%
\begin{Lemma}\label{lemma:ErrorRElemma1}
For a sub-gaussian design matrix $\bfX \in \BM(n,p)$ with columns having mean ${\bf 0}_p$ and covariance matrix $\Sigma_x$, the sample covariance matrix $\widehat \Sigma_x = \bfX^T \bfX/n$ satisfies the RE condition
%
$$
\widehat \Sigma_x \sim RE \left( \frac{\Lambda_{\min} ( \Sigma_x) }{2}, \frac{\Lambda_{\min} ( \Sigma_x) \log p }{2 n} \right)
$$
%
with probability $\geq 1 - 2 \exp(-c_3 n)$ for some $c_3 > 0$.
\end{Lemma}
%
Now denote $\widehat \bfE = \bfY - \bfX \widehat \bfB$. For $\bfv \in \BR^q$, we have
%
\begin{align}\label{eqn:ErrorREeqn3}
\bfv^T \widehat \bfS \bfv &= \frac{1}{n} \| \widehat \bfE \bfv \|^2 \notag\\
&= \frac{1}{n} \| (\bfE + \bfX ( \bfB_0 - \widehat \bfB ))\bfv \|^2 \notag\\
&= \bfv^T \bfS \bfv + \frac{1}{n} \| \bfX (\bfB_0 - \widehat \bfB) \bfv \|^2 + 2 \bfv^T (\bfB_0 - \widehat \bfB)^T \left( \frac {(\bfX)^T \bfE}{n} \right) \bfv
\end{align}
%
For the first summand, $ \bfv^T \bfS^k \bfv \geq \psi_y \| \bfv \|^2 - \phi_y \| \bfv \|_1^2$ with $\psi_y = \Lambda_{\min} (\Sigma_y)/2, \phi_y = \psi_y \log p/n$ by applying Lemma \ref{lemma:ErrorRElemma1} on $\bfS$. The second summand is greater than or equal to 0. For the third summand,
%
$$
2 \bfv^T (\bfB_0 - \widehat \bfB)^T \left( \frac {(\bfX)^T \bfE}{n} \right) \bfv \geq
-2 C_\beta \left\| \frac {(\bfX)^T \bfE}{n} \right\|_\infty \| \bfv \|_1^2
\sqrt{ \frac{ \log (pq)}{n}}
$$
%
by assumption (A1). Now we use another lemma:
%
\begin{Lemma}\label{lemma:ErrorRElemma2}
For zero-mean independent sub-gaussian matrices $\bfX \in \BM(n,p), \bfE \in \BM(n,q)$ with parameters $(\Sigma_x, \sigma_x^2)$ and $(\Sigma_e, \sigma_e^2)$ respectively, given that $n \succsim \log(pq)$ the following holds with probability $\geq 1 - 6c_1 \exp [-(c_2^2-1) \log(pq)]$ for some $c_1 >0, c_2 > 1$:
%
$$
\frac{1}{n} \| \bfX^T \bfE \|_\infty \leq c_2 [ \Lambda_{\max} (\Sigma_x) \Lambda_{\max} (\Sigma_e) ]^{1/2} \sqrt{\frac{ \log(pq)}{n}}
$$
%
\end{Lemma}
%
Subsequently we collect all summands in (\ref{eqn:ErrorREeqn3}) and get
%
$$
\bfv^T \widehat{ \bfS} \bfv \geq \psi_y \| \bfv \|^2 - \left( \phi_y + 2 C_\beta c_2 [ \Lambda_{\max} (\Sigma_x) \Lambda_{\max} (\Sigma_y) ]^{1/2} \frac{ \log(pq)}{n} \right) \| \bfv \|_1^2
$$
with probability $\geq 1 - 2\exp(- c_3 n) - 6c_1 \exp [-(c_2^2-1) \log(pq)]$. This concludes the proof of part 1.

To prove part 2, we decompose the quantity in question:
%
\begin{align}\label{eqn:ErrorRElemma2maineqn}
\left\| \frac{1}{n} \widehat \bfE_{-j}^T \widehat \bfE \bfT_{0,j} \right\|_\infty &=
\left\| \frac{1}{n} \left[ \bfE_{-j} + \bfX (\bfB_{0,j} - \widehat \bfB_j) \right]^T \left[ \bfE + \bfX (\bfB_0 - \widehat \bfB) \right] \bfT_{0,j} \right\|_\infty \notag\\
& \leq \left\| \frac{1}{n} \bfE_{-j}^T \bfE \bfT_{0,j} \right\|_\infty +
\left\| \frac{1}{n} \bfE_{-j}^T \bfX (\bfB_0 - \widehat \bfB) \bfT_{0,j} \right\|_\infty \notag\\
& + \left\| \frac{1}{n} (\bfB_{0,j} - \widehat \bfB_j)^T \bfX^T \bfX (\bfB_0 - \widehat \bfB) \bfT_{0,j} \right\|_\infty +
\left\| \frac{1}{n} (\bfB_{0,j} - \widehat \bfB_j)^T \bfX^T \bfE \bfT_{0,j} \right\|_\infty \notag\\
&= \| \bfW_1 \|_\infty + \| \bfW_2 \|_\infty + \| \bfW_3 \|_\infty + \| \bfW_4 \|_\infty
\end{align}
%
Now
%
$$
\bfW_1 = \frac{1}{n} \bfE_{-j}^T ( \bfE_j - \bfE_{-j} \bftheta_{0,j})
$$
%
For node $j$ in the $y$-network, $\BE_{-j}$ and $E_j - \BE_{-j} \bftheta_{0,j}$ are the neighborhood regression coefficients and residuals, respectively. Thus they are orthogonal, so we can apply Lemma \ref{lemma:ErrorRElemma2} on $\bfE_{-j}$ and $\bfE_j - \bfE_{-j} \bftheta_{0,j}$ to obtain that for $n \succsim \log (q-1)$,
%
\begin{align}\label{eqn:ErrorRElemma2eqn5}
\| \bfW_1 \|_\infty & \leq c_5 \left[ \Lambda_{\max} ( \Sigma_{y,-j}) \sigma_{y,j,-j} \right]^{1/2} \sqrt{\frac{\log(q-1)}{n}}
\end{align}
%
holds with probability $\geq 1 - 6c_4 \exp [-(c_5^2-1) \log(pq)]$ for some $c_4 > 0, c_5 > 1$.
%

For $\bfW_2$ and $\bfW_4$, identical bounds hold:
%
\begin{align*}
\| \bfW_2 \|_\infty & \leq \left\| \frac{1}{n} \bfE_{-j}^T \bfX (\bfB_0 - \widehat \bfB) \right\|_\infty \| \bfT_{0,j} \|_1 \leq
\left\| \frac{1}{n} \bfE^T \bfX \right\|_\infty \| \bfB_0 - \widehat \bfB \|_1 \| \bfT_{0,j} \|_1\\
\| \bfW_4 \|_\infty & \leq \left\| \frac{1}{n} (\bfB_{0,j} - \widehat \bfB_j)^T \bfX^T \bfE \right\|_\infty \| \bfT_{0,j} \|_1 \leq
\left\| \frac{1}{n} \bfE^T \bfX \right\|_\infty \| \bfB_0 - \widehat \bfB \|_1 \| \bfT_{0,j} \|_1\\
\end{align*}
%
Since $\Omega_y$ is diagonally dominant, $|\omega_{y,jj}| \geq \sum_{j \neq j'} |\omega_{y,jj'}|$ for any $j \in \cI_q$. Hence
%
$$
\| \bfT_{0,j} \|_1 = \sum_{j'=1}^q | T_{jj'} | = 1 + \sum_{j \neq j'} | \theta_{jj'} | = 1 + \frac{1}{\omega_{y,jj}} \sum_{j \neq j'} | \omega_{y,jj'} | \leq 2
$$
%
so that for $n \succsim \log (pq)$,
%
\begin{align}\label{eqn:ErrorRElemma2eqn6}
\| \bfW_2 \|_\infty + \| \bfW_4 \|_\infty  & \leq
4 C_\beta c_2 [ \Lambda_{\max} (\Sigma_x) \Lambda_{\max} (\Sigma_y) ]^{1/2} \frac{ \log(pq)}{n}
\end{align}
%
with probability $\geq 1 - 12 c_1 \exp [-(c_2^2-1) \log(pq)]$ by applying Lemma~\ref{lemma:ErrorRElemma2} and assumption (A1).

Finally for $\bfW_3$, we apply Lemma 8 of \cite{RavikumarEtal11} on the (sub-gaussian) design matrix $\bfX$ to obtain that for sample size
%
\begin{align}\label{eqn:ErrorRElemma2eqn7}
n \geq 512 ( 1 + 4 \Lambda_{\max} (\Sigma_x^k))^4 \max_i (\sigma_{x,ii}^k )^4 \log (4p^{\tau_1})
\end{align}
%
we get that with probability $ \geq 1 - 1/p^{\tau_1-2}, \tau_1 > 2$,
%
$$
\left\| \frac{\bfX^T \bfX}{n} \right\|_\infty \leq \sqrt{ \frac{ \log 4 + \tau_1 \log p}{c_x n}} + \max_i \sigma_{x,ii} = V_x; \quad
c_x = \left[ 128 ( 1 + 4 \Lambda_{\max} (\Sigma_x)  )^2 \max_i (\sigma_{x,ii})^2 \right]^{-1}
$$
%
Thus with the same probability,
%
\begin{align}\label{eqn:ErrorRElemma2eqn4}
\| \bfW_4 \|_\infty \leq \left\| \frac{\bfX^T \bfX}{n} \right\|_\infty \| \widehat \bfB - \bfB_0 \|_1^2 \| \bfT_{0,j} \|_1 
\leq 2 C_\beta^2 V_x \frac{ \log(pq)}{n}
\end{align}
%
We now bound the right hand side of (\ref{eqn:ErrorRElemma2maineqn}) using (\ref{eqn:ErrorRElemma2eqn5}), (\ref{eqn:ErrorRElemma2eqn6}) and (\ref{eqn:ErrorRElemma2eqn4}) to complete the proof, with the leading term of the sample size requirement being $n \succsim \log(pq)$.
\end{proof}

\begin{proof}[Proof of Lemma~\ref{thm:BetaThm}]
The proof follows that of part (I) of Lemma~\ref{thm:ThetaThm}, with a different group norm structure. We only point out the differences.

Putting $\bfbeta = \bfbeta_0$ in (\ref{eqn:EstEqn1}) we get
%
$$
-2 \widehat \bfbeta^T \widehat \bfgamma + \bfbeta^T \widehat \bfGamma \widehat \bfbeta + \lambda_n \sum_{h \in \cH} \| \widehat \bfbeta^{[h]}  \| \leq
-2 \bfbeta_0^T \widehat \bfgamma + \bfbeta_0^T \widehat \bfGamma \bfbeta_0 + \lambda_n \sum_{h \in \cH} \| \bfbeta_0^{[h]}  \|
$$
%
Denote $\bfb = \widehat \bfbeta - \bfbeta_0$. Then we have
%
$$
\bfb^T \widehat \bfGamma \bfb \leq 2 \bfb^T ( \widehat \bfgamma - \widehat \bfGamma \bfbeta_0 ) + \lambda_n
\sum_{h \in \cH} ( \| \bfbeta_0^{[h]} \| - \| \bfbeta_0^{[h]} + \bfb^{[h]} \|)
$$
%
Proceeding similarly as the proof of part (I) of Lemma~\ref{thm:ThetaThm}, with a different deviation bound and choice of $\lambda_n$, we get expressions equivalent to (\ref{eqn:Thm1ProofProp2Eqn3}) and (\ref{eqn:Thm1ProofProp2Eqn2}) respectively:
%
\begin{align}
\bfb^T \widehat \bfGamma \bfb & \leq \frac{3}{2} \sum_{h \in \cH} \| \bfb^{[h]} \| \\
\frac{\psi^*}{3} \| \bfb \|^2 & \leq \lambda_n \sum_{h \in \cH} \| \bfb^{[h]} \| \leq 4 \lambda_n \sqrt{s_\beta} \| \bfb \|
\end{align}
%
Furthermore, $\| \bfb \|_1 \leq \sqrt{ | h_{\max} |} \sum_{h \in \cH} \| \bfb^{[h]} \| $. The bounds in (\ref{eqn:BetaThmEqn1}), (\ref{eqn:BetaThmEqn2}), (\ref{eqn:BetaThmEqn3}) and (\ref{eqn:BetaThmEqn4}) now
follow.

\end{proof}

\begin{proof}[Proof of Lemma~\ref{prop:ThmBetaRE}]
For part 1 it is enough to prove that with $ \widehat \Sigma_x^k := (\bfX^k)^T \bfX^k/n$,
%
\begin{align}\label{eqn:ThmBetaREProofEqn1}
\widehat \bfT_k^2 \otimes \widehat \Sigma_x^k & \sim RE (\psi_*^k, \phi_*^k)
\end{align}
%
with high enough probability. because then we can take $\psi_* = \min_k \psi_*^k, \phi_* = \max_k \phi_*^k$. The proof of (\ref{eqn:ThmBetaREProofEqn1}) follows similar lines of the proof of Proposition 1 in \cite{LinEtal16}, only replacing $\Theta_\epsilon, \widehat \Theta_\epsilon, \bfX$ therein with $(\bfT^k)^2, (\widehat \bfT^k)^2, \bfX^k$, respectively. We omit the details.

Part 2 follows the proof of Proposition 2 in \cite{LinEtal16}.
\end{proof}

\begin{proof}[Proof of Lemma~\ref{lemma:ErrorRElemma1}]
This is same as Lemma 2 in Appendix B of \cite{LinEtal16} and its proof can be found there.
\end{proof}

\begin{proof}[Proof of Lemma~\ref{lemma:ErrorRElemma2}]
This is a part of Lemma 3 of Appendix B in \cite{LinEtal16}, and is proved therein. 
\end{proof}

\begin{proof}[Proof of Lemma~\ref{Lemma:ThmTestingLemma}]
To show \eqref{eqn:ThmTestingProofeq3} we have
%
\begin{align*}
\frac{1}{\sqrt n \widehat s_i}  \widehat \Omega_y^{1/2} \bfE^T \bfR_i =
\frac{1}{\sqrt n \widehat s_i}  (\widehat \Omega_y^{1/2} - \Omega_y^{1/2}) \bfE^T \bfR_i +
\frac{1}{\sqrt n \widehat s_i}  \Omega_y^{1/2} \bfE^T \bfR_i
\end{align*}
%
The second summand is distributed as $\cN_q ({\bf 0}, \bfI)$. For the first summand,
%
\begin{align*}
\frac{1}{\sqrt n}  \left\| (\widehat \Omega_y^{1/2} - \Omega_y^{1/2}) \bfE^T \bfR_i \right\|_\infty & \leq
\frac{1}{\sqrt n}  \left\| \widehat \Omega_y^{1/2} - \Omega_y^{1/2} \right\|_\infty  \left\| \bfE^T \bfR_i \right\|_1 \\
& \leq \sqrt n v_\Omega \frac{1}{n} \left[ \| \bfE^T (\bfX_i -  \bfX_{-i} \bfzeta_i ) \|_1 + \| \bfE^T \bfX_{-i} (\widehat \bfzeta_i - \bfzeta_{0,i} ) \|_1 \right] \\
& \leq \sqrt n v_\Omega \frac{1}{n} \left[ \| \bfE^T \bfX_i \|_\infty + \| \bfE^T \bfX_{-i} \|_\infty
\left\{ \| \bfzeta_i  \|_1 + \| \widehat \bfzeta_i - \bfzeta_i  \|_1 \right\} \right] \notag\\
& \leq \sqrt n v_\Omega \left[ \frac{1}{n} \| \bfE^T \bfX_i \|_\infty + 
\frac{1 + v_\zeta}{n} \| \bfE^T \bfX_{-i} \|_\infty \right] \\
& \leq \sqrt n v_\Omega (2 + v_\zeta) .\frac{1}{n} \| \bfE^T \bfX \|_\infty
\end{align*}
%
because $\Omega_x$ is diagonally dominant implies $\| \bfzeta_i \|_1 = \sum_{i' \neq i} |\omega_{x,ii'}|/ \omega_{x,ii} \leq 1$, and using assumption (C1). Applying Lemma~\ref{lemma:ErrorRElemma2}, the following holds:
%
\begin{align}\label{eqn:ThmTestingProofeq31}
\frac{1}{\sqrt n}  \left\| (\widehat \Omega_y^{1/2} - \Omega_y^{1/2}) \bfE^T \bfR_i \right\|_\infty & \leq v_\Omega (2 + v_\zeta) c_2 [ \Lambda_{\max} (\Sigma_x) \Lambda_{\max} (\Sigma_e) ]^{1/2} \sqrt{ \log(pq)}
\end{align}
%
with probability $ \geq 1 - 6c_1 \exp [-(c_2^2-1) \log(pq)]$ for some $c_1 >0, c_2 > 1$.

On the other hand
%
\begin{align*}
s_i^2 := \frac{1}{n} \left\| \bfX_i - \bfX_{-i}  \bfzeta_i \right\|^2 & \leq 
\widehat s_i^2 + \frac{1}{n} \left\| \bfX_{-i} (\widehat \bfzeta_i - \bfzeta_{0,i} ) \right\|^2
\leq \widehat s_i^2 + \| \widehat \bfzeta_i - \bfzeta_{0,i} \|_1^2 \left\| \frac{1}{n} \bfX_{-i}^T \bfX_{-i} \right\|_\infty
\end{align*}
%
which implies $s_i \leq \widehat s_i + v_\zeta \sqrt{ V_x}$. By applying Lemma 8 of \cite{RavikumarEtal11},
%
\begin{align}\label{eqn:ThmTestingProofeq32}
\left\| \frac{1}{n} \bfX_{-i}^T \bfX_{-i} \right\|_\infty \leq
\left\| \frac{1}{n} \bfX^T \bfX \right\|_\infty \leq V_x
\end{align}
%
with probability $ \geq 1 - 1/p^{\tau_1-2}, \tau_1>2$, and
%
\begin{align}\label{eqn:ThmTestingProofeq33}
n \geq 512 ( 1 + 4 \Lambda_{\max} (\Sigma_{x}))^4 \max_i (\sigma_{x,ii} )^4 \log (4p^{\tau_1})
\end{align}
%
On the other hand, by Chebyshev inequality, for any $\epsilon>0$
%
$$
P\left( | s_i - \sigma_{x,i,-i}| \geq \epsilon \right) \leq \frac{\BV s_i}{\epsilon^2} =
\frac{\kappa_i}{n \epsilon^2}
$$
%
Taking $\epsilon = n^{-1/4}$, we have $s_i \geq \sigma_{x,i,-i} - n^{-1/4}$ with probability $\geq 1 - \kappa_i n^{-1/2}$. Then, for $n$ satisfying \eqref{eqn:ThmTestingProofeq32} and $\sigma_{x,i,-i} - n^{-1/4} >  v_\zeta \sqrt{ V_x}$, we get the bound with the above probability:
%
\begin{align}\label{eqn:ThmTestingProofeq34}
\frac{1}{\widehat s_i} \leq \frac{1}{\sigma_{x,i,-i} - n^{-1/4} - v_\zeta \sqrt{ V_x}}
\end{align}
%
Combining \eqref{eqn:ThmTestingProofeq31} and \eqref{eqn:ThmTestingProofeq34} gives the upper bound for the right hand side of \eqref{eqn:ThmTestingProofeq3} with the requisite probability and sample size conditions.

To prove \eqref{eqn:ThmTestingProofeq4} we have
%
\begin{align}\label{eqn:ThmTestingProofeq41}
\frac{1}{n} \| \bfR_i^T \bfX_{-i} \|_\infty & \leq
\frac{1}{n} \| (\bfX_i - \bfX_{-i} \bfzeta_{0,i})^T \bfX_{-i} \|_\infty +
\frac{1}{n} \| \bfX_{-i}^T \bfX_{-i} ( \widehat \bfzeta_i - \bfzeta_{0,i}) \|_\infty
\end{align}
%
Applying Lemma~\ref{lemma:ErrorRElemma2}, for $n \succsim \log(p-1)$ we have
%
%
\begin{align}
\frac{1}{n} \| (\bfX_i -  \bfX_{-i} \bfzeta_i )^T \bfX_{-i} \|_\infty \leq 
c_{7} [ \sigma_{x,i,-i} \Lambda_{\max} (\Sigma_{x, -i}) ]^{1/2} \sqrt{\frac{ \log (p-1)}{n}}
\end{align}
%
with probability $\geq 1 - 6c_{6} \exp [-(c_{7}^2-1) \log (p-1)]$ for some $c_{6} >0, c_{7} > 1$. By \eqref{eqn:ThmTestingProofeq32}, the second term on the right side of \eqref{eqn:ThmTestingProofeq41} is bounded above by $v_\zeta V_x$ with probability $ \geq 1 - 1/p^{\tau_1-2}$ and $n$ satisfying \eqref{eqn:ThmTestingProofeq33}. The bound of \eqref{eqn:ThmTestingProofeq4} now follows by conditions (C2), (C3) and \eqref{eqn:ThmTestingProofeq34}.
\end{proof}

%-----------------------------
%\hrulefill
%-----------------------------
%\section{Numerical performance}

\end{document}