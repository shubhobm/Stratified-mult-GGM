\section{Proofs of auxiliary results}

\begin{proof}[Proof of Lemma~\ref{thm:ThetaThm}]
The proof has the same structure as the proof of Theorem 1 in \citet{MaMichailidis15}, where consistency of the (single layer) JSEM estimates are established. Part (I) is analogous to part A.1 therein, but the proof strategy is completely different, which we provide in detail next. Our part (II) follows along similar lines as parts A.2 and A.3, incorporating the updated quantities from the first part (A.1). For this part of the proof, we provide an outline and leave details to the reader.

\paragraph{Proof of part (I).}
%The following proposition establishes error bounds for estimated neighborhood coefficients in the Y-network.
%
%\begin{Proposition}\label{prop:Thm1ProofProp2}
%Consider the estimation problem in (\ref{eqn:EstEqn2}) and take $\gamma_n \geq 4 \sqrt{| g_{\max}|} \BQ_0$. Given the conditions (T2) and (T3) hold, for any solution of (\ref{eqn:EstEqn2}) we shall have
%%
%\begin{align}
%\| \widehat \Theta_j - \Theta_{0,j} \|_F & \leq 12 \sqrt{s_j} \gamma_n / \psi \label{eqn:Thm1ProofProp2Bd1}\\
%\sum_{j \neq j', g \in \cG_y^{jj'}} \| \hat \bftheta_{jj'}^{[g]} - \bftheta_{0,jj'}^{[g]} \| & \leq 48 s_j \gamma_n / \psi \label{eqn:Thm1ProofProp2Bd2}
%\end{align}
%%
%Also denote the non-zero support of $\widehat \Theta_j$ by $\widehat \cS_j$, i.e. $\widehat \cS_j = \{ (j',g): \hat \bftheta_{jj'}^{[g]} \neq {\bf 0} \}$. Then
%%
%\begin{align}
%| \widehat \cS_j| \leq 128 s_j / \psi \label{eqn:Thm1ProofProp2Bd3}
%\end{align}
%\end{Proposition}
In its reparametrized version, (\ref{eqn:EstEqn2}) becomes
%
\begin{align}
\widehat \bfT_j = \argmin_{\bfT_j} \left\{ \frac{1}{n} \sum_{k=1}^K \| (\bfY^k - \bfX^k \widehat \bfB^k) \bfT_j^k \|^2 + \gamma_n \sum_{j \neq j', g \in \cG_y^{jj'}} \| \bfT_{jj'}^{[g]} \| \right\}
\end{align}
%
with $\bfT_{jj'}^{[g]} := (T_{jj'}^k)_{k \in g}$. Now for any $\bfT_j \in \BM(q, K)$ we have
%
$$
\frac{1}{n} \sum_{k=1}^K \| (\bfY^k - \bfX^k \widehat \bfB^k) \widehat \bfT_j^k \|^2 + \gamma_n \sum_{j \neq j', g \in \cG_y^{jj'}} \| \widehat \bfT_{jj'}^{[g]} \| \leq
\frac{1}{n} \sum_{k=1}^K \| (\bfY^k - \bfX^k \widehat \bfB^k) \bfT_j^k \|^2 + \gamma_n \sum_{j \neq j', g \in \cG_y^{jj'}} \| \bfT_{jj'}^{[g]} \|
$$
%
For $\bfT_j = \bfT_{0,j}$ this reduces to
%
\begin{align}\label{eqn::Thm1ProofProp2Eqn1}
\sum_{k=1}^K (\bfd_j^k)^T \widehat \bfS^k \bfd_j^k & \leq - 2 \sum_{k=1}^K (\bfd_j^k)^T \widehat \bfS^k \bfT_{0,j}^k + \gamma_n \sum_{j \neq j', g \in \cG_y^{jj'}} \left( \| \bfT_{jj'}^{[g]} \| -  \| \bfT_{jj'}^{[g]} + \bfd_{jj'}^{[g]}\| \right)
\end{align}
%
with $\bfd_{j}^k := \widehat \bfT_j^k - \bfT_{0,j}^k$ etc. For the $k^\text{th}$ summand in the first term on the right hand side, since $d_{jj}^k = 0$, $\widehat \bfE^k \bfd_j^k = \widehat \bfE_{-j}^k \bfd_{-j}^k$. Thus
%
\begin{align*}
\sum_{k=1}^K \left| (\bfd_j^k)^T \widehat \bfS^k \bfT_{0,j}^k \right| &=
\sum_{k=1}^K \left| \bfd_j^k. \frac{1}{n} (\widehat \bfE^k)^T \widehat \bfE^k \bfT_{0,j}^k \right| \\
& \leq \sum_{k=1}^K \left\| \frac{1}{n} (\widehat \bfE_{-j}^k)^T \widehat \bfE^k \bfT_{0,j}^k \right\|_\infty \| \bfd_{-j}^k \|_1 \\
& \leq \left[ \sum_{j \neq j', g \in \cG_y^{jj'}} \| \bfd_{jj'}^{[g]} \| \right]
\BQ_0 \sqrt {| g_{\max}|} \sqrt{ \frac{\log (pq)}{n}}
\end{align*}
%
by assumption (A2). For the second term, suppose $\cS_{0,j}$ is the support of $\Theta_{0,j}$, i.e. $\cS_{0,j} = \{ (j',g): \bftheta_{jj'}^{[g]} \neq 0 \}$. Then
%
\begin{align*}
\sum_{j \neq j', g \in \cG_y^{jj'}} \left( \| \bfT_{jj'}^{[g]} \| -  \| \bfT_{jj'}^{[g]} + \bfd_{jj'}^{[g]}\| \right) & \leq
\sum_{(j',g) \in \cS_{0,j}} \left( \| \bfT_{jj'}^{[g]} \| -  \| \bfT_{jj'}^{[g]} + \bfd_{jj'}^{[g]}\| \right) -
\sum_{(j',g) \notin \cS_{0,j}} \| \bfd_{jj'}^{[g]} \|\\
& \leq \sum_{(j',g) \in \cS_{0,j}} \| \bfd_{jj'}^{[g]} \| - \sum_{(j',g) \notin \cS_{0,j}} \| \bfd_{jj'}^{[g]} \|
\end{align*}
%
so that by choice of $\gamma_n$, (\ref{eqn::Thm1ProofProp2Eqn1}) reduces to
%
\begin{align}\label{eqn:Thm1ProofProp2Eqn3}
\sum_{k=1}^K (\bfd_j^k)^T \widehat \bfS^k \bfd_j^k & \leq 
\frac{\gamma_n}{2} \left[ \sum_{(j',g) \in \cS_{0,j}} \| \bfd_{jj'}^{[g]} \| + \sum_{(j',g) \notin \cS_{0,j}} \| \bfd_{jj'}^{[g]} \| \right] +
\gamma_n \left[ \sum_{(j',g) \in \cS_{0,j}} \| \bfd_{jj'}^{[g]} \| - \sum_{(j',g) \notin \cS_{0,j}} \| \bfd_{jj'}^{[g]} \| \right] \notag\\
& = \frac{3 \gamma_n}{2} \sum_{(j',g) \in \cS_{0,j}} \| \bfd_{jj'}^{[g]} \| - \frac{\gamma_n}{2} \sum_{(j',g) \notin \cS_{0,j}} \| \bfd_{jj'}^{[g]} \| \notag\\
& \leq \frac{3 \gamma_n}{2} \sum_{j \neq j', g \in \cG_y^{jj'}} \| \bfd_{jj'}^{[g]} \|
\end{align}
%
Since the left hand side is $\geq 0$, this also implies
%
$$
\sum_{(j',g) \notin \cS_{0,j}} \| \bfd_{jj'}^{[g]} \| \leq 3 \sum_{(j',g) \in \cS_{0,j}} \| \bfd_{jj'}^{[g]} \| \quad \Rightarrow
\sum_{j \neq j', g \in \cG_y^{jj'}} \| \bfd_{jj'}^{[g]} \| \leq
4 \sum_{(j',g) \in \cS_{0,j}} \| \bfd_{jj'}^{[g]} \| \leq 4 \sqrt{s_j} \| \bfD_j \|_F
$$
% 
with $\bfD_j = (\bfd_j^1, \ldots, \bfd_j^K)$. Now the RE condition on $\widehat \bfS^k$ means that
%
$$
\sum_{k=1}^K (\bfd_j^k)^T \widehat \bfS^k \bfd_j^k \geq 
\sum_{k=1}^K \left( \psi_k \| \bfd_j^k \|^2 - \phi_k \| \bfd_j^k \|_1^2 \right) \geq
\psi \| \bfD_j \|_F^2 - \phi \| \bfD_j \|_1^2 \geq 
(\psi - Kq \phi ) \| \bfD_j \|_F^2 \geq \frac{\psi}{2}  \| \bfD_j \|_F^2
$$
%
by assumption (A3).

Combining the above with \eqref{eqn:Thm1ProofProp2Eqn3}, we finally have
%
\begin{align}\label{eqn:Thm1ProofProp2Eqn2}
\frac{\psi}{3} \| \bfD_j \|_F^2 & \leq
\gamma_n \sum_{j \neq j', g \in \cG_y^{jj'}} \| \bfd_{jj'}^{[g]} \| \leq
4 \gamma_n \sqrt{s_j} \| \bfD_j\|_F
\end{align}
%
Since
%
$$
(\bfD_j)_{j',k} = \widehat T_{jj'}^k - T_{0,jj'}^k = \begin{cases}
0 \text{ if } j = j'\\
-(\widehat \theta_{jj'}^k - \theta_{0,jj'}^k ) \text{ if } j \neq j'
\end{cases}
$$
%
The bounds in \eqref{eqn:theta-norm-bound-1a} and \eqref{eqn:theta-norm-bound-2a} are obtained by replacing the corresponding elements in (\ref{eqn:Thm1ProofProp2Eqn2}).

For the bound on $| \widehat \cS_j| := |\supp( \widehat \Theta_j)|$, notice that if $\hat \bftheta_{jj'}^{[g]} \neq 0$ for some $(j',g)$,
%
\begin{align*}
\frac{1}{n} \sum_{k \in g} \left| ((\widehat \bfE_{-j}^k)^T \widehat \bfE^k ( \widehat \bfT_j^k - \bfT_{0,j}^k ))^{j'} \right| & \geq
\frac{1}{n} \sum_{k \in g} \left| ((\widehat \bfE_{-j}^k)^T \widehat \bfE^k \widehat \bfT_j^k )^{j'} \right| - \frac{1}{n} \sum_{k \in g} \left| ((\widehat \bfE_{-j}^k)^T \widehat \bfE^k \bfT_{0,j}^k )^{j'} \right|\\
& \geq |g| \gamma_n - \sum_{k \in g} \BQ ( C_\beta, \Sigma_x^k, \Sigma_y^k ) \sqrt{ \frac{ \log (pq)}{n}}
\end{align*}
%
using the KKT condition for (\ref{eqn:EstEqn2}) and assumption (A2). The choice of $\gamma_n$ now ensures that the right hand side is $\geq 3|g| \gamma_n / 4$. Hence,
%
\begin{align*}
| \hat \cS_j| & \leq \sum_{(j',g) \in \widehat \cS_j} \frac{16}{9 n^2 |g|^2 \gamma_n^2 } \sum_{k \in g} \left| ((\widehat \bfE_{-j}^k)^T \widehat \bfE^k ( \widehat \bfT_j^k - \bfT_{0,j}^k ))^{j'} \right|^2\\
& \leq \frac{16}{9 \gamma_n^2} \sum_{k=1}^K \frac{1}{n} \left\| (\widehat \bfE_{-j}^k)^T \widehat \bfE^k ( \widehat \bfT_j^k - \bfT_{0,j}^k ) \right\|^2 \\
& = \frac{16}{9 \gamma_n^2} \sum_{k=1}^K (\bfd_j^k)^T \widehat \bfS^k \bfd_j^k \\
& \leq \frac{8}{3 \gamma_n} \sum_{j \neq j', g \in \cG_y^{jj'}} \| \bfd_{jj'}^{[g]} \| \leq \frac{128 s_j}{\psi} 
\end{align*}
%
using (\ref{eqn:Thm1ProofProp2Eqn3}) and (\ref{eqn:Thm1ProofProp2Eqn2}).

\paragraph{Proof of part (II).}
We denote the selected edge set for the $k^\text{th}$ Y-network by $\hat E^k$. Denote its population version by $E_0^k$. Further, let
%
$$
\tilde \Omega_y^k = \diag (\Omega_{y0}^k) + \Omega_{y, E_0^k \cap \hat E^k}^k
$$
%
Based on similar derivations as in the proof of Corollary A.1 in \citet{MaMichailidis15}, the following two upper bounds can be established:
%
\begin{align}
| \hat E^k | \leq \frac{ 128 S }{\psi} \label{eqn:Thm1ProofProp2Bd4}\\
\frac{1}{K} \sum_{k=1}^K \| \tilde \Omega_y^k - \Omega_{y0}^k \|_F \leq
\frac{12 c_y \sqrt{S} \gamma_n} {\sqrt K \psi} \label{eqn:Thm1ProofProp2Bd5}
\end{align}
%
following which, taking $\gamma_n = 4 \sqrt{| g_{\max}|} \BQ_0 \sqrt{ \log (pq)/ n}$,
%
\begin{align}
\Lambda_{\min} ( \tilde \Omega_y^k) \geq d_y - \frac{48 c_y \BQ_0 \sqrt{| g_{\max}| S}}{ \psi}
\sqrt{ \frac{ \log (pq)}{n}} \geq (1 - t_1) d_y > 0 \label{eqn:Thm1ProofProp2Bd6}\\
\Lambda_{\max} ( \tilde \Omega_y^k) \leq c_y + \frac{48 c_y \BQ_0 \sqrt{| g_{\max}| S}}{ \psi}
\sqrt{ \frac{ \log (pq)}{n}} \leq c_y + t_1 d_y < \infty \label{eqn:Thm1ProofProp2Bd7}
\end{align}
%
with $0 < t_1 < 1$, and the sample size $n$ satisfying
%
$$
n \geq | g_{\max}| S \left[ \frac{48 c_y \BQ_0}{\psi t_1 d_y} \right]^2 \log (pq).
$$

Following the same steps as part A.3 in the proof of Theorem 4.1 in \citet{MaMichailidis15}, it can be proven using \eqref{eqn:Thm1ProofProp2Bd4}--\eqref{eqn:Thm1ProofProp2Bd7} that
%
$$
\sum_{k=1}^K \left\| \widehat \Omega_y^k - \tilde \Omega_y^k \right\|_F^2 \leq
O \left( \BQ_0^2 | g_{\max}| S \frac{ \log (pq)}{n} \right)
$$
%
The proof is now complete by combining this with \eqref{eqn:Thm1ProofProp2Bd5}  and then applying the Cauchy-Schwarz inequality and the triangle inequality.
\end{proof}

\begin{proof}[Proof of Lemma~\ref{prop:ErrorRE}]

%For any sub-gaussian zero-mean design matrix $\bfX \in \BM(n,p)$ with parameters $(\Sigma_x, \sigma_x^2)$, and any $\hat \bfB, \bfB_0 \in \BM(p,q)$ such that $\| \hat \bfB - \bfB_0 \|_F \leq v_\beta$, we follow the proof of Proposition 3 in \citet{LinEtal16} to obtain that for sample size
%%
%\begin{align}\label{eqn:ErrorREeqn1}
%n & \geq 512 ( 1 + 4 \sigma_x^2)^4 \max_j (\Sigma_{x,jj})^4 \log (4p^{\tau_1})
%\end{align}
%the following holds
%%
%\begin{align}\label{eqn:ErrorREeqn2}
%\left\| (\widehat \bfB - \bfB_0)^T \left( \frac{\bfX^T \bfX}{n} \right) (\widehat \bfB - \bfB_0) \right\|_\infty & \leq
%v_\beta^2 \left[ \sqrt{ \frac{ \log 4 + \tau_1 \log p}{c_x n}} + \max_j \Sigma_{x,jj} \right]
%\end{align}
%%
%with probability $\geq 1 - 1/p^{\tau_1-2}$ for some $\tau_1>2$, where
%%
%$$
%c_x = \left[ 128 ( 1 + 4 \sigma_x^2)^2 \max_j (\Sigma_{x,jj})^2 \right]^{-1}
%$$
%%
%Here we substitute $\bfX, \hat \bfB, \bfB_0$ with $\bfX^k, \hat \bfB^k, \hat \bfB_0^k$ respectively. Since rows of $\bfX^k$ come independently from $\cN( {\bf 0}, \Sigma_x^k)$, $\sigma_x^2$ in our case is the spectral norm of $\Sigma_x^k$ \citetp{LohWainwright12}, which is $\Lambda_{\max} (\Sigma_x^k)$. Finally
%%
%$$
%\| \bfX^k ( \hat \bfB^k - \bfB_0^k ) \|_\infty \leq 
%\sqrt{\left\| (\widehat \bfB^k - \bfB_0^k)^T (\bfX^k)^T \bfX^k (\widehat \bfB^k - \bfB_0^k) \right\|_\infty}
%$$
%%
%The expression of part 1 is immediate now, and (\ref{eqn:ErrorREeqn1}) ensures that part 1 holds when the leading term of the sample size requirement is $n \succsim \log (pq)$.

We drop the subscript 0 for true values and the superscript $k$ since there is no scope of ambiguity. For part 1, we start with an auxiliary lemma:
%
\begin{lemma}\label{lemma:ErrorRElemma1}
For a sub-Gaussian design matrix $\bfX \in \BM(n,p)$ with columns having mean ${\bf 0}_p$ and covariance matrix $\Sigma_x$, the sample covariance matrix $\widehat \Sigma_x = \bfX^T \bfX/n$ satisfies the RE condition
%
$$
\widehat \Sigma_x \sim RE \left( \frac{\Lambda_{\min} ( \Sigma_x) }{2}, \frac{\Lambda_{\min} ( \Sigma_x) \log p }{2 n} \right)
$$
%
with probability $\geq 1 - 2 \exp(-c_3 n)$ for some $c_3 > 0$.
\end{lemma}
%
Denote $\widehat \bfE = \bfY - \bfX \widehat \bfB$. For $\bfv \in \BR^q$, we have
%
\begin{align}\label{eqn:ErrorREeqn3}
\bfv^T \widehat \bfS \bfv &= \frac{1}{n} \| \widehat \bfE \bfv \|^2 \notag\\
&= \frac{1}{n} \| (\bfE + \bfX ( \bfB_0 - \widehat \bfB ))\bfv \|^2 \notag\\
&= \bfv^T \bfS \bfv + \frac{1}{n} \| \bfX (\bfB_0 - \widehat \bfB) \bfv \|^2 + 2 \bfv^T (\bfB_0 - \widehat \bfB)^T \left( \frac {(\bfX)^T \bfE}{n} \right) \bfv
\end{align}
%
For the first summand, $ \bfv^T \bfS^k \bfv \geq \psi_y \| \bfv \|^2 - \phi_y \| \bfv \|_1^2$ with $\psi_y = \Lambda_{\min} (\Sigma_y)/2, \phi_y = \psi_y \log p/n$ by applying Lemma \ref{lemma:ErrorRElemma1} on $\bfS$. The second summand is greater than or equal to 0. For the third summand,
%
$$
2 \bfv^T (\bfB_0 - \widehat \bfB)^T \left( \frac {(\bfX)^T \bfE}{n} \right) \bfv \geq
-2 C_\beta \left\| \frac {(\bfX)^T \bfE}{n} \right\|_\infty \| \bfv \|_1^2
\sqrt{ \frac{ \log (pq)}{n}}
$$
%
by assumption (A1). Now, we use another lemma:
%
\begin{lemma}\label{lemma:ErrorRElemma2}
For zero-mean independent sub-gaussian matrices $\bfX \in \BM(n,p), \bfE \in \BM(n,q)$ with parameters $(\Sigma_x, \sigma_x^2)$ and $(\Sigma_e, \sigma_e^2)$ respectively, given that $n \succsim \log(pq)$ the following holds with probability $\geq 1 - 6c_1 \exp [-(c_2^2-1) \log(pq)]$ for some $c_1 >0, c_2 > 1$:
%
$$
\frac{1}{n} \| \bfX^T \bfE \|_\infty \leq c_2 [ \Lambda_{\max} (\Sigma_x) \Lambda_{\max} (\Sigma_e) ]^{1/2} \sqrt{\frac{ \log(pq)}{n}}
$$
%
\end{lemma}
%
Subsequently we collect all summands in (\ref{eqn:ErrorREeqn3}) and get
%
$$
\bfv^T \widehat{ \bfS} \bfv \geq \psi_y \| \bfv \|^2 - \left( \phi_y + 2 C_\beta c_2 [ \Lambda_{\max} (\Sigma_x) \Lambda_{\max} (\Sigma_y) ]^{1/2} \frac{ \log(pq)}{n} \right) \| \bfv \|_1^2
$$
with probability $\geq 1 - 2\exp(- c_3 n) - 6c_1 \exp [-(c_2^2-1) \log(pq)]$. This concludes the proof of part 1.

To prove part 2, we decompose the quantity in question:
%
\begin{align}\label{eqn:ErrorRElemma2maineqn}
\left\| \frac{1}{n} \widehat \bfE_{-j}^T \widehat \bfE \bfT_{0,j} \right\|_\infty &=
\left\| \frac{1}{n} \left[ \bfE_{-j} + \bfX (\bfB_{0,j} - \widehat \bfB_j) \right]^T \left[ \bfE + \bfX (\bfB_0 - \widehat \bfB) \right] \bfT_{0,j} \right\|_\infty \notag\\
& \leq \left\| \frac{1}{n} \bfE_{-j}^T \bfE \bfT_{0,j} \right\|_\infty +
\left\| \frac{1}{n} \bfE_{-j}^T \bfX (\bfB_0 - \widehat \bfB) \bfT_{0,j} \right\|_\infty \notag\\
& + \left\| \frac{1}{n} (\bfB_{0,j} - \widehat \bfB_j)^T \bfX^T \bfX (\bfB_0 - \widehat \bfB) \bfT_{0,j} \right\|_\infty +
\left\| \frac{1}{n} (\bfB_{0,j} - \widehat \bfB_j)^T \bfX^T \bfE \bfT_{0,j} \right\|_\infty \notag\\
&= \| \bfW_1 \|_\infty + \| \bfW_2 \|_\infty + \| \bfW_3 \|_\infty + \| \bfW_4 \|_\infty
\end{align}
%
Now
%
$$
\bfW_1 = \frac{1}{n} \bfE_{-j}^T ( \bfE_j - \bfE_{-j} \bftheta_{0,j})
$$
%
For node $j$ in the $y$-network, $\BE_{-j}$ and $E_j - \BE_{-j} \bftheta_{0,j}$ are the neighborhood regression coefficients and residuals, respectively. Thus they are orthogonal, so we can apply Lemma \ref{lemma:ErrorRElemma2} on $\bfE_{-j}$ and $\bfE_j - \bfE_{-j} \bftheta_{0,j}$ to obtain that for $n \succsim \log (q-1)$,
%
\begin{align}\label{eqn:ErrorRElemma2eqn5}
\| \bfW_1 \|_\infty & \leq c_5 \left[ \Lambda_{\max} ( \Sigma_{y,-j}) \sigma_{y,j,-j} \right]^{1/2} \sqrt{\frac{\log(q-1)}{n}}
\end{align}
%
holds with probability $\geq 1 - 6c_4 \exp [-(c_5^2-1) \log(pq)]$ for some $c_4 > 0, c_5 > 1$.
%

For $\bfW_2$ and $\bfW_4$, identical bounds hold:
%
\begin{align*}
\| \bfW_2 \|_\infty & \leq \left\| \frac{1}{n} \bfE_{-j}^T \bfX (\bfB_0 - \widehat \bfB) \right\|_\infty \| \bfT_{0,j} \|_1 \leq
\left\| \frac{1}{n} \bfE^T \bfX \right\|_\infty \| \bfB_0 - \widehat \bfB \|_1 \| \bfT_{0,j} \|_1\\
\| \bfW_4 \|_\infty & \leq \left\| \frac{1}{n} (\bfB_{0,j} - \widehat \bfB_j)^T \bfX^T \bfE \right\|_\infty \| \bfT_{0,j} \|_1 \leq
\left\| \frac{1}{n} \bfE^T \bfX \right\|_\infty \| \bfB_0 - \widehat \bfB \|_1 \| \bfT_{0,j} \|_1\\
\end{align*}
%
Since $\Omega_y$ is diagonally dominant, $|\omega_{y,jj}| \geq \sum_{j \neq j'} |\omega_{y,jj'}|$ for any $j \in \cI_q$. Hence
%
$$
\| \bfT_{0,j} \|_1 = \sum_{j'=1}^q | T_{jj'} | = 1 + \sum_{j \neq j'} | \theta_{jj'} | = 1 + \frac{1}{\omega_{y,jj}} \sum_{j \neq j'} | \omega_{y,jj'} | \leq 2
$$
%
so that for $n \succsim \log (pq)$,
%
\begin{align}\label{eqn:ErrorRElemma2eqn6}
\| \bfW_2 \|_\infty + \| \bfW_4 \|_\infty  & \leq
4 C_\beta c_2 [ \Lambda_{\max} (\Sigma_x) \Lambda_{\max} (\Sigma_y) ]^{1/2} \frac{ \log(pq)}{n}
\end{align}
%
with probability $\geq 1 - 12 c_1 \exp [-(c_2^2-1) \log(pq)]$ by applying Lemma~\ref{lemma:ErrorRElemma2} and assumption (A1).

Finally, for $\bfW_3$, we apply Lemma 8 of \citet{RavikumarEtal11} on the (sub-gaussian) design matrix $\bfX$ to obtain that for sample size
%
\begin{align}\label{eqn:ErrorRElemma2eqn7}
n \geq 512 ( 1 + 4 \Lambda_{\max} (\Sigma_x^k))^4 \max_i (\sigma_{x,ii}^k )^4 \log (4p^{\tau_1})
\end{align}
%
we get that with probability $ \geq 1 - 1/p^{\tau_1-2}, \tau_1 > 2$,
%
$$
\left\| \frac{\bfX^T \bfX}{n} \right\|_\infty \leq \sqrt{ \frac{ \log 4 + \tau_1 \log p}{c_x n}} + \max_i \sigma_{x,ii} = V_x; \quad
c_x = \left[ 128 ( 1 + 4 \Lambda_{\max} (\Sigma_x)  )^2 \max_i (\sigma_{x,ii})^2 \right]^{-1}
$$
%
Thus, with the same probability,
%
\begin{align}\label{eqn:ErrorRElemma2eqn4}
\| \bfW_4 \|_\infty \leq \left\| \frac{\bfX^T \bfX}{n} \right\|_\infty \| \widehat \bfB - \bfB_0 \|_1^2 \| \bfT_{0,j} \|_1 
\leq 2 C_\beta^2 V_x \frac{ \log(pq)}{n}
\end{align}
%
We now bound the right hand side of (\ref{eqn:ErrorRElemma2maineqn}) using (\ref{eqn:ErrorRElemma2eqn5}), (\ref{eqn:ErrorRElemma2eqn6}) and (\ref{eqn:ErrorRElemma2eqn4}) to complete the proof, with the leading term of the sample size requirement being $n \succsim \log(pq)$.
\end{proof}

\begin{proof}[Proof of Lemma~\ref{thm:BetaThm}]
The proof follows that of part (I) of Lemma~\ref{thm:ThetaThm}, with a different group norm structure. We only point out the differences.

Putting $\bfbeta = \bfbeta_0$ in (\ref{eqn:EstEqn1}) we get
%
$$
-2 \widehat \bfbeta^T \widehat \bfgamma + \bfbeta^T \widehat \bfGamma \widehat \bfbeta + \lambda_n \sum_{h \in \cH} \| \widehat \bfbeta^{[h]}  \| \leq
-2 \bfbeta_0^T \widehat \bfgamma + \bfbeta_0^T \widehat \bfGamma \bfbeta_0 + \lambda_n \sum_{h \in \cH} \| \bfbeta_0^{[h]}  \|
$$
%
Denote $\bfb = \widehat \bfbeta - \bfbeta_0$. Then we have
%
$$
\bfb^T \widehat \bfGamma \bfb \leq 2 \bfb^T ( \widehat \bfgamma - \widehat \bfGamma \bfbeta_0 ) + \lambda_n
\sum_{h \in \cH} ( \| \bfbeta_0^{[h]} \| - \| \bfbeta_0^{[h]} + \bfb^{[h]} \|)
$$
%
Proceeding similarly as the proof of part (I) of Lemma~\ref{thm:ThetaThm}, with a different deviation bound and choice of $\lambda_n$, we get expressions equivalent to (\ref{eqn:Thm1ProofProp2Eqn3}) and (\ref{eqn:Thm1ProofProp2Eqn2}) respectively:
%
\begin{align}
\bfb^T \widehat \bfGamma \bfb & \leq \frac{3}{2} \sum_{h \in \cH} \| \bfb^{[h]} \| \\
\frac{\psi^*}{3} \| \bfb \|^2 & \leq \lambda_n \sum_{h \in \cH} \| \bfb^{[h]} \| \leq 4 \lambda_n \sqrt{B} \| \bfb \|
\end{align}
%
Furthermore, $\| \bfb \|_1 \leq \sqrt{ | h_{\max} |} \sum_{h \in \cH} \| \bfb^{[h]} \| $. The bounds in (\ref{eqn:BetaThmEqn1}), (\ref{eqn:BetaThmEqn2}), (\ref{eqn:BetaThmEqn3}) and (\ref{eqn:BetaThmEqn4}) now
follow.

\end{proof}

\begin{proof}[Proof of Lemma~\ref{prop:ThmBetaRE}]
For part 1 it is enough to prove that with $ \widehat \Sigma_x^k := (\bfX^k)^T \bfX^k/n$,
%
\begin{align}\label{eqn:ThmBetaREProofEqn1}
\widehat \bfT_k^2 \otimes \widehat \Sigma_x^k & \sim RE (\psi_*^k, \phi_*^k)
\end{align}
%
with high enough probability. because then we can take $\psi_* = \min_k \psi_*^k, \phi_* = \max_k \phi_*^k$. The proof of (\ref{eqn:ThmBetaREProofEqn1}) follows similar lines of the proof of Proposition 1 in \citet{LinEtal16}, only replacing $\Theta_\epsilon, \widehat \Theta_\epsilon, \bfX$ therein with $(\bfT^k)^2, (\widehat \bfT^k)^2, \bfX^k$, respectively. We omit the details.

Part 2 follows the proof of Proposition 2 in \citet{LinEtal16}.
\end{proof}

\begin{proof}[Proof of Lemma~\ref{Lemma:ThmTestingLemma}]
To show \eqref{eqn:ThmTestingProofeq3} we have
%
\begin{align*}
\frac{1}{\sqrt n \widehat s_i}  \widehat \Omega_y^{1/2} \bfE^T \bfR_i =
\frac{1}{\sqrt n \widehat s_i}  (\widehat \Omega_y^{1/2} - \Omega_y^{1/2}) \bfE^T \bfR_i +
\frac{1}{\sqrt n \widehat s_i}  \Omega_y^{1/2} \bfE^T \bfR_i
\end{align*}
%
The second summand is distributed as $\cN_q ({\bf 0}, \bfI)$. For the first summand,
%
\begin{align*}
\frac{1}{\sqrt n}  \left\| (\widehat \Omega_y^{1/2} - \Omega_y^{1/2}) \bfE^T \bfR_i \right\|_\infty & \leq
\frac{1}{\sqrt n}  \left\| \widehat \Omega_y^{1/2} - \Omega_y^{1/2} \right\|_\infty  \left\| \bfE^T \bfR_i \right\|_1 \\
& \leq \sqrt{n D_\Omega} \frac{1}{n} \left[ \| \bfE^T (\bfX_i -  \bfX_{-i} \bfzeta_i ) \|_1 + \| \bfE^T \bfX_{-i} (\widehat \bfzeta_i - \bfzeta_{i} ) \|_1 \right] \\
& \leq \sqrt{n D_\Omega} \frac{1}{n} \left[ \| \bfE^T \bfX_i \|_\infty + \| \bfE^T \bfX_{-i} \|_\infty
\left\{ \| \bfzeta_i  \|_1 + \| \widehat \bfzeta_i - \bfzeta_i  \|_1 \right\} \right] \notag\\
& \leq \sqrt{n D_\Omega} \left[ \frac{1}{n} \| \bfE^T \bfX_i \|_\infty + 
\frac{1 + D_\zeta}{n} \| \bfE^T \bfX_{-i} \|_\infty \right] \\
& \leq \sqrt{n D_\Omega} (2 + D_\zeta) .\frac{1}{n} \| \bfE^T \bfX \|_\infty
\end{align*}
%
because $\Omega_x$ is diagonally dominant implies $\| \bfzeta_i \|_1 = \sum_{i' \neq i} |\omega_{x,ii'}|/ \omega_{x,ii} \leq 1$, and using assumption (T1) and \eqref{eqn:omega-sqrt-bound}. Applying Lemma~\ref{lemma:ErrorRElemma2}, the following holds for $n \succsim \log (pq)$:
%
\begin{align}\label{eqn:ThmTestingProofeq31}
\frac{1}{\sqrt n}  \left\| (\widehat \Omega_y^{1/2} - \Omega_y^{1/2}) \bfE^T \bfR_i \right\|_\infty & \leq \sqrt{ D_\Omega} (2 + D_\zeta) c_2 [ \Lambda_{\max} (\Sigma_x) \Lambda_{\max} (\Sigma_e) ]^{1/2} \sqrt{ \log(pq)}
\end{align}
%
with probability $ \geq 1 - 6c_1 \exp [-(c_2^2-1) \log(pq)]$.

On the other hand,
%
\begin{align*}
s_i^2 := \frac{1}{n} \left\| \bfX_i - \bfX_{-i}  \bfzeta_{0,i} \right\|^2 & \leq 
\widehat s_i^2 + \frac{1}{n} \left\| \bfX_{-i} (\widehat \bfzeta_i - \bfzeta_{0,i} ) \right\|^2
\leq \widehat s_i^2 + \| \widehat \bfzeta_i - \bfzeta_{0i} \|_1^2 \left\| \frac{1}{n} \bfX_{-i}^T \bfX_{-i} \right\|_\infty
\end{align*}
%
which implies $s_i \leq \widehat s_i + D_\zeta \sqrt{ V_x}$. By applying Lemma 8 of \citet{RavikumarEtal11},
%
\begin{align}\label{eqn:ThmTestingProofeq32}
\left\| \frac{1}{n} \bfX_{-i}^T \bfX_{-i} \right\|_\infty \leq
\left\| \frac{1}{n} \bfX^T \bfX \right\|_\infty \leq V_x
\end{align}
%
with probability $ \geq 1 - 1/p^{\tau_1-2}, \tau_1>2$, and
%
\begin{align}\label{eqn:ThmTestingProofeq33}
n \geq 512 ( 1 + 4 \Lambda_{\max} (\Sigma_{x}))^4 \max_i (\sigma_{x,ii} )^4 \log (4p^{\tau_1})
\end{align}
%
On the other hand, by Chebyshev's inequality, for any $\epsilon>0$
%
$$
P\left( | s_i - \sqrt{ \sigma_{x,i,-i}} | \geq \epsilon \right) \leq \frac{Var (s_i)}{\epsilon^2} =
\frac{\kappa_i}{n \epsilon^2}
$$
%
Taking $\epsilon = n^{-1/4}$, we have $s_i \geq \sqrt{ \sigma_{x,i,-i}} - n^{-1/4}$ with probability $\geq 1 - \kappa_i n^{-1/2}$. Then, for $n$ satisfying \eqref{eqn:ThmTestingProofeq33} and $\sqrt{ \sigma_{x,i,-i}} - n^{-1/4} >  D_\zeta \sqrt{ V_x} $, we get the bound with the above probability:
%
\begin{align}\label{eqn:ThmTestingProofeq34}
\frac{1}{\widehat s_i} \leq \frac{1}{\sqrt{ \sigma_{x,i,-i}} - n^{-1/4} - D_\zeta \sqrt{ V_x}}
\end{align}
%
Combining \eqref{eqn:ThmTestingProofeq31} and \eqref{eqn:ThmTestingProofeq34} gives the upper bound for the right hand side of \eqref{eqn:ThmTestingProofeq3} with the requisite probability and sample size conditions.

To prove \eqref{eqn:ThmTestingProofeq4} we have
%
\begin{align}\label{eqn:ThmTestingProofeq41}
\frac{1}{n} \| \bfR_i^T \bfX_{-i} \|_\infty & \leq
\frac{1}{n} \| (\bfX_i - \bfX_{-i} \bfzeta_{0,i})^T \bfX_{-i} \|_\infty +
\frac{1}{n} \| \bfX_{-i}^T \bfX_{-i} ( \widehat \bfzeta_i - \bfzeta_{0,i}) \|_\infty
\end{align}
%
Applying Lemma~\ref{lemma:ErrorRElemma2}, for $n \succsim \log(p-1)$ we have
%
%
\begin{align}
\frac{1}{n} \| (\bfX_i -  \bfX_{-i} \bfzeta_i )^T \bfX_{-i} \|_\infty \leq 
c_{7} [ \sigma_{x,i,-i} \Lambda_{\max} (\Sigma_{x, -i}) ]^{1/2} \sqrt{\frac{ \log (p-1)}{n}}
\end{align}
%
with probability $\geq 1 - 6c_{6} \exp [-(c_{7}^2-1) \log (p-1)]$ for some $c_{6} >0, c_{7} > 1$. By \eqref{eqn:ThmTestingProofeq32}, the second term on the right side of \eqref{eqn:ThmTestingProofeq41} is bounded above by $D_\zeta V_x$ with probability $ \geq 1 - 1/p^{\tau_1-2}$ and $n$ satisfying \eqref{eqn:ThmTestingProofeq33}. The bound of \eqref{eqn:ThmTestingProofeq4} now follows by conditions (T2), (T3) and \eqref{eqn:ThmTestingProofeq34}. Since $\sqrt{ \sigma_{x,i,-i}} - n^{-1/4} >  D_\zeta \sqrt{ V_x} $ implies $\sqrt{ \sigma_{x,i,-i}} >  D_\zeta \sqrt{ V_x} $, and $D_\zeta = O(\sqrt{ \log p/n})$, the leading term of the overall sample size requirement is $n \succsim \log (pq)$.
\end{proof}

\begin{proof}[Proof of Lemma~\ref{Lemma:PowerThmLemma}] We drop $k$ in the superscripts. By definition,
\begin{align}
\frac{m_i}{\sqrt n} &= \frac{1}{\widehat s_i} \frac{ ( \bfX_i - \bfX_{-i} \widehat \bfzeta_i)^T \bfX_i}{n} \notag\\
&= \frac{1}{\widehat s_i} \left[ \frac{\| \bfX_i - \bfX_{-i} \widehat \bfzeta_i \|^2}{n} +
 \frac{(\bfX_i - \bfX_{-i} \widehat \bfzeta_i)^T \bfX_{-i} \widehat \bfzeta_i}{n} \right] \notag\\
& \leq \widehat s_i + \frac{1}{\widehat s_i} .\frac{1}{n} \| \bfR_i^T \bfX_{-i} \|_\infty
\left (\| \widehat \bfzeta_i - \bfzeta_{0i} \|_1 + \| \bfzeta_{0i} \|_1 \right) \notag\\
\Rightarrow \left| \frac{m_i}{\sqrt n} - \sqrt{\sigma_{x,i,-i}} \right| & \leq
| \widehat s_i - \sqrt{\sigma_{x,i,-i}} | + \frac{1}{\widehat s_i} .\frac{1}{n} \| \bfR_i^T \bfX_{-i} \|_\infty
\left (\| \widehat \bfzeta_i - \bfzeta_{i} \|_1 + \| \bfzeta_{i} \|_1 \right) \label{eqn:PowerThmLemmaProofEqn1}
\end{align}
%
By applying Lemma 8 in \citet{RavikumarEtal11}, we have a bound for the first summand on the right hand side:
%
$$
| \widehat s_i - \sqrt{\sigma_{x,i,-i}} | \leq \sqrt{ \frac{ \log 4 + \tau_2}{c_i n}}; \quad
c_i = \left[ 128 (1 + 4 \sigma_{x,i,-i})^2 \sigma_{x,i,-i}^2 \right]^{-1},
$$
%
with probability $1 - 1/p^{\tau_2 - 2}$ for some $\tau_2>2$, and $n \geq 512 (1 + 4 \sigma_{x,i,-i})^42 \sigma_{x,i,-i}^4 \log(4) $. For the second summand in the right-hand side of \eqref{eqn:PowerThmLemmaProofEqn1}, $1/\widehat s_i$ can be bounded using \eqref{eqn:ThmTestingProofeq34}, $(1/n) \| \bfR_i^T \bfX_{-i} \|_\infty$ can be bounded using derivations following \eqref{eqn:ThmTestingProofeq41}. Finally, $\| \widehat \bfzeta_i - \bfzeta_{i} \|_1 \leq D_\zeta$ from assumption (T1), and $\| \bfzeta_i \|_1 \leq 1$ because $\Omega_x$ is diagonally dominant and $|\zeta_{ii'}| = |\omega_{x,ii'}|/ \omega_{x,ii}$ for $i' \neq i$. The lemma now follows by putting everything back together in \eqref{eqn:PowerThmLemmaProofEqn1}.
\end{proof}

\begin{proof}[Proof of Lemma~\ref{lemma:omega-diff}]
$\| \bfA - \bfA_1 \|_\infty \leq \delta$ implies that $\bfA_1 + \delta \bfJ_a \geq \bfA$ and $\bfA + \delta \bfJ_a \geq \bfA_1$, where $\bfJ_a \in \BM(a,a)$ has all entries 1, and for positive definite matrices $\bfP, \bfQ$, $\bfP \geq \bfQ$ means $\bfP - \bfQ$ is positive definite. Now applying Theorem 1 part (a) in \citet{Bellman68} we have
%
$$
(\bfA + \delta \bfJ_a)^{1/2} \geq \bfA_1^{1/2}; \quad
(\bfA_1 + \delta \bfJ_a)^{1/2} \geq \bfA^{1/2}.
$$
%
Using the same result, it is easy to prove that
%
$$ \bfA^{1/2} + \sqrt \delta \bfJ_a \geq (\bfA + \delta \bfJ_a)^{1/2}, $$
%
and the same for $\bfA_1$. The lemma follows.
\end{proof}

\begin{proof}[Proof of Lemma~\ref{Lemma:PowerThmLemma2}]
We drop $k$ in the superscripts and 0 in subscripts. Note that it is enough to prove
%
$$
\left\| \frac{n \widehat \Sigma_y}{(m_i)^2} - \frac{ \Sigma_y}{\sigma_{x,i,-i}} \right\|_\infty = o_P(1).
$$
%
For this, consider the decomposition
%
\begin{align*}
\frac{n \widehat \Sigma_y}{(m_i)^2} &= \frac{\widehat \Sigma_y - \Sigma_y + \Sigma_y}{\sigma_{x,i,-i}}.
\frac{\sigma_{x,i,-i}}{(m_i)^2/n}\\
\Rightarrow \frac{n \widehat \Sigma_y}{(m_i)^2} - \frac{ \Sigma_y}{\sigma_{x,i,-i}} &=
\frac{\widehat \Sigma_y - \Sigma_y}{(m_i)^2/n} +
\frac{\Sigma_y}{\sigma_{x,i,-i}} \left[ 1 - \frac{\sigma_{x,i,-i}}{(m_i)^2/n} \right]\\
& = \frac{n}{(m_i)^2} \left[ \widehat \Sigma_y - \Sigma_y +
\frac{\Sigma_y}{\sigma_{x,i,-i}} \left( \frac{(m_i)^2}{n} - \sigma_{x,i,-i} \right) \right].
\end{align*}
%
From Lemma~\ref{Lemma:PowerThmLemma} we now have
%
$$
\frac{m_i}{\sqrt n} \geq \sqrt{\sigma_{x,i,-i}} - \delta_i \quad \Rightarrow \quad
\frac{m_i^2}{n} \geq (\sqrt{\sigma_{x,i,-i}} - \delta)^2 \geq \sigma_{x,i,-i} - \delta_i^2,
$$
%
so that
%
\begin{align}\label{eqn:PowerThmLemma2ProofEqn1}
\left\| \frac{n \widehat \Sigma_y}{(m_i)^2} - \frac{ \Sigma_y}{\sigma_{x,i,-i}} \right\|_\infty \leq
\frac{ \| \widehat \Sigma_y - \Sigma_y \|_\infty + \sigma_{x,i,-i}^{-1} \delta_i^2 \| \Sigma_y \|_\infty }{\sigma_{x,i,-i} - \delta_i^2},
\end{align}
%
with probability $\geq 1 - 6c_{6} \exp [-(c_{7}^2-1) \log (p-1)] - 1/p^{\tau_2-2} - \kappa_i / \sqrt n$ and for sample size satisfying $n \succsim \log p$,  $n \geq 512 (1 + 4 \sigma_{x,i,-i})^42 (\sigma_{x,i,-i})^4 \log(4) $ and $\sqrt{ \sigma_{x,i,-i}} > \max \{ \delta_i, n^{-1/4} - D_\zeta \sqrt{V_x} \}$. For the $\ell_\infty$ norms on the right-hand side, we have
%
\begin{align}\label{eqn:PowerThmLemma2ProofEqn2}
\| \Sigma_y \|_\infty = \| \Omega_y^{-1} \|_\infty \leq (\Delta_0 (\Omega_y))^{-1}
\end{align}
%
following \citet{Varah75}. For a bound on $\| \widehat \Sigma_y - \Sigma_y \|_\infty$, if condition (II) of Theorem~\ref{thm:PowerThm} is satisfied then we have
%
\begin{align}\label{eqn:PowerThmLemma2ProofEqn31}
\| \widehat \Sigma_y - \Sigma_y \|_\infty \leq \tilde D_\Omega
\end{align}
%
where $\tilde D_\Omega = O(D_\Omega)$ and $D_\Omega = O(\tilde D_\Omega)$ \citet{BickelLevina08}. If condition (I) is satisfied, denote $\epsilon = D_\Omega / \Delta_0 (\Omega_y)$. Then
%
\begin{align}
\| \widehat \Sigma_y - \Sigma_y \|_\infty &= \| \widehat \Sigma_y (\Omega_y - \widehat \Omega_y) \Sigma_y \|_\infty \notag\\
& \leq \| \widehat \Sigma_y \|_\infty \| \Omega_y - \widehat \Omega_y \|_\infty
\| \Sigma_y \|_\infty \notag\\
& \leq \| (\bfI + (\Omega_y - \widehat \Omega_y) \Sigma_y)^{-1}\|_\infty \| \Sigma_y \|_\infty \epsilon \notag\\
& \leq \frac{\epsilon}{\Delta_0( \Omega_y)}
\left[ 1 + \sum_{t=1}^\infty (\| (\Omega_y - \widehat \Omega_y) \Sigma_y \|_\infty)^t \right] \notag\\
& \leq \frac{\epsilon}{(1- \epsilon) \Delta_0( \Omega_y)} \notag\\
& = \frac{D_\Omega}{(\Delta_0 (\Omega_y) - D_\Omega ) \Delta_0 (\Omega_y)}\label{eqn:PowerThmLemma2ProofEqn32}
\end{align}
%
Combining \eqref{eqn:PowerThmLemma2ProofEqn2} with \eqref{eqn:PowerThmLemma2ProofEqn31} or \eqref{eqn:PowerThmLemma2ProofEqn32} as required and putting them back in the right-hand side of \eqref{eqn:PowerThmLemma2ProofEqn1}, we get the needed.
\end{proof}

\begin{proof}[Proof of Lemma~\ref{lemma:ErrorRElemma1}]
This is the same as in Lemma 2 in Appendix B of \citet{LinEtal16} and its proof can be found there.
\end{proof}

\begin{proof}[Proof of Lemma~\ref{lemma:ErrorRElemma2}]
This is a part of Lemma 3 of Appendix B in \citet{LinEtal16}, and is proved therein.
\end{proof}