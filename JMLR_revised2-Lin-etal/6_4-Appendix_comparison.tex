\subsection{Numerical comparisons between different parametrizations.}

In this subsection, we provide some numerical evidence to substantiate the point we made in Section~\ref{sec:grouping}, that the two parametrizations are not always equivalent. This is a point also mentioned in the original work on
AMP graphs by \citet{andersson2001alternative}, the framework adopted in this paper. The other parametrization which we referred to as the \textit{$(\Omega_{XY},\Omega_Y)$-parametrization} corresponds to the LWF framework (\citet[see][p.34-35]{andersson2001alternative}).  In the presence of sparsity penalization, a specific sparsity pattern for the \textit{$(B,\Theta_\epsilon)$-parameterization} may not be recoverable through the \textit{$(\Omega_{XY},\Omega_Y)$-parametrization} and vice versa. 

Consider the following two simulation settings, in which the data are generated from the AMP framework ($(B,\Theta_\epsilon)$-parameterization) and the LWF framework (\textit{$(\Omega_{XY},\Omega_Y)$-parametrization}) respectively. 
\begin{itemize}
\item AMP framework. The data are generated according to the model $Y=XB^* + E$, similar to Model~A described in Section~\ref{sec: Implementation}; that is, each entry in $B^*$ is nonzero with probability $5/p_1$, and off-diagonal entries for $\Theta_\epsilon^*$ are nonzero with probability $5/p_2$.  Nonzero entries of $B^*$ and $\Theta_\epsilon^*$ are generated from $\mathsf{Unif}\left[(-1,-0.5)\cup(0.5,1)\right]$, and diagonals of $\Theta_\epsilon^*$ are set identical, such that the condition number of $\Theta_\epsilon^*$ is $p_2$. Table~\ref{table:AMP} shows the performance of estimated $B$ using different methods that are designed for different parameterizations: the node-conditional method (mixed MRF) and the proposed method
in this study (PML):
	\begin{table}[!h]
		\centering
		\caption{Performance for $\widehat{B}$ using different parameterizations}\label{table:AMP}
		\begin{tabular}{llccc}
			\hline
			$(p_1,p_2,n)$ 	  &  Method    & SEN   & SPC   & MCC \\ \hline
			$(30,60,100)$ & mixed MRF (th) & 0.86 & 0.71 & 0.45 \\ 
			& PML-th        & 0.96 & 0.99 & 0.93 \\  \hline		
			$(60,30,100)$ & mixed MRF (th)& 0.96 &  0.76 & 0.70 \\
			& PML-th &  0.99 & 0.99 & 0.93\\ \hline
			$(200,200,150)$ & mixed MRF (th)& 0.80 & 0.99  & 0.70  \\
			& PML-th &  0.99 & 0.99 & 0.88 \\ \hline
		\end{tabular}
	\end{table}
\item LWF framework. The data are generated based on the multivariate Normal specification:
\begin{equation*}
\begin{pmatrix}
\bs{X} \\ \bs{Y}
\end{pmatrix}\sim \mathcal{N}\left( 0, \begin{pmatrix}
\Omega_{X} & \Omega_{XY} \\ \Omega_{YX} & \Omega_{Y}
\end{pmatrix}^{-1} \right), 
\end{equation*}
Specifically, $\Omega_X$ is banded with 1 on the diagonal and 0.2 on the upper and lower first diagonal, $\Omega_Y$ is also banded with 1 on the diagonal and 0.3 on the upper and lower first diagonal. Each entry in $\Omega_XY$ is nonzero with probability $5/p_1$, and the nonzero entries are generated from $\mathsf{Unif}\left[(-1,-0.8)\cup (0.8,1)\right]$. Further, we bump up the diagonal of the joint precision matrix $\left[ \begin{smallmatrix} \Omega_X & \Omega_{XY} \\ \Omega'_{XY} & \Omega_Y \end{smallmatrix} \right]$ such that it is positive definite. Table~\ref{table:LWF} depicts the selection property of the estimated $\Omega_{XY}$ using different methods that are designed for different parameterizations: 
\begin{table}[!h]
	\centering
	\caption{Performance for $\widehat{\Omega}_{XY}$ using different parameterizations}\label{table:LWF}
	\begin{tabular}{llccc}
		\hline
		$(p_1,p_2,n)$ 	  &  Method    & SEN   & SPC   & MCC \\ \hline
		$(30,60,100)$ & mixed MRF & 0.84 & 0.88 & 0.63 \\ 
		& PML-th        & 0.99 & 0.52 & 0.39 \\  \hline		
		$(60,30,100)$ & mixed MRF & 0.847 &  0.95 & 0.70 \\
		& PML-th & 1 & 0.80 & 0.52 \\ \hline
		$(200,200,150)$ & mixed MRF & 0.89 &  0.93 & 		0.70 \\
		& PML-th &  1 & 0.79 & 0.30 \\ \hline
	\end{tabular}
\end{table}
\end{itemize}
Note that for both data generating procedures, the final estimates are thresholded at a proper level to retrieve meaningful results for comparison purposes. 

It can be seen that the method compatible with the data generation mechanism exhibits superior performance, vis-a-vis its competitor that was designed for another parameterization.
Further, the mixed MRF method suffers in terms of both sensititvity and specificity under the AMP parameterization, while the PML method suffers in terms of specificity only under
the LWF parameterization.