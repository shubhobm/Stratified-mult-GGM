\section{Appendix}
\subsection{Proofs for Main Theorems}\label{sec:proof-theorems}
\begin{proof}[\textbf{Proof of Theorem~\ref{thm:convergence}}]
	We initialize the algorithm at $(\widehat{B}^{(0)},\widehat{\Theta}^{(0)}_\epsilon)\in\dom(f)$. Then for all $k\geq 1$:
	\begin{eqnarray}
	\widehat{B}^{(k)} &=& \argmin\limits_{B} f(B,\widehat{\Theta}_\epsilon^{(k-1)}) \label{getB}\\
	\widehat{\Theta}^{(k)}_\epsilon & = & \argmin\limits_{\Theta_\epsilon} f(\widehat{B}^k,\Theta_\epsilon)\label{getTheta}
	\end{eqnarray} 
	Now, consider a limit point $(B^\infty,\Theta_\epsilon^\infty)$ of the sequence $\{(\widehat{B}^{(k)},\widehat{\Theta}_\epsilon^{(k)})\}_{k\geq 1}$. Note that such limit point exists by Bolzano-Weierstrass theorem since the sequence $\{(\widehat{B}^{(k)},\widehat{\Theta}_\epsilon^{(k)})\}_{k\geq 1}$ is bounded. Consider a subsequence $\mathcal{K}\subseteq\{1,2,\cdots\}$ such that $(\widehat{B}^{(k)},\widehat{\Theta}_\epsilon^{(k)})_{k\in \mathcal{K}}$ converges to $(B^\infty,\Theta_\epsilon^{\infty})$. Now for the bounded sequence $\{(\widehat{B}^{(k+1)},\widehat{\Theta}_\epsilon^{(k)})\}_{k\in\mathcal{K}}$, without loss of generality\footnote{switching to some further subsequence of $\mathcal{K}$ if necessary.}, we can say that 
	\begin{equation*}
	\{(\widehat{B}^{(k+1)},\widehat{\Theta}_\epsilon^{(k)})\}_{k\in\mathcal{K}}\rightarrow (\widetilde{B}^\infty,\widetilde{\Theta}_\epsilon^\infty), \quad \text{for some }(\widetilde{B}^\infty,\widetilde{\Theta}_\epsilon^\infty)\in\dom(f).
	\end{equation*}
	By (\ref{getB}) it follows immediately that $\widetilde{\Theta}_\epsilon^\infty = \Theta^\infty_\epsilon$. Also, the following inequality holds: 
	\begin{equation*}
	f(\widehat{B}^{(k+1)},\widehat{\Theta}_\epsilon^{(k+1)})\leq f(\widehat{B}^{(k+1)},\widehat{\Theta}_\epsilon^{(k)})\leq f(\widehat{B}^{(k)},\widehat{\Theta}_\epsilon^{(k)}).
	\end{equation*}
	Thus, by letting $k\rightarrow\infty$ over $\mathcal{K}$, we have 
	\begin{equation*}
	f(B^\infty,\Theta_\epsilon^\infty)\leq f(\widetilde{B}^\infty,\Theta_\epsilon^\infty)\leq f(B^\infty,\Theta_\epsilon^\infty),
	\end{equation*}
	since $f$ is continuous. This implies that 
	\begin{equation} \label{equal}
	f(\widetilde{B}^\infty,\Theta_\epsilon^\infty) = f(B^\infty,\Theta_\epsilon^\infty)
	\end{equation}
	Next, since $f(\widehat{B}^{(k+1)},\widehat{\Theta}_\epsilon^{(k)})\leq f(B,\widehat{\Theta}_\epsilon^{(k)})$, for all $B\in \mathbb{R}^{p_1\times p_2}$, let $k$ grow along $\mathcal{K}$, and we obtain the following:
	\begin{equation*}
	f(\widetilde{B}^\infty,\Theta_\epsilon^\infty)\leq f(B,\Theta_\epsilon^\infty),\quad \forall B\in \mathbb{R}^{p_1\times p_2}.
	\end{equation*}
	It then follows from (\ref{equal}) that 
	\begin{equation}\label{unequal-1}
	f(B^\infty,\Theta_\epsilon^\infty)\leq f(B,\Theta_\epsilon^\infty), \quad \forall B\in\mathbb{R}^{p_1\times p_2}.
	\end{equation}
	Finally, note that $f(\widehat{B}^{(k)},\widehat{\Theta}_\epsilon^{(k)})\leq f(\widehat{B}^{(k)},\Theta_\epsilon)$, for all $\Theta\in\mathbb{S}^{p_2\times p_2}_{++}$. As before, let $k$ grow along $\mathcal{K}$ and with the continuity of $f$, we obtain:
	\begin{equation}\label{unequal-2}
	f(B^\infty,\Theta_\epsilon^\infty) \leq f(B^\infty,\Theta_\epsilon), \quad \forall \Theta_\epsilon\in\mathbb{S}_{++}^{p_2\times p_2}. 
	\end{equation}
	Now, (\ref{unequal-1}) and (\ref{unequal-2}) together imply that $(B^\infty,\Theta_\epsilon^\infty)$ is a coordinate-wise minimum of $f$ and by Fact 1, 
	also a stationary point of $f$. 
\end{proof}

\medskip
\begin{proof}[\textbf{Proof of Theorem~\ref{thm:ErrorBound_beta}}]
	The statement of  Theorem~\ref{thm:ErrorBound_beta} is a variation of Proposition 4.1 in \citet{basu2015estimation}, and its proof follows directly from the proof of the
	proposition in \citet[Appendix B]{basu2015estimation}. We only outline how the statement differs. In the original statement of Proposition 4.1 in \citet{basu2015estimation}, the authors provide the error bound for $\bar{\beta}$, obtained as per (\ref{est-1.1}) whose dimension is $qp^2$ with $q$ denoting the true lag of the vector-autoregressive process, under an RE condition for $\bar{\Gamma}$ and a deviation bound for $(\bar{\gamma},\bar{\Gamma})$. For our problem, we impose a similar RE condition on $\widehat{\Gamma}$ and deviation bound on $(\widehat{\gamma},\widehat{\Gamma})$, so as to yield a bound on $\widehat{\beta}$ that lies in a $p_1p_2$-dimensional space. 
\end{proof}
\medskip

\begin{proof}[\textbf{Proof of Theorem~\ref{thm:ErrorBound_Theta}}]
	The statement of this theorem is a variation of Theorem~1 in \citet{ravikumar2011high}, so here, instead of providing a complete proof of the theorem, we only outline how the estimation problem differs in our setting, as well as the required changes in its proof.
	
	In \citet{ravikumar2011high}, the authors consider the optimization problem in (\ref{est-2.1}), and show that for a random realization, with certain sample size requirement and choice of the regularization parameter, the following bound for $\bar{\Theta}_\epsilon$ holds with probability at least $1-1/p_2^\tau$ for some $\tau>2$:
	\begin{equation}\label{glasso-bound}
	\|\bar{\Theta}_\epsilon -\Theta_\epsilon^*\|_\infty \leq \{2(1+8\xi^{-1})\kappa_{H^*}\}\bar{\delta}_f(p_2^\tau,n),
	\end{equation}
	where $\bar{\delta}(r,n)$ is defined as:
	\begin{equation}\label{delta-bar}
	\bar{\delta}(r,n): = 8(1+4\sigma^2)\max_i(\Sigma^*_{\epsilon,ii})\sqrt{\frac{2\log(4r)}{n}}.
	\end{equation}
	The quantity  $\bar{\delta}(p_2^\tau,n)$ that shows up in expression (\ref{glasso-bound}) is the bound for $\|S-\Sigma_\epsilon^*\|_\infty \equiv \|\widehat{\Sigma}_\epsilon - \Sigma_\epsilon^*\|_\infty$. In particular, in Lemma~8 \citep{ravikumar2011high}, they show that with probability at least $1-1/p_2^\tau$, $\tau>2$,  the following bound holds:
	\begin{equation*}
	\|S - \Sigma_\epsilon^*\|_\infty \leq \bar{\delta}(p_2^\tau,n).
	\end{equation*}
	In our optimization problem (\ref{est-2}), we are using $\widehat{S}$ instead of $S$, hence a bound for $\|\widehat{S}-\Sigma_\epsilon^*\|_\infty$ is necessary, and the remaining argument in the proof of Theorem~1 \citep{ravikumar2011high} will follow through. 
	
	Therefore in our theorem statement, we use $g(\nu_\beta)$ as a bound for $\|\widehat{S}-\Sigma_\epsilon^*\|_\infty$ then yield the bound for $\|\widehat{\Theta}_\epsilon-\Theta^*_\epsilon\|_\infty$, since we are using the surrogate error $\widehat{E}=Y-X\widehat{B}$ in the estimation, instead of the true error $E$. 
\end{proof}

\medskip
\begin{proof}[\textbf{Proof of Theorem~\ref{thm:beta-theta-bound}}] We first consider part (I) of the theorem. Note that by (\ref{eqn:initialB}), $\widehat{\beta}^{(0)}$ can be equivalently written as:
\begin{equation}\label{eqn:beta0}
\widehat{\beta}^{(0)} \equiv \argmin\limits_{\beta\in\mathbb{R}^{p_1\times p_2}}\left\{-2\beta'\gamma^0 + \beta'\Gamma^0\beta + \lambda_n^0\|\beta\|_1 \right\},
\end{equation}
where 
\begin{equation*}
\Gamma^{(0)} = \mathrm{I} \otimes \frac{X'X}{n}, \quad \gamma^{(0)} = (\mathrm{I} \otimes X')\mathrm{vec}{Y}/n.
\end{equation*}
Consider the following events:
\begin{itemize}
\item[\bf E1.] $\left\{\frac{X'X}{n}\sim RE(\varphi^*,\phi^*)\right\}$, 
\item[\bf E2.] $\left\{\frac{1}{n}\left\| X'E \right\|_\infty \leq c_2\left[\Lambda_{\max}(\Sigma_X^*)\Lambda_{\max}(\Sigma_\epsilon^*)\right]^{1/2}\sqrt{\frac{\log(p_1p_2)}{n}}\right\}$.
\end{itemize}
Note that \textbf{E1} $\cap$ \textbf{E2} implies the following events:
\begin{equation*}
\Gamma^{(0)} \equiv \mathrm{I}\otimes \frac{X'X}{n} \sim RE(\varphi^*,\phi^*), \quad \text{where }\varphi^*=\Lambda_{\min}(\Sigma_X^*)/2. 
\end{equation*}
and
\begin{equation}\label{bound:gamma0}
\|\gamma^{(0)} - \Gamma^{(0)}\beta^*\|_\infty = \frac{1}{n}\left\| X'E \right\|_\infty \leq c_2\left[\Lambda_{\max}(\Sigma_X^*)\Lambda_{\max}(\Sigma_\epsilon^*)\right]^{1/2}\sqrt{\frac{\log(p_1p_2)}{n}}.
\end{equation}
Hence, by Proposition~4.1 of \citet{basu2015estimation}, the bound \eqref{eqn:beta0bound} holds on \textbf{E1} $\cap$ \textbf{E2}.

\noindent By Lemmas~\ref{lemma:restateB.1} and ~\ref{lemma:RESX}, $\mathbb{P}(\mathbf{E1})$ is at least $1-2\exp(-c_3n)$, for some $c_3>0$. 
%\begin{equation*}
%\Gamma^{(0)} \equiv \mathrm{I}\otimes \frac{X'X}{n} \sim RE(\varphi^*,\phi^*), \quad \text{where }\varphi^*=\Lambda_{\min}(\Sigma_X^*)/2. 
%\end{equation*}
By Lemma~\ref{lemma:deviation_aux}, $\mathbb{P}(\mathbf{E2})$  is at least $1-6c_1\exp[-(c_2^2-1)\log(p_1p_2)]$ for some $c_1>0$, $c_2>1$.
%\begin{equation}\label{bound:gamma0}
%\|\gamma^{(0)} - \Gamma^{(0)}\beta^*\|_\infty = \frac{1}{n}\left\| X'E \right\|_\infty \leq c_2\left[\Lambda_{\max}(\Sigma_X^*)\Lambda_{\max}(\Sigma_\epsilon^*)\right]^{1/2}\sqrt{\frac{\log(p_1p_2)}{n}}.
%\end{equation}
Hence, with probability at least 
\begin{equation*}
\mathbb{P} \left(\mathbf{E1} \cap \mathbf{E2} \right) \ge 1 - \mathbb{P} \left(\mathbf{E1}^c \right) - \mathbb{P} \left(\mathbf{E2}^c \right)
\end{equation*}
the bound in (\ref{eqn:beta0bound}) holds, which proves the first part of (I). In particular, we have $\|\hat{\beta}^0 - \beta^*\|_1 \le \nu_\beta^{(0)}\sim O(\sqrt{\log(p_1p_2)/n})$ on $\mathbf{E1} \cap \mathbf{E2}$. 

\noindent To prove the second part of (I), note that by Theorem \ref{thm:ErrorBound_Theta} the bound in (\ref{eqn:Theta0bound}) holds when B1-B3 are satisfied. Now, from the argument above, B1 holds on the event \textbf{E1} $\cap$ \textbf{E2}. Also, from the proof of Proposition~\ref{prop:residual-concentration}, B2 is satisfied, i.e., 
\begin{equation}\label{eqn:S0prob}
\left\|\widehat{S}^{(0)} - \Sigma_\epsilon^*\right\|_\infty\leq g(\nu_\beta^{(0)}),\quad \text{where }\widehat{S}^{(0)}= \frac{1}{n}(Y-X\widehat{B}^{(0)})'(Y-X\widehat{B}^{(0)}),
\end{equation}
on \textbf{E1} $\cap$ \textbf{E2} $\cap$ \textbf{E3} $\cap$ \textbf{E4}, where the events \textbf{E3} and \textbf{E4} are given by:
\begin{itemize}
\item[\bf E3.] $\left\{  \left\|\frac{E'E}{n}-\Sigma^*_\epsilon\right\|_\infty \leq \sqrt{\frac{\log 4 + \tau_2 \log p_2}{c_\epsilon^* n}} \right\}$ for some $\tau_2>2$ and $c_\epsilon^*>0$ that depends on $\Sigma_{\epsilon}^*$,
\item[\bf E4.] $\left\{  \left\|\frac{X'X}{n}-\Sigma^*_X\right\|_\infty \leq \sqrt{\frac{\log 4 + \tau_1 \log p_1}{c_X^* n}} \right\}$ for some $\tau_1>2$ and $c_X*>0$ that depends on $\Sigma_{X}^*$.
\end{itemize}

Therefore, the probability of the bound for $\widehat{\Theta}_\epsilon^{(0)}$ in (\ref{eqn:Theta0bound}) to hold is at least
\begin{equation}\label{eqn:prob-intersection}
\mathbb{P}\left(\mathbf{E1}\cap \mathbf{E2} \cap \mathbf{E3} \cap \mathbf{E4}\right),
\end{equation}
By Lemma~\ref{lemma:RESX}, Lemma~\ref{lemma:deviation_aux} and the proof of Proposition~\ref{prop:residual-concentration}, the probability in (\ref{eqn:prob-intersection}) is lower bounded by:
\begin{equation*}
1-2\exp(-c_3n) - 6c_1\exp[-(c_2^2-1)\log (p_1p_2)]-1/p_1^{\tau_1-2}-1/p_2^{\tau_2-2}.
\end{equation*}

Consider the following two cases where the relative order of $p_1$ and $p_2$ differ. Case 1: $p_1\prec p_2$, then $\nu_\Theta^{(0)}\sim O(\sqrt{\log p_2/n})$; case 2: $p_1\succsim p_2$, then $\nu_\Theta^{(0)}\sim O\left(\log(p_1p_2)/n\right)$. In either case, since we are assuming $\log(p_1p_2)/n$ to be a small quantity and it follows that $\sqrt{\log(p_1p_2)/n}\succsim \log(p_1p_2)/n$, the following bound always holds: 
\begin{equation*}
\nu_\Theta^{(0)} \leq C_\Theta \sqrt{\frac{\log(p_1p_2)}{n}} \equiv  M_\Theta,
\end{equation*}
where $C_\Theta$ is some large fixed constant that bounds the constant terms in front of $\sqrt{\log(p_1p_2)/n}$.


Now we consider part (II) of the theorem. Note that for each $k\geq 1$, $\widehat{\beta}^{(k)}$ and $\widehat{\Theta}_\epsilon^{(k)}$ are obtained via solving the following two optimizations: 
\begin{eqnarray}
\widehat{\beta}^{(k)} & =& \argmin\limits_{\beta\in\mathbb{R}^{p_1\times p_2}} \left\{ -2\beta'\widehat{\gamma}^{(k-1)} + \beta'\widehat{\Gamma}^{(k-1)}\beta + \lambda_n\|\beta\|_1 \right\}, \\
\widehat{\Theta}_\epsilon^{(k)} & = & \argmin\limits_{\Theta_\epsilon\in\mathbb{S}_{++}^{p_2\times p_2}}\left\{ \log\det \Theta_\epsilon - \text{tr}(\widehat{S}^{(k)} \Theta_\epsilon) + \rho_n\|\Theta_\epsilon\|_{1,\text{off}}  \right\},
\end{eqnarray}
where 
\begin{equation*}
\widehat{\gamma}^{(k)} = \widehat{\Theta}^{(k)}\otimes \frac{X'Y}{n}, \quad \widehat{\Gamma}^{(k)} = \widehat{\Theta}^{(k)}\otimes \frac{X'X}{n}, \quad \widehat{S}^{(k)} = \frac{1}{n}(Y-X\widehat{B}^{(k)})'(Y-X\widehat{B}^{(k)}).
\end{equation*}
Consider the bound on $\hat{\beta}^{(k)}$ for $k=1$. The argument is similar to that of $\hat{\beta}^{(0)}$, with appropriate modifications to account for the fact that the objective function now involves log likelihood instead of least squares. Formally, we consider the event \textbf{E1} $\cap$ \textbf{E2} $\cap$ \textbf{E3} $\cap$ \textbf{E4} $\cap$ \textbf{E5}, where 
\begin{itemize}
\item[\bf E5.] $\left\{\frac{1}{n}\left\| X'E\Theta^*_\epsilon\right\|_\infty \leq c_2\left[\frac{\Lambda_{\max}(\Sigma^*_X)}{\Lambda_{\min}(\Sigma^*_\epsilon)}\right]^{1/2} \sqrt{\frac{\log (p_1p_2) }{n}}\right\}$.
\end{itemize}
Note that $\{\|\widehat{\Theta}^{(0)}_\epsilon-\Theta^*_\epsilon\|_\infty\leq \nu_\Theta^{(0)}\}$ holds on this event.  By Lemma~\ref{lemma:deviation_aux}, $\mathbb{P}(\textbf{E5}) \geq 1- 6c_1\exp[-(c_2^2-1)\log(p_1p_2)]$. Combining this with the lower bound on \eqref{eqn:prob-intersection} and 
% As in previous analyses for Theorem~\ref{thm:ErrorBound_beta}, Proposition~\ref{prop:REcondition}, Proposition~\ref{prop:deviation} and Lemma~\ref{lemma:deviation_aux}, if we take  as a deterministic event for now, the randomness in event $\{\|\widehat{\beta}_1 -\beta^*\|_1\leq \nu_\beta^{(1)}\}$ comes from the randomness in \textbf{E1}, \textbf{E2}, and the following event: 
%Now conditioning on the random event $\{\|\widehat{\Theta}^{(0)}-\Theta^*_\epsilon\|_\infty\leq \nu_\Theta^{(0)}\}$, based on the previous discussion, the only randomness that needs to be taken care of is that in event \textbf{E5}. By Lemma~\ref{lemma:deviation_aux}, $\mathbb{P}(\textbf{E5}) \geq 1- 6c_1\exp[-(c_2^2-1)\log(p_1p_2)]$. Combining $\mathbb{P}\left( \{\|\widehat{\Theta}^{(0)}-\Theta^*_\epsilon\|_\infty\leq \nu_\Theta^{(0)}\} \right)$ with 
the sample size requirement (note this sample size requirement can be relaxed to $n\succsim \log(p_1p_2)$ if $p_1\prec p_2$), we obtain that with probability at least
\begin{equation*}
1-1/p_1^{\tau_1-2} - 1/p_2^{\tau_2-2} - 12c_1\exp[-(c^2_2-1)\log (p_1p_2)] - 2\exp[-c_3n], 
\end{equation*}
the following three events hold simultaneously: 
\begin{itemize}
\item[] \textbf{\em A1'} $\|\widehat{\Theta}^{(0)}_\epsilon-\Theta^*_\epsilon\|_\infty\leq \nu_\Theta^{(0)}\precsim O(\sqrt{\log(p_1p_2)/n})$;
\item[] \textbf{\em A2'} $\widehat{\Gamma}^{(0)}\sim RE(\varphi^{(0)},\phi^{(0)})$ where
\begin{equation*}
\varphi^{(0)} \geq \frac{\Lambda_{\min}(\Sigma^*_X)}{2}(\min_i\psi^i - dM_\Theta)~~~\text{and} ~~~ \phi^{(0)} \leq \frac{\log p_1}{n}\frac{\Lambda_{\min}(\Sigma^*_X)}{2}(\max_j\psi^j + dM_\Theta);
\end{equation*}
\item[] \textbf{\em A3'} $\|\widehat{\gamma}^{(0)}-\widehat{\Gamma}^{(0)}\beta^*\|_\infty \leq \mathbb{Q}(\nu_\Theta^{(0)})\sqrt{\frac{\log(p_1p_2)}{n}}$ with the expression for $\mathbb{Q}(\cdot)$ given in (\ref{Q-expression}).
\end{itemize}
By Theorem~\ref{thm:ErrorBound_beta}, by choosing $\lambda_n\geq 4\mathbb{Q}(M_\Theta)\sqrt{\frac{\log(p_1p_2)}{n}}$, the following bound holds:
\begin{equation}\label{beta-bound}
\|\widehat{\beta}^{(1)}-\beta^*\|_1 \leq 64s^{**}\lambda_n/\varphi^{(0)}
\end{equation}

%Note that until this step, all sources of randomness have been taken into consideration; thus, for all subsequent iterations, the probability that the error bounds for $\widehat{\beta}^{(k)},\widehat{\Theta}_\epsilon^{(k)}$ hold no longer changes. 
The error bound for $\widehat{\Theta}^{(1)}_\epsilon$ can now be established using the same argument for $\hat{\Theta}^{(0)}_\epsilon$, with the only difference that now we consider the event $\mathbf{E1} \cap \ldots \cap \mathbf{E5}$ instead of $\mathbf{E1} \cap \ldots \cap \mathbf{E4}$ and use \eqref{beta-bound} instead of \eqref{eqn:beta0bound}.

Note that an upper bound for the leading term of the right hand side of (\ref{beta-bound}) is at most of the order $O(\sqrt{\log(p_1p_2)/n})$, and can be written as:
\begin{equation*}
C_\beta\left( s^{**}\sqrt{\frac{\log(p_1p_2)}{n}}\right)\equiv M_\beta,
\end{equation*}
with $C_\beta$ being some potentially large number that bounds the constant term. Notice that $M_\beta$ is of the same order as $\nu_\beta^{(0)}$;
thus, for $\widehat{\Theta}^{(1)}_\epsilon$, we can also achieve the following bound: 
\begin{equation*}
\|\widehat{\Theta}_\epsilon^{(1)}-\Theta^*_\epsilon\|_\infty\leq M_\Theta
\end{equation*}
with high probability since we are assuming $C_\Theta$ to be some potentially large number. 

Note that the events $\textbf{E1}, \ldots, \textbf{E5}$ rely only on the parameters and not on the estimated quantities, and on their intersection we have uniform upper bounds on the errors of $\widehat{\beta}^{(k)}$ and $\widehat{\Theta}^{k}_{\epsilon}$ for $k = 0, 1$. Hence the error bounds for $k=1$ can be used to invoke Theorems 2 and 3 inductively on realizations $X$ and $E$ from the set $\textbf{E1} \cap \ldots \cap \textbf{E5}$ to provide high probability error bounds for all subsequent iterates as well. This leads to the uniform error bounds of part (II) with the desired probability.

%By analogy, the bounds for $\widehat{\beta}^{(2)}$, $\widehat{\Theta}_\epsilon^{(2)}$ and all subsequent estimates remain unchanged. 
\end{proof}
 % use updated proof avoiding conditional statements

\begin{proof}[\textbf{Proof of Theorem~\ref{thm:FWER}}]
	First, we note that with a Bonferroni correction, the family-wise type I error will be automatically controlled at level $\alpha$. Hence, we will focus on the power of the screening step. Also, from Theorem 7 of \citet{javanmard2014confidence}, it is easy to see that all the arguments below hold for a large set of random realizations of $X$, whose probability approaches 1 under the specified asymptotic regime when the eigenvalues of $\Sigma_X$ are bounded away from $0$ and infinity.
	
	Let $B^*=\begin{bmatrix}
	B_1^* & \cdots & B_{p_2}^*
	\end{bmatrix}$ denote the true value of the regression coefficients and $\check{B}_j,j=1,\cdots,p_2$ denote the estimates given by the de-biased Lasso procedure in \citet{javanmard2014confidence}. With the given level for sparsity, by Theorem~8 in  \citet{javanmard2014confidence}, each $\check{B}_j$ satisfies the following:
	\begin{equation*}
	\sqrt{n}(\check{B}_j-B_j^*) = Z+\Delta, ,
	\end{equation*}
	where $Z\sim\mathcal{N}\left(0,\sigma^2 M_j\widehat{\Sigma}_XM_j'\right)$ and $\Delta$ vanishes asymptically. Here $\widehat{\Sigma}_X$ is the sample covariance matrix of the predictors $X$, $\sigma$ is the population noise level of the error term $\epsilon_j$, and $M_j$ is the matrix corresponding to the $j$th regression, produced by the procedure described in \citet{javanmard2014confidence}\footnote{Details of the procedure is described in p.2871 in \citet{javanmard2014confidence}, with $M$ being an intermediate quantity obtained by solving an optimization problem.}. Let $\check{B}_{j,i}$ denote the $i$th coordinate of the $j$th regression coefficient vector $\check{B}_j$ and $\check{\Sigma}_{j}$ be the covariance matrix of the estimator $\check{B}_j$, then 
	\begin{equation*}
	\check{\Sigma}_j = \frac{\sigma^2}{n}M_j\widehat{\Sigma}_XM_j',
	\end{equation*}
	and in particular, the variance of $\check{B}_{j,i}$ is $\check{\Sigma}_{j,ii}:=\check{\sigma}^j_{ii}$. Using these notations, for a prespecified level $\alpha$, the test statistics for testing $H^{ji}_0:B^*_{j,i}=0$ vs. $H^{ji}_A:B^*_{j,i}\neq 0$, for all $i=1,\cdots,p_1;j=1,\cdots,p_2$ can be equivalently written as: 
	\begin{equation*}
	\widehat{T}_{j,i} = \begin{cases}
	1 \quad &\text{if }|\check{B}_{j,i}|/\check{\sigma}^j_{ii}> z_{\alpha/(2p_1p_2)}, \\
	0 \quad &\text{otherwise.}
	\end{cases}
	\end{equation*}
	where $z_{\alpha}$ denotes the upper $\alpha$ quantiles of $\mathcal{N}(0,1)$. 
	
	Define the ``family-wise" power as follows: 
	\begin{equation*}
	\begin{split}
	\mathbb{P}\left( \text{all true alternatives are detected} \right) & = \mathbb{P}\left( \bigcap\limits_{1\leq j\leq p_2}\bigcap\limits_{k\in S^*_j}\{\widehat{T}_{j,k}=1\}\right) \\
	& = 1 - \mathbb{P}\left( \bigcup\limits_{1\leq j\leq p_2}\bigcup\limits_{k\in S^*_j}\{\widehat{T}_{j,k}=0\}\right).
	\end{split}
	\end{equation*}
	Correspondingly, the family-wise type II error can be  written as:
	\begin{equation}\label{type2}
	\mathbb{P}\left( \bigcup\limits_{1\leq j\leq p_2}\bigcup\limits_{k\in S^*_j}\{\widehat{T}_{j,k}=0\}\right) \leq \sum_{j=1}^{p_2}\sum_{k\in S^*_j} \mathbb{P}\left( \widehat{T}_{j,k}=0\right). 
	\end{equation}
	By Theorem~16 in \citet{javanmard2014confidence}, asymptotically, $\forall k\in S_j,j=1,\cdots,p_2$:
	\begin{equation}\label{powerbound}
	\begin{split}
	\mathbb{P}\left(\widehat{T}_{j,k}=0\right) & \leq 1 - G\left(\frac{\alpha}{p_1p_2},\frac{\sqrt{n}\gamma}{\sigma[\Sigma^{-1}_{k,k}]^{1/2}}\right); \qquad 0<\gamma\leq \min|B^*_{j,k}|,~~\forall k\in S_j,j=1,\cdots,p_2.
	\end{split}
	\end{equation}
	Here 
	\begin{equation*}
	G(\alpha,u) \equiv  2- \mathbb{P}(\Phi < z_{\alpha/2}+u) - \mathbb{P}(\Phi < z_{\alpha/2} - u),
	\end{equation*}
	where we use $\Phi$ to denote the random variable following a standard Gaussian distribution and the choice of $n$ in (\ref{powerbound}) doesn't depend on $k$. Hence, (\ref{powerbound}) can be re-written as:
	\begin{equation}\label{type2:ind}
	\begin{split}
	\mathbb{P}\left(\widehat{T}_{j,k}=0\right) & \leq 1 - G\left(\frac{\alpha}{p_1p_2},\frac{\sqrt{n}\gamma}{\sigma[\Sigma^{-1}_{k,k}]^{1/2}}\right)  \\
	& =   \mathbb{P}\left( \Phi < z_{\alpha/(2p_1p_2)} - \frac{\sqrt{n}\gamma}{\sigma[\Sigma^{-1}_{k,k}]^{1/2}}\right) - \mathbb{P}\left( \Phi > z_{\alpha/(2p_1p_2)} + \frac{\sqrt{n}\gamma}{\sigma[\Sigma^{-1}_{k,k}]^{1/2}}\right) \\
	& \leq  \mathbb{P}\left( \Phi >  \frac{\sqrt{n}\gamma}{\sigma[\Sigma^{-1}_{k,k}]^{1/2}}-z_{\alpha/(2p_1p_2)} \right),
	\end{split}
	\end{equation}
	where we use $\Phi$ to denote the random variable following a standard Gaussian distribution.
	
	Note that the following inequality holds for standard Normal percentiles:
	\begin{equation*}
	2e^{-t^2}\leq \mathbb{P}(|\Phi|>t) \leq e^{-t^2/2},
	\end{equation*}
	and by taking the inverse function, the following inequality holds:
	\begin{equation*}
	\sqrt{-\log \frac{y}{2}} \leq z_{y/2} \leq \sqrt{-2\log y}.
	\end{equation*}
	Letting $y=\frac{\alpha}{p_1p_2}$, it follows that:
	\begin{equation*}
	\left( -\log\frac{\alpha}{2p_1p_2}\right)^{1/2} \leq z_{\alpha/(2p_1p_2)}\leq \left( -2\log \frac{\alpha}{p_1p_2} \right)^{1/2},
	\end{equation*}
	hence
	\begin{equation*}
	\mathbb{P}\left(  \Phi >\frac{\sqrt{n}\gamma}{\sigma[\Sigma^{-1}_{k,k}]^{1/2}}-z_{\alpha/(2p_1p_2)}  \right)\leq \mathbb{P}\left(  \Phi >\frac{\sqrt{n}\gamma}{\sigma[\Sigma^{-1}_{k,k}]^{1/2}}-\sqrt{-2\log \frac{\alpha}{p_1p_2}}\right)
	\end{equation*}
	Now given 
	\begin{equation*}
	\frac{\log(p_1p_2) }{n} \rightarrow 0, 
	\end{equation*}
	the following expression follows:
	\begin{equation*}
	\frac{\sqrt{2\log\left(\frac{p_1p_2}{\alpha}\right)}}{\sqrt{n}/\sigma[\Sigma^{-1}_{k,k}]^{1/2}}\rightarrow 0,
	\end{equation*}
	indicating that for sufficiently large $n$, the following lower bound holds for some constant $c_0>0$:
\begin{equation*}
\left(\frac{\sqrt{n}\gamma}{\sigma[\Sigma^{-1}_{k,k}]^{1/2}}-\sqrt{-2\log \frac{\alpha}{p_1p_2}}\right) \geq c_0\sqrt{n}.
\end{equation*}
Note that $c_0$ is univeral for all choices of $k$, since this lower bound can be achieved by substituting $\Sigma^{-1}_{k,k}$ by $(1/\Lambda_{\min}(\Sigma_X))$, which is assumed to be bounded away from infinity.  Combined with the fact that $\mathbb{P}(\Phi >t)\leq e^{-t^2/2}$, the last expression in (\ref{type2:ind}) can thus be bounded by:
\begin{equation}\label{powerbound1}
	\mathbb{P}\left( \Phi >  \frac{\sqrt{n}\gamma}{\sigma[\Sigma^{-1}_{k,k}]^{1/2}}-z_{\alpha/(2p_1p_2)} \right) \leq \exp\left[ -\frac{1}{2} \left(\frac{\sqrt{n}\gamma}{\sigma[\Sigma^{-1}_{k,k}]^{1/2}}-\sqrt{-2\log \frac{\alpha}{p_1p_2}}\right)^2 \right] \leq  e^{-c_1 n}, 
\end{equation}
for some universal constant $c_1>0$, and the bound in (\ref{powerbound1}) holds uniformly for all $k\in S_j,\forall j$. Combine (\ref{type2}), (\ref{powerbound}) and (\ref{powerbound1}), it follows that 
\begin{equation}\label{powerbound2}
\mathbb{P}\left( \bigcup\limits_{1\leq j\leq p_2}\bigcup\limits_{k\in S^*_j}\{\widehat{T}_{j,k}=0\}\right) \leq s^*p_2 \exp(-c_1n).
\end{equation}
Now with $\log(p_1p_2)/n = o(1)$ and the given sparsity level, that is, $s^* = o(\sqrt{n}/\log p_1)$, it follows that:
	\begin{equation*}
	s^*p_2 \exp(-c_1n) = o(1),
	\end{equation*}
and by (\ref{powerbound2}), we have:
	\begin{equation*}
	\mathbb{P}\left(\text{family-wise type II error}\right) \rightarrow 0, \quad \Leftrightarrow \quad \mathbb{P}\left(\text{family-wise power}\right) \rightarrow 1.
	\end{equation*}
	This is equivalent to establishing that, given $\log(p_1p_2)/n\rightarrow 0$, the screening step recovers the true support sets $S_j^*$ for all $j=1,2,\cdots,p_2$ with high probability, while keeping the family-wise type I error rate under control. 
\end{proof}


